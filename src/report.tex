
\title{Report on the ITER Clite Shutdown Doserate Calculations}
\author{
  Andrew Davis \\
  Department of Engineering Physics\\
  College of Engineering \\
  The University of Wisconsin-Madison\\
  Madison, Wisconsin, 53706, \underline{USA}
  \and
  Mohamed Sawan \\
  Department of Engineering Physics\\
  College of Engineering \\
  The University of Wisconsin-Madison\\
  Madison, Wisconsin, 53706, \underline{USA}
  \and
  Paul P. H. Wilson \\
  Department of Engineering Physics\\
  College of Engineering \\
  The University of Wisconsin-Madison\\
  Madison, Wisconsin, 53706, \underline{USA}
  \and
  Elliott Biondo \\
  Department of Engineering Physics\\
  College of Engineering \\
  The University of Wisconsin-Madison\\
  Madison, Wisconsin, 53706, \underline{USA}
  \and
  Ahmed Ibrahim \\
  Radiation Transport Group\\
  Oak Ridge National Laboratory \\
  P.O. Box 2008 \\
  Oak Ridge, Tennessee 37831, \underline{USA}
  \and
  Patrick Shriwise \\
  Department of Engineering Physics\\
  College of Engineering \\
  The University of Wisconsin-Madison\\
  Madison, Wisconsin, 53706, \underline{USA}
}

\date{\today}

\documentclass[12pt]{article}
%\usepackage[printwatermark]{xwatermark}
%\usepackage{mathptmx}
%\usepackage{fouriernc}
\usepackage{times}
\usepackage{graphicx}
\usepackage[a4paper, portrait, margin=0.5in]{geometry}
\usepackage[table]{xcolor}
\usepackage[nottoc,numbib]{tocbibind}
\usepackage{subcaption}
\usepackage{multirow}
\usepackage{draftwatermark}
\SetWatermarkText{DRAFT}
\SetWatermarkScale{1}

%% define this page left blank
\newcommand*{\blankpage}{%
\vspace*{\fill}
\begin{center}
 \centering \textbf{This page intentionally left blank}
\end{center}
\vspace{\fill}}

%% to get lof in toc
\renewcommand{\listoffigures}{\begingroup
\tocsection
\tocfile{\listfigurename}{lof}
\endgroup}

%% to get lot in toc
\renewcommand{\listoftables}{\begingroup
\tocsection
\tocfile{\listtablename}{lot}
\endgroup}

\begin{document}
\maketitle
\newpage
\tableofcontents
\newpage
\listoffigures
\newpage
\listoftables
\newpage
\section*{Acronyms}
\newpage
\section*{Executive Summary}
The results contained within this report show the results of complex 3D neutron
\& photon transport simulations to determine the shutdown photon dose rate
resulting from the neutron activation of structural materials within a
representative model of the ITER device. There are several ongoing questions
regarding the correct level of SDR photon dose in the equatorial and upper port
regions of ITER.
\\
\\
The neutron results showed....
\\
\\
The activation results showed ....
\\
\\
The photon results showed ....
\\
\\
This analysis focused on only shutdown photon sources, and therefore does not
contain prompt photon doses.
\newpage
\blankpage
\newpage
\section*{Abstract}
This report provides the methology, input data, assumptions and results for a
full analysis of neutron, neutron induced activation and the subsequent
transport of residual decay photons in a reference ITER CAD model. The purpose
of the analysis was to (1) establish a baseline result for the shutdown photon
doserate around the ITER device and (2) determine the effect upon the SDR of a
thin B$_4$C layer added to the plasma-side of the bioshield. A differentiating
factor in this analysis relative to others is the level of detail present in
the ports. The UP contained a detailed diagnostic model along side detailed PI
equipment upto the bioshield plug. Similarly, the EP contained the DD model of
an EPP with diagnostic drawers and significant quantities of internal B$_4$C
shielding, the EP interspace contained detailed models of rails, racks and
support frames out to the bioshield. The LPP contained a detailed model of the
cryopump. 
\newpage
\section{Purpose}
\subsection{Problem Statement}
The shutdown doserate in an around the equatorial and upper ports determine the
type and duration of maintenance that can be performed by person access. Thus,
minimization of the dose rate is desirable and indeed encouraged by the ALARA
principles used at ITER. It was recently suggested in the Neutronics Task Force
that lining the plasma side of the concrete bioshield with a thin ($\sim$ mm
thickness) layer of boron carbide (B$_4$C). The purpose of the layer is to
absorb much of the thermal flux which would otherwise lead to neutron induced
activation, typically (n,$\gamma$) capture reactions. Simple scoping
calculations have suggested that the thermal neutron flux should be depressed
significantly and lead to signficantly lower shutdown photon doserates, close
to an order of magnitude. The main goal of this report was therefore to examine
if the use of the B$_4$C liner does indeed lead to lower doserates, this would
be achived by modelling a detailed 40$^{\circ}$ sector of the ITER device once
with the layer included and once without.
\subsection{Initiating Documents}
The ITER Task Agreement which this work was initiated and performed is TA XXXX
Several documents provided the basis of the materials for components, and
several CAD models were provided by numerous IO groups.
\\
\\
This report fulfills deliverable 6 of the task agreement.
\newpage
\section{Solution Methodology}
\subsection{Background}
\subsubsection{FW-CADIS Method}
\subsubsection{Rigorous Two Step (R2S) Method}
\subsection{Input Description}
\subsubsection{The Model Geometry}
The CAD model was generated from several sources, ITER Clite CAD which was used
as input to generate the ITER Clite V1 MCNP model, ITER Diagnostics Division
CAD models of Upper and Equatorial ports, and previous MCNP analysis which were
detailed in \cite{cad_origination}. The model represents several ITER systems
in high levels of detail, with all detail retained in the Equatorial Port
Interspace (EPI), the  Upper Port Interspace (UPI), and the Lower Port Cryopump.
Some of these models were previsouly used to generate MCNP input decks so have
undergone some degree of simplification.
\\
\\
The overall CAD model in shown in Figure \ref{fig:cad_iter_global}, the broad
details of the model can be seen. 
\begin{figure}[ht!]
  \centering
  \includegraphics[scale=0.8]{../plots/cad/global.png}
  \caption{Section showing the overall model}
  \label{fig:cad_iter_global}
\end{figure}

\begin{figure}[p]
  \centering
  \includegraphics[scale=0.32]{../plots/cad/mats/label_1.png}
  \includegraphics[scale=0.32]{../plots/cad/mats/label_2.png}
  \caption{Section some of the major tokamak materials}
  \label{fig:material_assign_1}
\end{figure}

\begin{figure}[p]
  \centering
  \includegraphics[scale=0.32]{../plots/cad/mats/label_3.png}
  \caption{Section some of the major tokamak materials}
  \label{fig:material_assign_2}
\end{figure}

\newpage
\clearpage
\subsubsection{Materials}
All materials in the problem originate from the Clite MCNP model -
CLITE\textunderscore V1\textunderscore REV131031\textunderscore MOD, with the
exception of the Upper Port Plugs (UPP), the Equatorial Port Plug (EPP), the
Cryopump, the blanket modules and the contents of the UPI and EPI. The default
cross section set used was FENDL-2.1, with the exception of XX which was
expanded into an isoptopic defintion and used the ENDF-VIIR1 evaluation. The EPP
material definitions were taken from the Generic Equatorial Port Plug (GEPP)
MCNP model \cite{epp_materials}. The UPP material definitions were provided by
\cite{bertalot_communication}. The Cryopump materials assignments came from
\cite{cryopump_communication}. The blanket module material compositions were
taken from the Bl-lite CAD model since the blankets were homogenized identically
in this model. The full evaluated material definitions can be found in the
Appendix.
\subsubsection{Variance Reduction}
The weight window parameters were generated using the Oak Ridge National
Laboratory (ORNL) code ADVANTG, specifically using DAG-ADVANTG, which allows a
DAGMC geometry to be read and the geometry and material data handed off to
Denovo. The weight windows used for the problem are shown below in Figure
\ref{fig:wwinp}. The weight windows were produced using the Forward Weighted
Consistent Adjoint Driven Importance Sampling (FW-CADIS), which attempts to
produce a weight window map which tries to get particles everywhere int the
model, as opposed to CADIS for example, which attempts to get results to a
small number of specific regions.
\begin{figure}[ht!]
  \centering
  \includegraphics[scale=0.4]{../plots/wwinp/wwinp_y0.png}
  \includegraphics[scale=0.3]{../plots/wwinp/wwinp_z0.png}
  \caption{Slices through the ADVANTG produced weight window through y = 0.0 cm
  and through z = 0.0 cm}
  \label{fig:wwinp}
\end{figure}
Several artifacts are worthy of note, despite being a deterministic code Denovo
has handled the streaming of neutrons along the divertor duct well as shown in
left hand image of Figure \ref{fig:wwinp}. The overall span of the weight window
is seven orders of magnitude, which is largely proportional to the neutron
attenuation from the plasma to the port interspace, previous calculations put
the attenuation in the range of 6 to 8 orders of magnitude. There is some
evidence of ray effects in the weight window solution in the right hand image
of Figure \ref{fig:wwinp} due to the neutron attenuation of some EPI shielding
around the diagnostic equipment, but this is small deviation in an otherwise
reasonable solution.
\\
\\
It should be noted that ORNL specifically developed the 360$^{\circ}$ rotation
feature for DAG-ADVANTG that allows rotationally symmetric bodies, which is the
reason for the weight window values beyond the 40$^{\circ}$ sector.
\subsubsection{Irradiation Parameters}
\subsection{Irradiation Scenario}
\begin{table}[ht!]
   \begin{tabular}{| l | c |}
      \hline 
      Irradiation Period & Fractional Strength \\
      \hline
      5 Years & 0.0095 \\
      1 Year  & 0.0127 \\
      1 Year  & 0.0190 \\
      1 Year  & 0.0317 \\
      1 Year  & 0.0380 \\
      1 Year  & 0.0380 \\
      6 Days  & 0.1929 \\
      1 Year  & 0.0380 \\
      \cellcolor{blue!25} 400 Seconds & 1.0000 \\
      \cellcolor{blue!25}1673.6 Seconds & 0.0000 \\
      400 Seconds & 1.0000 \\
      \hline
\end{tabular}
\caption{The table shows the ITER-DRG2 scenario for aggresive pulsing, note the
         cells in \textcolor{blue!25}{blue} are repeated 249 times}
\label{tag:irrad_scenario}
\end{table}

\newpage
\subsection{Software Programs \& Validation Status}
\subsubsection{DAGMC}
The Direct Accelerated Geometry Monte Carlo (DAGMC) is a toolkit developed at
the University of Wisconsin-Madison designed specifically for ray tracing
efficiently on complex CAD based geometries. The DAGMC toolkit is distributed as
part of the Mesh Orientated dABase (MOAB) and is completely open source. Input
to DAGMC is a MOAB mesh file which contains the faceted representation of the
CAD geometry, this is loaded and Oriented Bounding Box (OBB) acceleration
structures are built to speed each ray fire query.
\subsubsection{DAG-MCNP5}
DAG-MCNP5 \cite{dagmc} is a version of MCNP \cite{mcnp} where the core ray
queries (point in volume, ray fire, next volume, etc) have their MCNP versions
replaced with the DAGMC equivlents. DAG-MCNP5 has been validated
\cite{dagmc_validation} and tested in several ITER analysis in the past.
\subsubsection{PyNE}
The Python for Nuclear Engineering (PyNE) \cite{Scopatz2012b} project aims to
make C++ wrapped Python-accessible library of well validated and tested
functions and capabilities available to nuclear engineers across the world.
For this project we used the following capabilities of PyNE.
\begin{itemize}
  \item{the \texttt{pyne.mesh} class to combine meshes, with an appropriate
        combined statistical error}
  \item{the \texttt{pyne.r2s} class to produce R2S inputs and outputs}
  \item{the \texttt{pyne.alara} class to produce alara inputs}
  \item{the \texttt{pyne.mcnp} class to read MCNP meshtal files}
\end{itemize}
\subsubsection{ALARA}
AlARA \cite{alara} is a nuclear inventory code developed at the University of
Wisconsin-Madison. The unique feature of ALARA is the way it handles the
pathways of the reactions, all of the potential pathways are accounted for and
duplicate links linearised. ALARA is used in this analysis as the method of
producing gamma ray sources for R2S shutdown photon calculations.
\subsection{Solution Methodology}
The neutron transport calculations were performed using version 1.0 of DAG-MCNP5
using the source from version 1.60 of MCNP5, from the develop branch with git
checkout e12e2a3. The two batches of calculations were performed, one case with
the B 4 C liner and one without. Due to the large amount of RAM required to run
these jobs (in excess of 12 Gb per CPU) these jobs were considered a good case
to run using a high throughput methodology. In this method the spatial domain
of the problem is duplicated several times but the random number seed is
strided in such a way that no CPU will share the same seed with another job.
\\
\\
neutron transport, variance reduction meshes, photon transport many sources etc.
\begin{figure}[ht!]
  \centering
  \includegraphics[scale=0.4]{../plots/transport/job_splits.png}
  \caption{The seven different mesh domains, (left) shown with some transparency
           to indicate where the major boundaries end and (right) showing the
           absolute scale of the mesh elements}
  \label{fig:mesh_domains}
\end{figure}

The calculation was split into seven calculation domains, shown in Figure
\ref{fig:mesh_domains}, which bound the regions of the problem requiring
activation. The physical size and boundaries that were used in the calculation
can be found in Table \ref{table:mesh_sizes}. With R2S calculations there is a
subtle interplay between the size of the mesh elements and the number of mesh
elements; the larger the mesh elements the lower the statistical errors will be
for a given fixed number of particles but the mesh will conform poorly to the
geometry, and vice versa. In this calculation the maximum number of mesh
elements in a given mesh is dictated by the size that the given mesh will
occupy in memory when transferred to an ALARA geometry. 

\begin{centering}
 \begin{table}[ht!]
  \begin{tabular}{c | c | c | c | c}
  \hline 
  Mesh & Dimension & Start position (cm) & End position (cm) & Number of bins\\
  \hline 
  \multirow{3}{*}{Mesh 1} & X & 0.0 & 500.0 & 25 \\ & Y & -250.0 & -250.0 & 25 \\
  & Z & -1500.0 & 1900.0 & 170 \\
  \hline
  \multirow{3}{*}{Mesh 2} & X & 500.0 & 1100.0 & 30 \\ & Y & -500.0 & 500.0 & 50\\
  & Z & 400.0 & 1900.0 & 75 \\
  \hline
  \multirow{3}{*}{Mesh 3} & X & 500.0 & 1100.0 & 30 \\ & Y & -500.0 & 500.0 & 50 \\
  & Z & -1500.0 & -300.0 & 60 \\
  \hline
  \multirow{3}{*}{Mesh 4} & X & 500.0 & 1100.0 & 30 \\ & Y & -500.0 & 500.0 & 50 \\
  & Z & -300.0 & 400.0 & 35 \\
  \hline
  \multirow{3}{*}{Mesh 5} & X & 1100.0 & 1700.0 & 30 \\ & Y & -600.0 & 600.0 & 60 \\
  & Z & -1500.0 & -300.0 & 60 \\
  \hline
  \multirow{3}{*}{Mesh 6} & X & 1100.0 & 1700.0 & 60 \\ & Y & -600.0 & 600.0 & 60 \\
  & Z & 400.0 & 1900.0 & 75\\
  \hline
  \multirow{3}{*}{Mesh 7} & X & 1100.0 & 1700.0 & 30 \\ & Y & -600.0 & 600.0 & 60 \\
  & Z & -300.0 & 400.0 & 36 
  \end{tabular}
 \caption{The meshes used in the problem, their start and end coordinate and
          number of divisions}
 \label{table:mesh_sizes}
 \end{table}
\end{centering}

\subsection{Software Risk Assesment}
\newpage
\section{Assumptions and Engineering Judgements}
\subsection{Source}
\subsection{Irradiation Data}
\subsection{Geometry}
\subsection{Variance Reduction}
\subsection{Materials and Nuclear Data}
\subsection{Transport Software}
\subsection{Activation Software}

\section{Neutron Transport Results and Conclusions}


Having been split appropriately, these jobs were submited to the University of
Wisconsin's Centre for High Throughput Computer (CHTC) system in batches of 200
for each mesh, with seven meshes used to determine the neutron flux, making a
total of 2800 individual MCNP simulations. These results were then combined
using PyNE to result in the statistically averaged final results. Intially
calculations were batched into groups of $5\times10^7$ particles per batch,
however these calculations did not finish in a timely enough manner and were
instead reduced to groups of $5\times10^6$ particles per batch. With the smaller
batching system calculations did finish, however, only 10\% of all calculations
for each mesh finished, but this represents $5\times10^7$ total source histories
for each mesh. 
\newpage
\clearpage
\subsection{Without the B4C Liner}
The baseline results represent a standard C-lite model since there is no B$_4$C
liner present, however relative to the standard C-lite native MCNP model there
is much more equipment present in the port inspaces (both upper and lower), and
the divertor pumping port is unplugged. 
\begin{figure}[ht!]
  \centering
  \includegraphics[scale=0.35]{../plots/neutron/nob4c/y_0.png}
  \includegraphics[scale=0.35]{../plots/neutron/nob4c/y_-17.png}
  \caption{Slices through the total neutron flux (n cm$^{-2}$ src$^{-1}$)
  y = -17.0 cm and through z = 0.0 cm}
  \label{fig:wwinp}
\end{figure}
\begin{figure}[ht!]
  \centering
  \includegraphics[scale=0.27]{../plots/neutron/nob4c/z_-500.png}
  \includegraphics[scale=0.27]{../plots/neutron/nob4c/z_0.png}       
  \includegraphics[scale=0.27]{../plots/neutron/nob4c/z_500.png}
  \caption{Slices through the total neutron flux (n cm$^{-2}$ src$^{-1}$)
  z = -500.0 cm and through z = 500.0 cm}
  \label{fig:wwinp}
\end{figure}
\newpage
\clearpage
\subsection{Including the B4C Liner}
These results represent a standard C-lite model with improved B$_4$C liner,
however relative to the standard C-lite native MCNP model there is much more
equipment present in the port inspaces (both upper and lower), and the divertor
pumping port is unplugged. 
\begin{figure}[ht!]
  \centering
  \includegraphics[scale=0.35]{../plots/neutron/b4c/flux_y0.png}     
  \includegraphics[scale=0.35]{../plots/neutron/b4c/flux_y-17.png}
  \caption{Slices through the total neutron flux (n cm$^{-2}$ src$^{-1}$)
           y = -17.0 cm and through z = 0.0 cm}
  \label{fig:wwinp}
\end{figure}
\begin{figure}[ht!]
  \centering
  \includegraphics[scale=0.27]{../plots/neutron/b4c/flux_z-500.png}
  \includegraphics[scale=0.27]{../plots/neutron/b4c/flux_z0.png}       
  \includegraphics[scale=0.27]{../plots/neutron/b4c/flux_z500.png}
  \caption{Slices through the total neutron flux (n cm$^{-2}$ src$^{-1}$)
           z = -500.0 cm and through z = 500.0 cm}
  \label{fig:wwinp}
\end{figure}
\subsection{Conclusion}
The final calulcations represent only around 10\% of the overall requested
runtime, this is mostly due to calculations not completing with the 3 day
allotted runtime on CHTC, the reason being that is seems the requested VR
parameters were too extreme resulting in severe oversplitting of particles.
This has been noticed to occur in several previous challenging attenuating
geometries and has been attributed to so called long histories. The long history
problem requires an ultimate solution beyond the scope of this report, but
perhaps solutions like that already underway at UW-Madison \& ORNL; GT-CADIS
\& MS-CADIS respectively.
\newpage
\clearpage

\section{Neutron Activation Results and Conclusions}
The R2S shutdown dose rate calculations were performed using the pyne.r2s
methods found PyNE. The setup scripts produce ALARA inputs and the neutron
activation calculations were performed using ALARA 2.9.1RC which used the
FENDL-3.0/A activation cross sections. 
\subsection{Including the B4C Liner}
\subsection{Without the B4C Liner}
\subsection{Conclusion}
\newpage
\section{Shutdown Photon Doserates Results and Conclusions}
The shutdown photon sources were developed in several seperate meshes and
therefore any individual mesh does not represent the entire problem. The
results of the previous ALARA calculations are then processed further to make
shutdown sources to be read into a sampling subroutine (also distributed with
PyNE) Each source is run independently, and the photon dose is recorded on a
mesh common to each photon calculation and then the meshes summed together to
create the final result. The final mesh has a uniform size of side 2 cm striding
from x = {0,1500}, y = {-500,500}, z = {-1900,1500} cm. The meshes used the
ICRP-74 dose response coeffiecients as a dose multiplier as recommended in
\cite{iter_sdr_coeffs}.
\subsection{Decay time 1 - 1.$\times$10$^5$ s}
\subsection{Decay time 2 - 1.$\times$10$^6$ s}
\subsection{Decay time 3 - 1.$\times$10$^7$ s}
\subsection{Conclusion}
It is clear from Figures XXX that the shutdown photon doserate in the port
interspaces is affected by the addition of boron carbide to the plasma side
surface of the bioshield. The reason for the this is significantly degraded
thermal flux which impacts the inportance of (n,$\gamma$) reactions and also
the transport of neutrons back from the bioshield. The true benefit appears
only to be very close to the B$_4$C layer, where is reduces the activation of
steel componenets near to the bioshield, those components away from the
bioshield appear to have their activation dominated by the neutron tranported
via long paths around other ports, known as cross talk.
\section{Calculation Details}
\subsection{Neutron Transport}
\subsubsection{Baseline}
\subsubsection{With B$_4$C}
\subsection{Activation Calculations}
\subsubsection{Baseline}
\subsubsection{With B$_4$C}
\subsection{Photon Transport}
\subsubsection{Baseline}
\subsubsection{With B$_4$C}

\section{Acceptance Criteria}
\section{Lessons learned and future work}
\section{Acknowledgements}

\bibliographystyle{unsrt}
\bibliography{bibliography}
\newpage
\section{Appendix}
\begin{centering}
\begin{longtable}[ht!]
{ p{0.3\textwidth} | p{0.3\textwidth} }
\hline
Nuclide & Atom Fraction\\
\hline
H1 & 6.997078e-04\\
H2 & 1.608276e-07\\
B10 & 1.831598e-06\\
B11 & 8.106036e-06\\
C12 & 2.946747e-04\\
N14 & 6.928888e-04\\
N15 & 2.769410e-06\\
O16 & 5.539577e-03\\
Si28 & 4.564952e-03\\
Si29 & 2.400762e-04\\
Si30 & 1.637057e-04\\
P31 & 2.484376e-04\\
S & 9.937764e-05\\
Ti46 & 7.870667e-05\\
Ti47 & 7.252245e-05\\
Ti48 & 7.335356e-04\\
Ti49 & 5.497666e-05\\
Ti50 & 5.371156e-05\\
Cr50 & 7.258372e-03\\
Cr52 & 1.455608e-01\\
Cr53 & 1.682310e-02\\
Cr54 & 4.266560e-03\\
Mn55 & 1.788766e-02\\
Fe54 & 3.637981e-02\\
Fe56 & 5.922114e-01\\
Fe57 & 1.392136e-02\\
Fe58 & 1.885133e-03\\
Co59 & 4.968802e-04\\
Ni58 & 8.180318e-02\\
Ni60 & 3.259580e-02\\
Ni61 & 1.440563e-03\\
Ni62 & 4.668329e-03\\
Ni64 & 1.227273e-03\\
Cu63 & 2.042160e-03\\
Cu65 & 9.391214e-04\\
Nb93 & 9.937617e-05\\
Mo92 & 3.532189e-03\\
Mo94 & 2.249541e-03\\
Mo95 & 3.912899e-03\\
Mo96 & 4.142835e-03\\
Mo97 & 2.396706e-03\\
Mo98 & 6.118223e-03\\
Mo100 & 2.491646e-03\\
Ta181 & 9.937573e-05\\
\caption{Table showing the isotopic description of material EPP3L}
\label{table:material_EPP3L}
\end{longtable}
\clearpage
\begin{longtable}[ht!]
{ p{0.3\textwidth} | p{0.3\textwidth} }
\hline
Nuclide & Mass Fraction\\
\hline
He4 & 3.355146e-03\\
B10 & 1.802792e-04\\
B11 & 7.978553e-04\\
C12 & 1.970612e-04\\
N14 & 9.160305e-04\\
N15 & 3.624343e-06\\
O16 & 1.378506e-02\\
Al27 & 6.199031e-04\\
Si28 & 1.295973e-02\\
Si29 & 6.826361e-04\\
Si30 & 4.652330e-04\\
P31 & 2.955888e-04\\
S & 1.970642e-04\\
K & 7.614452e-05\\
Ti46 & 1.096136e-04\\
Ti47 & 1.010008e-04\\
Ti48 & 1.022010e-03\\
Ti49 & 7.656520e-05\\
Ti50 & 7.480316e-05\\
V & 2.627356e-05\\
Cr50 & 4.743983e-03\\
Cr52 & 9.513602e-02\\
Cr53 & 1.099539e-02\\
Cr54 & 2.788589e-03\\
Mn55 & 1.313740e-02\\
Fe54 & 2.407534e-02\\
Fe56 & 3.919107e-01\\
Fe57 & 9.212806e-03\\
Fe58 & 1.247545e-03\\
Co59 & 3.284351e-04\\
Ni58 & 5.296795e-02\\
Ni60 & 2.110594e-02\\
Ni61 & 9.327718e-04\\
Ni62 & 3.022769e-03\\
Ni64 & 7.946654e-04\\
Cu63 & 1.898430e-01\\
Cu65 & 8.730242e-02\\
Zr & 1.313594e-05\\
Nb93 & 2.401275e-02\\
Mo92 & 2.334761e-03\\
Mo94 & 1.486929e-03\\
Mo95 & 2.586387e-03\\
Mo96 & 2.738387e-03\\
Mo97 & 1.584213e-03\\
Mo98 & 4.044120e-03\\
Mo100 & 1.646961e-03\\
Sn & 7.995497e-03\\
Ta181 & 6.052439e-03\\
W182 & 1.724655e-06\\
W183 & 9.367748e-07\\
W184 & 2.016320e-06\\
W186 & 1.890873e-06\\
Pb206 & 1.276447e-06\\
Pb207 & 1.176209e-06\\
Pb208 & 2.802325e-06\\
Bi209 & 5.254959e-06\\
\caption{Table showing the isotopic description of material M908}
\label{table:material_M908}
\end{longtable}\clearpage

\begin{longtable}[ht!]
{ p{0.3\textwidth} | p{0.3\textwidth} }
\hline
Nuclide & Mass Fraction\\
\hline
\\
B10 & 3.318107e-06\\
B11 & 1.468481e-05\\
C12 & 2.965728e-04\\
C13 & 3.475851e-06\\
N14 & 1.095890e-03\\
N15 & 4.288673e-06\\
Si28 & 9.188137e-03\\
Si29 & 4.834412e-04\\
Si30 & 3.300419e-04\\
P31 & 3.000486e-04\\
S32 & 1.420960e-04\\
S33 & 1.156998e-06\\
S34 & 6.754457e-06\\
S36 & 1.682823e-08\\
Ti46 & 7.921378e-05\\
Ti47 & 7.298965e-05\\
Ti48 & 7.385702e-04\\
Ti49 & 5.533087e-05\\
Ti50 & 5.405755e-05\\
Cr50 & 8.348725e-03\\
Cr52 & 1.674258e-01\\
Cr53 & 1.935031e-02\\
Cr54 & 4.907523e-03\\
Mn55 & 2.000324e-02\\
Fe54 & 3.613009e-02\\
Fe56 & 5.881456e-01\\
Fe57 & 1.382579e-02\\
Fe58 & 1.872207e-03\\
Co59 & 1.000162e-03\\
Ni58 & 8.065044e-02\\
Ni60 & 3.213624e-02\\
Ni61 & 1.420261e-03\\
Ni62 & 4.602659e-03\\
Ni64 & 1.209846e-03\\
Nb93 & 1.000162e-03\\
Mo92 & 6.959281e-04\\
Mo94 & 4.477765e-04\\
Mo95 & 7.834282e-04\\
Mo96 & 8.331563e-04\\
Mo97 & 4.848117e-04\\
Mo98 & 1.244427e-03\\
Mo100 & 5.112820e-04\\
Ta180 & 1.194554e-08\\
Ta181 & 1.000043e-04\\
\caption{Table showing the isotopic description of material CryoPipes}
\label{table:material_CryoPipes}
\end{longtable}\clearpage

\begin{longtable}[ht!]
{ p{0.3\textwidth} | p{0.3\textwidth} }
\hline
Nuclide & Mass Fraction\\
\hline
H1 & 5.004784e-06\\
H2 & 1.150349e-09\\
C12 & 2.968774e-05\\
N14 & 9.972407e-06\\
N15 & 3.985867e-08\\
O16 & 2.995422e-05\\
Na23 & 1.001186e-05\\
Mg & 5.005930e-06\\
Al27 & 1.501778e-05\\
Si28 & 1.839627e-05\\
Si29 & 9.674816e-07\\
Si30 & 6.597175e-07\\
P31 & 5.005883e-05\\
S & 5.006025e-06\\
K & 1.001096e-05\\
Ca & 1.001235e-05\\
Ti46 & 7.929504e-07\\
Ti47 & 7.306453e-07\\
Ti48 & 7.390152e-06\\
Ti49 & 5.538772e-07\\
Ti50 & 5.411284e-07\\
Cr50 & 4.178642e-07\\
Cr52 & 8.379937e-06\\
Cr53 & 9.685066e-07\\
Cr54 & 2.456255e-07\\
Mn55 & 5.005959e-06\\
Fe54 & 1.695680e-06\\
Fe56 & 2.760323e-05\\
Fe57 & 6.488791e-07\\
Fe58 & 8.786698e-08\\
Co59 & 1.001190e-05\\
Ni58 & 1.345544e-05\\
Ni60 & 5.361537e-06\\
Ni61 & 2.369519e-07\\
Ni62 & 7.678737e-07\\
Ni64 & 2.018687e-07\\
Cu63 & 6.858056e-06\\
Cu65 & 3.153799e-06\\
Zr & 1.001074e-05\\
Nb93 & 1.001189e-05\\
Mo92 & 1.423431e-05\\
Mo94 & 9.065397e-06\\
Mo95 & 1.576854e-05\\
Mo96 & 1.669515e-05\\
Mo97 & 9.658474e-06\\
Mo98 & 2.465584e-05\\
Mo100 & 1.004101e-05\\
Ta181 & 1.001185e-05\\
W182 & 2.624604e-01\\
W183 & 1.425122e-01\\
W184 & 3.068133e-01\\
W186 & 2.877791e-01\\
Pb206 & 2.398459e-06\\
Pb207 & 2.210035e-06\\
Pb208 & 5.265464e-06\\
\caption{Table showing the isotopic description of material M74}
\label{table:material_M74}
\end{longtable}\clearpage

\begin{longtable}[ht!]
{ p{0.3\textwidth} | p{0.3\textwidth} }
\hline
Nuclide & Mass Fraction\\
\hline
\\
B10 & 1.843098e-06\\
B11 & 8.156930e-06\\
C12 & 2.965249e-04\\
N14 & 6.972393e-04\\
N15 & 2.786798e-06\\
Si28 & 4.593614e-03\\
Si29 & 2.415835e-04\\
Si30 & 1.647335e-04\\
P31 & 2.499975e-04\\
S & 1.000016e-04\\
Ti46 & 7.920085e-05\\
Ti47 & 7.297781e-05\\
Ti48 & 7.381413e-04\\
Ti49 & 5.532184e-05\\
Ti50 & 5.404879e-05\\
Cr50 & 7.303946e-03\\
Cr52 & 1.464747e-01\\
Cr53 & 1.692872e-02\\
Cr54 & 4.293349e-03\\
Mn55 & 1.799997e-02\\
Fe54 & 3.660823e-02\\
Fe56 & 5.959297e-01\\
Fe57 & 1.400876e-02\\
Fe58 & 1.896969e-03\\
Co59 & 4.999999e-04\\
Ni58 & 8.231680e-02\\
Ni60 & 3.280046e-02\\
Ni61 & 1.449607e-03\\
Ni62 & 4.697640e-03\\
Ni64 & 1.234979e-03\\
Cu63 & 2.054981e-03\\
Cu65 & 9.450179e-04\\
Nb93 & 1.000001e-04\\
Mo92 & 3.554367e-03\\
Mo94 & 2.263664e-03\\
Mo95 & 3.937466e-03\\
Mo96 & 4.168845e-03\\
Mo97 & 2.411755e-03\\
Mo98 & 6.156638e-03\\
Mo100 & 2.507290e-03\\
Ta181 & 9.999969e-05\\

\caption{Table showing the isotopic description of material EppDucts}
\label{table:material_EppDucts}
\end{longtable}\clearpage

\begin{longtable}[ht!]
{ p{0.3\textwidth} | p{0.3\textwidth} }
\hline
Nuclide & Mass Fraction\\
\hline
H1 & 6.386261e-04\\
He4 & 3.220044e-03\\
B10 & 1.154247e-06\\
B11 & 5.108293e-06\\
C12 & 6.839086e-03\\
N14 & 1.718415e-03\\
N15 & 3.455416e-06\\
O16 & 1.269236e-02\\
Mg & 8.473257e-04\\
Al27 & 3.419589e-03\\
Si28 & 1.035880e-02\\
Si29 & 5.471327e-04\\
Si30 & 3.740661e-04\\
P31 & 2.818109e-04\\
S & 6.669566e-04\\
K & 3.130979e-06\\
Ti46 & 9.536886e-04\\
Ti47 & 8.787528e-04\\
Ti48 & 8.891961e-03\\
Ti49 & 6.661519e-04\\
Ti50 & 6.508222e-04\\
V & 2.504891e-05\\
Cr50 & 4.443454e-03\\
Cr52 & 8.910932e-02\\
Cr53 & 1.029885e-02\\
Cr54 & 2.611948e-03\\
Mn55 & 1.252507e-02\\
Fe54 & 2.295321e-02\\
Fe56 & 3.736446e-01\\
Fe57 & 8.783409e-03\\
Fe58 & 1.189398e-03\\
Co59 & 3.131276e-04\\
Ni58 & 5.317565e-02\\
Ni60 & 2.119099e-02\\
Ni61 & 9.412247e-04\\
Ni62 & 3.036404e-03\\
Ni64 & 8.081852e-04\\
Cu63 & 2.102419e-01\\
Cu65 & 9.700064e-02\\
Zr & 1.252369e-05\\
Nb93 & 1.828886e-02\\
Mo92 & 2.225932e-03\\
Mo94 & 1.417623e-03\\
Mo95 & 2.465848e-03\\
Mo96 & 2.610754e-03\\
Mo97 & 1.510374e-03\\
Mo98 & 3.855621e-03\\
Mo100 & 1.570197e-03\\
Sn & 1.252512e-05\\
Ta181 & 6.262504e-05\\
W182 & 1.644269e-06\\
W183 & 8.931100e-07\\
W184 & 1.922345e-06\\
W186 & 1.802741e-06\\
Pb206 & 1.216953e-06\\
Pb207 & 1.121388e-06\\
Pb208 & 2.671709e-06\\
Bi209 & 5.010036e-06]\\

\caption{Table showing the isotopic description of material M906}
\label{table:material_M906}
\end{longtable}\clearpage

\begin{longtable}[ht!]
{ p{0.3\textwidth} | p{0.3\textwidth} }
\hline
Nuclide & Mass Fraction\\
\hline
\\
He4 & 3.098619e-03\\
B10 & 1.699068e-04\\
B11 & 7.519497e-04\\
C12 & 2.047632e-04\\
N14 & 9.518301e-04\\
N15 & 3.765986e-06\\
O16 & 1.298416e-02\\
Al27 & 6.157792e-04\\
Si28 & 1.264583e-02\\
Si29 & 6.661027e-04\\
Si30 & 4.539647e-04\\
P31 & 3.071415e-04\\
S & 2.047662e-04\\
K & 7.203318e-05\\
Ti46 & 1.102498e-04\\
Ti47 & 1.015870e-04\\
Ti48 & 1.027943e-03\\
Ti49 & 7.700943e-05\\
Ti50 & 7.523736e-05\\
V & 2.730040e-05\\
Cr50 & 4.919782e-03\\
Cr52 & 9.866164e-02\\
Cr53 & 1.140286e-02\\
Cr54 & 2.891929e-03\\
Mn55 & 1.365088e-02\\
Fe54 & 2.501631e-02\\
Fe56 & 4.072277e-01\\
Fe57 & 9.572885e-03\\
Fe58 & 1.296304e-03\\
Co59 & 3.412721e-04\\
Ni58 & 5.503822e-02\\
Ni60 & 2.193083e-02\\
Ni61 & 9.692286e-04\\
Ni62 & 3.140908e-03\\
Ni64 & 8.257244e-04\\
Cu63 & 1.755623e-01\\
Cu65 & 8.073510e-02\\
Zr & 1.364937e-05\\
Nb93 & 2.218681e-02\\
Mo92 & 2.426000e-03\\
Mo94 & 1.545044e-03\\
Mo95 & 2.687481e-03\\
Mo96 & 2.845421e-03\\
Mo97 & 1.646130e-03\\
Mo98 & 4.202163e-03\\
Mo100 & 1.711333e-03\\
Sn & 7.386505e-03\\
Ta181 & 5.597890e-03\\
W182 & 1.792061e-06\\
W183 & 9.733890e-07\\
W184 & 2.095129e-06\\
W186 & 1.964781e-06\\
Pb206 & 1.326335e-06\\
Pb207 & 1.222183e-06\\
Pb208 & 2.911851e-06\\
Bi209 & 5.460353e-06\\

\caption{Table showing the isotopic description of material M907}
\label{table:material_M907}
\end{longtable}\clearpage

\begin{longtable}[ht!]
{ p{0.3\textwidth} | p{0.3\textwidth} }
\hline
Nuclide & Mass Fraction\\
\hline
H1 & 4.122292e-04\\
H2 & 9.475075e-08\\
B10 & 1.836323e-06\\
B11 & 8.126946e-06\\
C12 & 2.954349e-04\\
N14 & 6.946763e-04\\
N15 & 2.776554e-06\\
O16 & 3.263613e-03\\
Si28 & 4.576728e-03\\
Si29 & 2.406955e-04\\
Si30 & 1.641280e-04\\
P31 & 2.490785e-04\\
S & 9.963397e-05\\
Ti46 & 7.890971e-05\\
Ti47 & 7.270954e-05\\
Ti48 & 7.354280e-04\\
Ti49 & 5.511848e-05\\
Ti50 & 5.385011e-05\\
Cr50 & 7.277097e-03\\
Cr52 & 1.459363e-01\\
Cr53 & 1.686649e-02\\
Cr54 & 4.277567e-03\\
Mn55 & 1.793380e-02\\
Fe54 & 3.647366e-02\\
Fe56 & 5.937391e-01\\
Fe57 & 1.395727e-02\\
Fe58 & 1.889996e-03\\
Co59 & 4.981619e-04\\
Ni58 & 8.201421e-02\\
Ni60 & 3.267989e-02\\
Ni61 & 1.444278e-03\\
Ni62 & 4.680372e-03\\
Ni64 & 1.230439e-03\\
Cu63 & 2.047427e-03\\
Cu65 & 9.415441e-04\\
Nb93 & 9.963253e-05\\
Mo92 & 3.541301e-03\\
Mo94 & 2.255343e-03\\
Mo95 & 3.922992e-03\\
Mo96 & 4.153521e-03\\
Mo97 & 2.402889e-03\\
Mo98 & 6.134007e-03\\
Mo100 & 2.498074e-03\\
Ta181 & 9.963210e-05\\

\caption{Table showing the isotopic description of material UPDSM}
\label{table:material_UPDSM}
\end{longtable}\clearpage

\begin{longtable}[ht!]
{ p{0.3\textwidth} | p{0.3\textwidth} }
\hline
Nuclide & Mass Fraction\\
\hline
\\
B10 & 1.843098e-06\\
B11 & 8.156930e-06\\
C12 & 2.965249e-04\\
N14 & 6.972393e-04\\
N15 & 2.786798e-06\\
Si28 & 4.593614e-03\\
Si29 & 2.415835e-04\\
Si30 & 1.647335e-04\\
P31 & 2.499975e-04\\
S & 1.000016e-04\\
Ti46 & 7.920085e-05\\
Ti47 & 7.297781e-05\\
Ti48 & 7.381413e-04\\
Ti49 & 5.532184e-05\\
Ti50 & 5.404879e-05\\
Cr50 & 7.303946e-03\\
Cr52 & 1.464747e-01\\
Cr53 & 1.692872e-02\\
Cr54 & 4.293349e-03\\
Mn55 & 1.799997e-02\\
Fe54 & 3.660823e-02\\
Fe56 & 5.959297e-01\\
Fe57 & 1.400876e-02\\
Fe58 & 1.896969e-03\\
Co59 & 4.999999e-04\\
Ni58 & 8.231680e-02\\
Ni60 & 3.280046e-02\\
Ni61 & 1.449607e-03\\
Ni62 & 4.697640e-03\\
Ni64 & 1.234979e-03\\
Cu63 & 2.054981e-03\\
Cu65 & 9.450179e-04\\
Nb93 & 1.000001e-04\\
Mo92 & 3.554367e-03\\
Mo94 & 2.263664e-03\\
Mo95 & 3.937466e-03\\
Mo96 & 4.168845e-03\\
Mo97 & 2.411755e-03\\
Mo98 & 6.156638e-03\\
Mo100 & 2.507290e-03\\
Ta181 & 9.999969e-05\\

\caption{Table showing the isotopic description of material EPTRAP}
\label{table:material_EPTRAP}
\end{longtable}\clearpage

\begin{longtable}[ht!]
{ p{0.3\textwidth} | p{0.3\textwidth} }
\hline
Nuclide & Mass Fraction\\
\hline
\\
B10 & 1.843098e-06\\
B11 & 8.156930e-06\\
C12 & 2.965249e-04\\
N14 & 6.972393e-04\\
N15 & 2.786798e-06\\
Si28 & 4.593614e-03\\
Si29 & 2.415835e-04\\
Si30 & 1.647335e-04\\
P31 & 2.499975e-04\\
S & 1.000016e-04\\
Ti46 & 7.920085e-05\\
Ti47 & 7.297781e-05\\
Ti48 & 7.381413e-04\\
Ti49 & 5.532184e-05\\
Ti50 & 5.404879e-05\\
Cr50 & 7.303946e-03\\
Cr52 & 1.464747e-01\\
Cr53 & 1.692872e-02\\
Cr54 & 4.293349e-03\\
Mn55 & 1.799997e-02\\
Fe54 & 3.660823e-02\\
Fe56 & 5.959297e-01\\
Fe57 & 1.400876e-02\\
Fe58 & 1.896969e-03\\
Co59 & 4.999999e-04\\
Ni58 & 8.231680e-02\\
Ni60 & 3.280046e-02\\
Ni61 & 1.449607e-03\\
Ni62 & 4.697640e-03\\
Ni64 & 1.234979e-03\\
Cu63 & 2.054981e-03\\
Cu65 & 9.450179e-04\\
Nb93 & 1.000001e-04\\
Mo92 & 3.554367e-03\\
Mo94 & 2.263664e-03\\
Mo95 & 3.937466e-03\\
Mo96 & 4.168845e-03\\
Mo97 & 2.411755e-03\\
Mo98 & 6.156638e-03\\
Mo100 & 2.507290e-03\\
Ta181 & 9.999969e-05\\

\caption{Table showing the isotopic description of material PPWater}
\label{table:material_PPWater}
\end{longtable}\clearpage

\begin{longtable}[ht!]
{ p{0.3\textwidth} | p{0.3\textwidth} }
\hline
Nuclide & Mass Fraction\\
\hline
\\
Al27 & 9.749971e-02\\
Si28 & 1.837444e-03\\
Si29 & 9.663327e-05\\
Si30 & 6.589366e-05\\
Mn55 & 9.999996e-03\\
Fe54 & 2.258229e-03\\
Fe56 & 3.676065e-02\\
Fe57 & 8.641477e-04\\
Fe58 & 1.170170e-04\\
Co59 & 5.000005e-04\\
Ni58 & 3.359869e-02\\
Ni60 & 1.338796e-02\\
Ni61 & 5.916763e-04\\
Ni62 & 1.917401e-03\\
Ni64 & 5.040734e-04\\
Cu63 & 5.461480e-01\\
Cu65 & 2.511552e-01\\
Nb93 & 9.999996e-04\\
Sn & 1.000003e-03\\
Ta181 & 4.999996e-04\\
Pb206 & 4.791233e-05\\
Pb207 & 4.414810e-05\\
Pb208 & 1.051841e-04\\

\caption{Table showing the isotopic description of material M303}
\label{table:material_M303}
\end{longtable}\clearpage

\begin{longtable}[ht!]
{ p{0.3\textwidth} | p{0.3\textwidth} }
\hline
Nuclide & Mass Fraction\\
\hline
\\
B10 & 1.843112e-05\\
B11 & 8.156993e-05\\
C12 & 7.907392e-04\\
Al27 & 3.500018e-03\\
Si28 & 9.187303e-03\\
Si29 & 4.831706e-04\\
Si30 & 3.294700e-04\\
P31 & 3.999987e-04\\
S & 3.000072e-04\\
Ti46 & 1.683034e-03\\
Ti47 & 1.550790e-03\\
Ti48 & 1.568564e-02\\
Ti49 & 1.175600e-03\\
Ti50 & 1.148546e-03\\
V & 2.999879e-03\\
Cr50 & 6.156236e-03\\
Cr52 & 1.234582e-01\\
Cr53 & 1.426861e-02\\
Cr54 & 3.618704e-03\\
Mn55 & 2.000011e-02\\
Fe54 & 2.947910e-02\\
Fe56 & 4.798757e-01\\
Fe57 & 1.128065e-02\\
Fe58 & 1.527545e-03\\
Co59 & 2.000021e-03\\
Ni58 & 1.713547e-01\\
Ni60 & 6.827897e-02\\
Ni61 & 3.017577e-03\\
Ni62 & 9.778832e-03\\
Ni64 & 2.570789e-03\\
Nb93 & 1.000009e-03\\
Mo92 & 1.777191e-03\\
Mo94 & 1.131836e-03\\
Mo95 & 1.968744e-03\\
Mo96 & 2.084445e-03\\
Mo97 & 1.205889e-03\\
Mo98 & 3.078348e-03\\
Mo100 & 1.253653e-03\\
Ta181 & 5.000035e-04\\

\caption{Table showing the isotopic description of material EPPCH}
\label{table:material_EPPCH}
\end{longtable}\clearpage

\begin{longtable}[ht!]
{ p{0.3\textwidth} | p{0.3\textwidth} }
\hline
Nuclide & Mass Fraction\\
\hline
\\
B10 & 1.442461e-01\\
B11 & 6.383841e-01\\
C12 & 2.148517e-01\\
C13 & 2.518076e-03\\

\caption{Table showing the isotopic description of material B4C}
\label{table:material_B4C}
\end{longtable}\clearpage

\begin{longtable}[ht!]
{ p{0.3\textwidth} | p{0.3\textwidth} }
\hline
Nuclide & Mass Fraction\\
\hline
H1 & 1.847040e-03\\
H2 & 5.536889e-07\\
B10 & 2.078773e-06\\
B11 & 8.447382e-06\\
C12 & 2.246913e-04\\
N14 & 6.589756e-04\\
N15 & 2.459362e-06\\
O16 & 1.468420e-02\\
Al27 & 5.622494e-04\\
Si28 & 4.416591e-03\\
Si29 & 2.247218e-04\\
Si30 & 1.481326e-04\\
P31 & 2.389155e-04\\
S & 7.355948e-05\\
K & 4.723997e-06\\
Ti46 & 1.297295e-04\\
Ti47 & 1.172395e-04\\
Ti48 & 1.163744e-03\\
Ti49 & 8.562027e-05\\
Ti50 & 8.214717e-05\\
V & 3.778995e-05\\
Cr50 & 7.333487e-03\\
Cr52 & 1.415351e-01\\
Cr53 & 1.605524e-02\\
Cr54 & 3.998101e-03\\
Mn55 & 1.707051e-02\\
Fe54 & 3.604376e-02\\
Fe56 & 5.659124e-01\\
Fe57 & 1.307070e-02\\
Fe58 & 1.739600e-03\\
Co59 & 5.028159e-04\\
Ni58 & 8.513362e-02\\
Ni60 & 3.287757e-02\\
Ni61 & 1.430997e-03\\
Ni62 & 4.568483e-03\\
Ni64 & 1.166425e-03\\
Cu63 & 1.472533e-02\\
Cu65 & 6.762427e-03\\
Zr & 4.164644e-05\\
Nb93 & 1.010637e-03\\
Mo92 & 3.582076e-03\\
Mo94 & 2.233810e-03\\
Mo95 & 3.845468e-03\\
Mo96 & 4.029974e-03\\
Mo97 & 2.307890e-03\\
Mo98 & 5.832650e-03\\
Mo100 & 2.328829e-03\\
Sn & 1.890806e-05\\
Ta181 & 1.034368e-04\\
W182 & 2.505661e-06\\
W183 & 1.331036e-06\\
W184 & 2.915618e-06\\
W186 & 2.705784e-06\\
Pb206 & 1.819663e-06\\
Pb207 & 1.667171e-06\\
Pb208 & 3.944902e-06\\
Bi209 & 7.547778e-06\\

\caption{Table showing the isotopic description of material ShieldBlock}
\label{table:material_ShieldBlock}
\end{longtable}\clearpage

\begin{longtable}[ht!]
{ p{0.3\textwidth} | p{0.3\textwidth} }
\hline
Nuclide & Mass Fraction\\
\hline
\\
B10 & 1.843098e-06\\
B11 & 8.156930e-06\\
C12 & 2.965249e-04\\
N14 & 6.972393e-04\\
N15 & 2.786798e-06\\
Si28 & 4.593614e-03\\
Si29 & 2.415835e-04\\
Si30 & 1.647335e-04\\
P31 & 2.499975e-04\\
S & 1.000016e-04\\
Ti46 & 7.920085e-05\\
Ti47 & 7.297781e-05\\
Ti48 & 7.381413e-04\\
Ti49 & 5.532184e-05\\
Ti50 & 5.404879e-05\\
Cr50 & 7.303946e-03\\
Cr52 & 1.464747e-01\\
Cr53 & 1.692872e-02\\
Cr54 & 4.293349e-03\\
Mn55 & 1.799997e-02\\
Fe54 & 3.660823e-02\\
Fe56 & 5.959297e-01\\
Fe57 & 1.400876e-02\\
Fe58 & 1.896969e-03\\
Co59 & 4.999999e-04\\
Ni58 & 8.231680e-02\\
Ni60 & 3.280046e-02\\
Ni61 & 1.449607e-03\\
Ni62 & 4.697640e-03\\
Ni64 & 1.234979e-03\\
Cu63 & 2.054981e-03\\
Cu65 & 9.450179e-04\\
Nb93 & 1.000001e-04\\
Mo92 & 3.554367e-03\\
Mo94 & 2.263664e-03\\
Mo95 & 3.937466e-03\\
Mo96 & 4.168845e-03\\
Mo97 & 2.411755e-03\\
Mo98 & 6.156638e-03\\
Mo100 & 2.507290e-03\\
Ta181 & 9.999969e-05\\

\caption{Table showing the isotopic description of material UppDucts}
\label{table:material_UppDucts}
\end{longtable}\clearpage

\begin{longtable}[ht!]
{ p{0.3\textwidth} | p{0.3\textwidth} }
\hline
Nuclide & Mass Fraction\\
\hline
H1 & 5.556235e-03\\
H2 & 1.277099e-06\\
O16 & 4.968034e-01\\
Na23 & 1.710308e-02\\
Mg & 2.563458e-03\\
Al27 & 4.697334e-02\\
Si28 & 2.895215e-01\\
Si29 & 1.522632e-02\\
Si30 & 1.038271e-02\\
S & 1.281754e-03\\
K & 1.924422e-02\\
Ca & 8.294590e-02\\
Fe54 & 6.998641e-04\\
Fe56 & 1.139283e-02\\
Fe57 & 2.678156e-04\\
Fe58 & 3.626579e-05\\

\caption{Table showing the isotopic description of material M200}
\label{table:material_M200}
\end{longtable}\clearpage

\begin{longtable}[ht!]
{ p{0.3\textwidth} | p{0.3\textwidth} }
\hline
Nuclide & Mass Fraction\\
\hline
\\
B10 & 1.843098e-06\\
B11 & 8.156930e-06\\
C12 & 2.965249e-04\\
N14 & 6.972393e-04\\
N15 & 2.786798e-06\\
Si28 & 4.593614e-03\\
Si29 & 2.415835e-04\\
Si30 & 1.647335e-04\\
P31 & 2.499975e-04\\
S & 1.000016e-04\\
Ti46 & 7.920085e-05\\
Ti47 & 7.297781e-05\\
Ti48 & 7.381413e-04\\
Ti49 & 5.532184e-05\\
Ti50 & 5.404879e-05\\
Cr50 & 7.303946e-03\\
Cr52 & 1.464747e-01\\
Cr53 & 1.692872e-02\\
Cr54 & 4.293349e-03\\
Mn55 & 1.799997e-02\\
Fe54 & 3.660823e-02\\
Fe56 & 5.959297e-01\\
Fe57 & 1.400876e-02\\
Fe58 & 1.896969e-03\\
Co59 & 4.999999e-04\\
Ni58 & 8.231680e-02\\
Ni60 & 3.280046e-02\\
Ni61 & 1.449607e-03\\
Ni62 & 4.697640e-03\\
Ni64 & 1.234979e-03\\
Cu63 & 2.054981e-03\\
Cu65 & 9.450179e-04\\
Nb93 & 1.000001e-04\\
Mo92 & 3.554367e-03\\
Mo94 & 2.263664e-03\\
Mo95 & 3.937466e-03\\
Mo96 & 4.168845e-03\\
Mo97 & 2.411755e-03\\
Mo98 & 6.156638e-03\\
Mo100 & 2.507290e-03\\
Ta181 & 9.999969e-05\\

\caption{Table showing the isotopic description of material EppDiagPipes}
\label{table:material_EppDiagPipes}
\end{longtable}\clearpage

\begin{longtable}[ht!]
{ p{0.3\textwidth} | p{0.3\textwidth} }
\hline
Nuclide & Mass Fraction\\
\hline
\\
B10 & 1.843098e-06\\
B11 & 8.156930e-06\\
C12 & 2.965249e-04\\
N14 & 6.972393e-04\\
N15 & 2.786798e-06\\
Si28 & 4.593614e-03\\
Si29 & 2.415835e-04\\
Si30 & 1.647335e-04\\
P31 & 2.499975e-04\\
S & 1.000016e-04\\
Ti46 & 7.920085e-05\\
Ti47 & 7.297781e-05\\
Ti48 & 7.381413e-04\\
Ti49 & 5.532184e-05\\
Ti50 & 5.404879e-05\\
Cr50 & 7.303946e-03\\
Cr52 & 1.464747e-01\\
Cr53 & 1.692872e-02\\
Cr54 & 4.293349e-03\\
Mn55 & 1.799997e-02\\
Fe54 & 3.660823e-02\\
Fe56 & 5.959297e-01\\
Fe57 & 1.400876e-02\\
Fe58 & 1.896969e-03\\
Co59 & 4.999999e-04\\
Ni58 & 8.231680e-02\\
Ni60 & 3.280046e-02\\
Ni61 & 1.449607e-03\\
Ni62 & 4.697640e-03\\
Ni64 & 1.234979e-03\\
Cu63 & 2.054981e-03\\
Cu65 & 9.450179e-04\\
Nb93 & 1.000001e-04\\
Mo92 & 3.554367e-03\\
Mo94 & 2.263664e-03\\
Mo95 & 3.937466e-03\\
Mo96 & 4.168845e-03\\
Mo97 & 2.411755e-03\\
Mo98 & 6.156638e-03\\
Mo100 & 2.507290e-03\\
Ta181 & 9.999969e-05\\

\caption{Table showing the isotopic description of material EppWaterPipes}
\label{table:material_EppWaterPipes}
\end{longtable}\clearpage

\begin{longtable}[ht!]
{ p{0.3\textwidth} | p{0.3\textwidth} }
\hline
Nuclide & Mass Fraction\\
\hline
\\
B10 & 1.843098e-06\\
B11 & 8.156930e-06\\
C12 & 2.965249e-04\\
N14 & 6.972393e-04\\
N15 & 2.786798e-06\\
Si28 & 4.593614e-03\\
Si29 & 2.415835e-04\\
Si30 & 1.647335e-04\\
P31 & 2.499975e-04\\
S & 1.000016e-04\\
Ti46 & 7.920085e-05\\
Ti47 & 7.297781e-05\\
Ti48 & 7.381413e-04\\
Ti49 & 5.532184e-05\\
Ti50 & 5.404879e-05\\
Cr50 & 7.303946e-03\\
Cr52 & 1.464747e-01\\
Cr53 & 1.692872e-02\\
Cr54 & 4.293349e-03\\
Mn55 & 1.799997e-02\\
Fe54 & 3.660823e-02\\
Fe56 & 5.959297e-01\\
Fe57 & 1.400876e-02\\
Fe58 & 1.896969e-03\\
Co59 & 4.999999e-04\\
Ni58 & 8.231680e-02\\
Ni60 & 3.280046e-02\\
Ni61 & 1.449607e-03\\
Ni62 & 4.697640e-03\\
Ni64 & 1.234979e-03\\
Cu63 & 2.054981e-03\\
Cu65 & 9.450179e-04\\
Nb93 & 1.000001e-04\\
Mo92 & 3.554367e-03\\
Mo94 & 2.263664e-03\\
Mo95 & 3.937466e-03\\
Mo96 & 4.168845e-03\\
Mo97 & 2.411755e-03\\
Mo98 & 6.156638e-03\\
Mo100 & 2.507290e-03\\
Ta181 & 9.999969e-05\\

\caption{Table showing the isotopic description of material EppDIagBox}
\label{table:material_EppDIagBox}
\end{longtable}\clearpage

\begin{longtable}[ht!]
{ p{0.3\textwidth} | p{0.3\textwidth} }
\hline
Nuclide & Mass Fraction\\
\hline
H1 & 5.454527e-03\\
H2 & 1.253721e-06\\
B10 & 1.753451e-06\\
B11 & 7.760184e-06\\
C12 & 2.821022e-04\\
N14 & 6.633262e-04\\
N15 & 2.651251e-06\\
O16 & 4.318342e-02\\
Si28 & 4.370185e-03\\
Si29 & 2.298331e-04\\
Si30 & 1.567210e-04\\
P31 & 2.378378e-04\\
S & 9.513758e-05\\
Ti46 & 7.534858e-05\\
Ti47 & 6.942822e-05\\
Ti48 & 7.022387e-04\\
Ti49 & 5.263103e-05\\
Ti50 & 5.141990e-05\\
Cr50 & 6.948688e-03\\
Cr52 & 1.393503e-01\\
Cr53 & 1.610532e-02\\
Cr54 & 4.084524e-03\\
Mn55 & 1.712447e-02\\
Fe54 & 3.482764e-02\\
Fe56 & 5.669441e-01\\
Fe57 & 1.332739e-02\\
Fe58 & 1.804702e-03\\
Co59 & 4.756803e-04\\
Ni58 & 7.831298e-02\\
Ni60 & 3.120507e-02\\
Ni61 & 1.379099e-03\\
Ni62 & 4.469151e-03\\
Ni64 & 1.174910e-03\\
Cu63 & 1.955029e-03\\
Cu65 & 8.990530e-04\\
Nb93 & 9.513620e-05\\
Mo92 & 3.381485e-03\\
Mo94 & 2.153561e-03\\
Mo95 & 3.745951e-03\\
Mo96 & 3.966076e-03\\
Mo97 & 2.294449e-03\\
Mo98 & 5.857184e-03\\
Mo100 & 2.385338e-03\\
Ta181 & 9.513578e-05\\

\caption{Table showing the isotopic description of material UPDFW2}
\label{table:material_UPDFW2}
\end{longtable}\clearpage

\begin{longtable}[ht!]
{ p{0.3\textwidth} | p{0.3\textwidth} }
\hline
Nuclide & Mass Fraction\\
\hline
\\
B10 & 1.843098e-06\\
B11 & 8.156930e-06\\
C12 & 2.965249e-04\\
N14 & 6.972393e-04\\
N15 & 2.786798e-06\\
Si28 & 4.593614e-03\\
Si29 & 2.415835e-04\\
Si30 & 1.647335e-04\\
P31 & 2.499975e-04\\
S & 1.000016e-04\\
Ti46 & 7.920085e-05\\
Ti47 & 7.297781e-05\\
Ti48 & 7.381413e-04\\
Ti49 & 5.532184e-05\\
Ti50 & 5.404879e-05\\
Cr50 & 7.303946e-03\\
Cr52 & 1.464747e-01\\
Cr53 & 1.692872e-02\\
Cr54 & 4.293349e-03\\
Mn55 & 1.799997e-02\\
Fe54 & 3.660823e-02\\
Fe56 & 5.959297e-01\\
Fe57 & 1.400876e-02\\
Fe58 & 1.896969e-03\\
Co59 & 4.999999e-04\\
Ni58 & 8.231680e-02\\
Ni60 & 3.280046e-02\\
Ni61 & 1.449607e-03\\
Ni62 & 4.697640e-03\\
Ni64 & 1.234979e-03\\
Cu63 & 2.054981e-03\\
Cu65 & 9.450179e-04\\
Nb93 & 1.000001e-04\\
Mo92 & 3.554367e-03\\
Mo94 & 2.263664e-03\\
Mo95 & 3.937466e-03\\
Mo96 & 4.168845e-03\\
Mo97 & 2.411755e-03\\
Mo98 & 6.156638e-03\\
Mo100 & 2.507290e-03\\
Ta181 & 9.999969e-05\\

\caption{Table showing the isotopic description of material PPWheelsDrives}
\label{table:material_PPWheelsDrives}
\end{longtable}\clearpage

\begin{longtable}[ht!]
{ p{0.3\textwidth} | p{0.3\textwidth} }
\hline
Nuclide & Mass Fraction\\
\hline
\\
Mg24 & 1.886648e-01\\
Mg25 & 2.488124e-02\\
Mg26 & 2.848708e-02\\
Al27 & 1.932231e-01\\
Si28 & 1.482315e-01\\
Si29 & 7.799320e-03\\
Si30 & 5.324541e-03\\
Ti46 & 2.396156e-03\\
Ti47 & 2.207881e-03\\
Ti48 & 2.234118e-02\\
Ti49 & 1.673716e-03\\
Ti50 & 1.635200e-03\\
Cr50 & 2.946330e-03\\
Cr52 & 5.908588e-02\\
Cr53 & 6.828873e-03\\
Cr54 & 1.731903e-03\\
Mn55 & 3.025413e-02\\
Fe54 & 7.970736e-03\\
Fe56 & 1.297521e-01\\
Fe57 & 3.050137e-03\\
Fe58 & 4.130316e-04\\
Cu63 & 5.524743e-02\\
Cu65 & 2.543026e-02\\
Zn64 & 2.424389e-02\\
Zn66 & 1.409971e-02\\
Zn67 & 2.085388e-03\\
Zn68 & 9.665589e-03\\
Zn70 & 3.289786e-04\\

\caption{Table showing the isotopic description of material UppExFrames}
\label{table:material_UppExFrames}
\end{longtable}\clearpage

\begin{longtable}[ht!]
{ p{0.3\textwidth} | p{0.3\textwidth} }
\hline
Nuclide & Mass Fraction\\
\hline
\\
B10 & 1.843098e-06\\
B11 & 8.156930e-06\\
C12 & 2.965249e-04\\
N14 & 6.972393e-04\\
N15 & 2.786798e-06\\
Si28 & 4.593614e-03\\
Si29 & 2.415835e-04\\
Si30 & 1.647335e-04\\
P31 & 2.499975e-04\\
S & 1.000016e-04\\
Ti46 & 7.920085e-05\\
Ti47 & 7.297781e-05\\
Ti48 & 7.381413e-04\\
Ti49 & 5.532184e-05\\
Ti50 & 5.404879e-05\\
Cr50 & 7.303946e-03\\
Cr52 & 1.464747e-01\\
Cr53 & 1.692872e-02\\
Cr54 & 4.293349e-03\\
Mn55 & 1.799997e-02\\
Fe54 & 3.660823e-02\\
Fe56 & 5.959297e-01\\
Fe57 & 1.400876e-02\\
Fe58 & 1.896969e-03\\
Co59 & 4.999999e-04\\
Ni58 & 8.231680e-02\\
Ni60 & 3.280046e-02\\
Ni61 & 1.449607e-03\\
Ni62 & 4.697640e-03\\
Ni64 & 1.234979e-03\\
Cu63 & 2.054981e-03\\
Cu65 & 9.450179e-04\\
Nb93 & 1.000001e-04\\
Mo92 & 3.554367e-03\\
Mo94 & 2.263664e-03\\
Mo95 & 3.937466e-03\\
Mo96 & 4.168845e-03\\
Mo97 & 2.411755e-03\\
Mo98 & 6.156638e-03\\
Mo100 & 2.507290e-03\\
Ta181 & 9.999969e-05\\

\caption{Table showing the isotopic description of material PPF}
\label{table:material_PPF}
\end{longtable}\clearpage

\begin{longtable}[ht!]
{ p{0.3\textwidth} | p{0.3\textwidth} }
\hline
Nuclide & Mass Fraction\\
\hline
\\
B10 & 1.843098e-06\\
B11 & 8.156930e-06\\
C12 & 2.965249e-04\\
N14 & 6.972393e-04\\
N15 & 2.786798e-06\\
Si28 & 4.593614e-03\\
Si29 & 2.415835e-04\\
Si30 & 1.647335e-04\\
P31 & 2.499975e-04\\
S & 1.000016e-04\\
Ti46 & 7.920085e-05\\
Ti47 & 7.297781e-05\\
Ti48 & 7.381413e-04\\
Ti49 & 5.532184e-05\\
Ti50 & 5.404879e-05\\
Cr50 & 7.303946e-03\\
Cr52 & 1.464747e-01\\
Cr53 & 1.692872e-02\\
Cr54 & 4.293349e-03\\
Mn55 & 1.799997e-02\\
Fe54 & 3.660823e-02\\
Fe56 & 5.959297e-01\\
Fe57 & 1.400876e-02\\
Fe58 & 1.896969e-03\\
Co59 & 4.999999e-04\\
Ni58 & 8.231680e-02\\
Ni60 & 3.280046e-02\\
Ni61 & 1.449607e-03\\
Ni62 & 4.697640e-03\\
Ni64 & 1.234979e-03\\
Cu63 & 2.054981e-03\\
Cu65 & 9.450179e-04\\
Nb93 & 1.000001e-04\\
Mo92 & 3.554367e-03\\
Mo94 & 2.263664e-03\\
Mo95 & 3.937466e-03\\
Mo96 & 4.168845e-03\\
Mo97 & 2.411755e-03\\
Mo98 & 6.156638e-03\\
Mo100 & 2.507290e-03\\
Ta181 & 9.999969e-05\\

\caption{Table showing the isotopic description of material UPPFW}
\label{table:material_UPPFW}
\end{longtable}\clearpage

\begin{longtable}[ht!]
{ p{0.3\textwidth} | p{0.3\textwidth} }
\hline
Nuclide & Mass Fraction\\
\hline
H1 & 1.195209e-02\\
H2 & 2.747184e-06\\
B10 & 1.646661e-06\\
B11 & 7.287539e-06\\
C12 & 2.649211e-04\\
N14 & 6.229271e-04\\
N15 & 2.489779e-06\\
O16 & 9.462485e-02\\
Si28 & 4.104019e-03\\
Si29 & 2.158360e-04\\
Si30 & 1.471767e-04\\
P31 & 2.233526e-04\\
S & 8.934346e-05\\
Ti46 & 7.075966e-05\\
Ti47 & 6.519978e-05\\
Ti48 & 6.594697e-04\\
Ti49 & 4.942570e-05\\
Ti50 & 4.828815e-05\\
Cr50 & 6.525487e-03\\
Cr52 & 1.308633e-01\\
Cr53 & 1.512446e-02\\
Cr54 & 3.835761e-03\\
Mn55 & 1.608154e-02\\
Fe54 & 3.270650e-02\\
Fe56 & 5.324162e-01\\
Fe57 & 1.251572e-02\\
Fe58 & 1.694793e-03\\
Co59 & 4.467107e-04\\
Ni58 & 7.354343e-02\\
Ni60 & 2.930445e-02\\
Ni61 & 1.295109e-03\\
Ni62 & 4.196951e-03\\
Ni64 & 1.103354e-03\\
Cu63 & 1.835962e-03\\
Cu65 & 8.442973e-04\\
Nb93 & 8.934187e-05\\
Mo92 & 3.175533e-03\\
Mo94 & 2.022394e-03\\
Mo95 & 3.517801e-03\\
Mo96 & 3.724509e-03\\
Mo97 & 2.154712e-03\\
Mo98 & 5.500460e-03\\
Mo100 & 2.240058e-03\\
Ta181 & 8.934166e-05\\

\caption{Table showing the isotopic description of material M170}
\label{table:material_M170}
\end{longtable}\clearpage

\begin{longtable}[ht!]
{ p{0.3\textwidth} | p{0.3\textwidth} }
\hline
Nuclide & Mass Fraction\\
\hline
\\
B10 & 5.529302e-05\\
B11 & 2.447075e-04\\
C12 & 2.965248e-04\\
N14 & 9.960573e-04\\
N15 & 3.981142e-06\\
Si28 & 9.187247e-03\\
Si29 & 4.831668e-04\\
Si30 & 3.294675e-04\\
P31 & 2.999971e-04\\
S & 2.000030e-04\\
Cr50 & 7.199606e-03\\
Cr52 & 1.443820e-01\\
Cr53 & 1.668693e-02\\
Cr54 & 4.232016e-03\\
Mn55 & 2.000005e-02\\
Fe54 & 3.663419e-02\\
Fe56 & 5.963522e-01\\
Fe57 & 1.401877e-02\\
Fe58 & 1.898314e-03\\
Co59 & 1.000000e-03\\
Ni58 & 8.063689e-02\\
Ni60 & 3.213105e-02\\
Ni61 & 1.420018e-03\\
Ni62 & 4.601767e-03\\
Ni64 & 1.209775e-03\\
Nb93 & 4.999994e-04\\
Mo92 & 3.554365e-03\\
Mo94 & 2.263663e-03\\
Mo95 & 3.937464e-03\\
Mo96 & 4.168843e-03\\
Mo97 & 2.411753e-03\\
Mo98 & 6.156635e-03\\
Mo100 & 2.507289e-03\\

\caption{Table showing the isotopic description of material M111}
\label{table:material_M111}
\end{longtable}\clearpage

\begin{longtable}[ht!]
{ p{0.3\textwidth} | p{0.3\textwidth} }
\hline
Nuclide & Mass Fraction\\
\hline
\\
B10 & 5.529303e-05\\
B11 & 2.447076e-04\\
C12 & 2.965248e-04\\
N14 & 1.593692e-03\\
N15 & 6.369823e-06\\
Si28 & 9.187249e-03\\
Si29 & 4.831669e-04\\
Si30 & 3.294676e-04\\
P31 & 2.999971e-04\\
S & 2.000031e-04\\
Cr50 & 7.199608e-03\\
Cr52 & 1.443820e-01\\
Cr53 & 1.668693e-02\\
Cr54 & 4.232017e-03\\
Mn55 & 2.000005e-02\\
Fe54 & 3.660032e-02\\
Fe56 & 5.958007e-01\\
Fe57 & 1.400578e-02\\
Fe58 & 1.896568e-03\\
Co59 & 1.000001e-03\\
Ni58 & 8.063691e-02\\
Ni60 & 3.213106e-02\\
Ni61 & 1.420018e-03\\
Ni62 & 4.601768e-03\\
Ni64 & 1.209776e-03\\
Nb93 & 4.999995e-04\\
Mo92 & 3.554366e-03\\
Mo94 & 2.263663e-03\\
Mo95 & 3.937465e-03\\
Mo96 & 4.168844e-03\\
Mo97 & 2.411754e-03\\
Mo98 & 6.156637e-03\\
Mo100 & 2.507290e-03\\

\caption{Table showing the isotopic description of material M110}
\label{table:material_M110}
\end{longtable}\clearpage

\begin{longtable}[ht!]
{ p{0.3\textwidth} | p{0.3\textwidth} }
\hline
Nuclide & Mass Fraction\\
\hline
H1 & 1.121426e-01\\
H2 & 2.577594e-05\\
O16 & 8.878316e-01\\

\caption{Table showing the isotopic description of material M400}
\label{table:material_M400}
\end{longtable}\clearpage

\begin{longtable}[ht!]
{ p{0.3\textwidth} | p{0.3\textwidth} }
\hline
Nuclide & Mass Fraction\\
\hline
H1 & 1.466991e-03\\
H2 & 3.371874e-07\\
B10 & 1.818988e-06\\
B11 & 8.050226e-06\\
C12 & 2.926459e-04\\
N14 & 6.881184e-04\\
N15 & 2.750343e-06\\
O16 & 1.161415e-02\\
Si28 & 4.533523e-03\\
Si29 & 2.384233e-04\\
Si30 & 1.625786e-04\\
P31 & 2.467271e-04\\
S & 9.869340e-05\\
Ti46 & 7.816479e-05\\
Ti47 & 7.202315e-05\\
Ti48 & 7.284853e-04\\
Ti49 & 5.459815e-05\\
Ti50 & 5.334176e-05\\
Cr50 & 7.208399e-03\\
Cr52 & 1.445586e-01\\
Cr53 & 1.670727e-02\\
Cr54 & 4.237186e-03\\
Mn55 & 1.776451e-02\\
Fe54 & 3.612934e-02\\
Fe56 & 5.881340e-01\\
Fe57 & 1.382551e-02\\
Fe58 & 1.872153e-03\\
Co59 & 4.934592e-04\\
Ni58 & 8.123998e-02\\
Ni60 & 3.237138e-02\\
Ni61 & 1.430644e-03\\
Ni62 & 4.636188e-03\\
Ni64 & 1.218823e-03\\
Cu63 & 2.028099e-03\\
Cu65 & 9.326557e-04\\
Nb93 & 9.869198e-05\\
Mo92 & 3.507870e-03\\
Mo94 & 2.234052e-03\\
Mo95 & 3.885958e-03\\
Mo96 & 4.114311e-03\\
Mo97 & 2.380205e-03\\
Mo98 & 6.076101e-03\\
Mo100 & 2.474491e-03\\
Ta181 & 9.869155e-05\\

\caption{Table showing the isotopic description of material UPTRAP}
\label{table:material_UPTRAP}
\end{longtable}\clearpage

\begin{longtable}[ht!]
{ p{0.3\textwidth} | p{0.3\textwidth} }
\hline
Nuclide & Mass Fraction\\
\hline
\\
Mg24 & 1.886648e-01\\
Mg25 & 2.488124e-02\\
Mg26 & 2.848708e-02\\
Al27 & 1.932231e-01\\
Si28 & 1.482315e-01\\
Si29 & 7.799320e-03\\
Si30 & 5.324541e-03\\
Ti46 & 2.396156e-03\\
Ti47 & 2.207881e-03\\
Ti48 & 2.234118e-02\\
Ti49 & 1.673716e-03\\
Ti50 & 1.635200e-03\\
Cr50 & 2.946330e-03\\
Cr52 & 5.908588e-02\\
Cr53 & 6.828873e-03\\
Cr54 & 1.731903e-03\\
Mn55 & 3.025413e-02\\
Fe54 & 7.970736e-03\\
Fe56 & 1.297521e-01\\
Fe57 & 3.050137e-03\\
Fe58 & 4.130316e-04\\
Cu63 & 5.524743e-02\\
Cu65 & 2.543026e-02\\
Zn64 & 2.424389e-02\\
Zn66 & 1.409971e-02\\
Zn67 & 2.085388e-03\\
Zn68 & 9.665589e-03\\
Zn70 & 3.289786e-04\\

\caption{Table showing the isotopic description of material EppExFrames}
\label{table:material_EppExFrames}
\end{longtable}\clearpage

\begin{longtable}[ht!]
{ p{0.3\textwidth} | p{0.3\textwidth} }
\hline
Nuclide & Mass Fraction\\
\hline
H1 & 3.212605e-03\\
H2 & 7.384167e-07\\
B10 & 1.790300e-06\\
B11 & 7.923249e-06\\
O16 & 2.574419e-02\\
Mg & 3.885405e-04\\
Al27 & 2.914059e-05\\
Si28 & 3.569616e-04\\
Si29 & 1.877306e-05\\
Si30 & 1.280119e-05\\
P31 & 1.359879e-04\\
S & 3.885477e-05\\
Cr50 & 3.040592e-04\\
Cr52 & 6.097658e-03\\
Cr53 & 7.047341e-04\\
Cr54 & 1.787298e-04\\
Mn55 & 1.942713e-05\\
Fe54 & 1.096767e-05\\
Fe56 & 1.785373e-04\\
Fe57 & 4.196966e-06\\
Fe58 & 5.683249e-07\\
Co59 & 4.856775e-04\\
Ni58 & 1.958178e-04\\
Ni60 & 7.802652e-05\\
Ni61 & 3.448359e-06\\
Ni62 & 1.117486e-05\\
Ni64 & 2.937796e-06\\
Cu63 & 6.578606e-01\\
Cu65 & 3.025276e-01\\
Zr & 1.068370e-03\\
Sn & 9.713576e-05\\
Ta181 & 9.713515e-05\\
Pb206 & 2.326982e-05\\
Pb207 & 2.144172e-05\\
Pb208 & 5.108568e-05\\
Bi209 & 2.914063e-05\\

\caption{Table showing the isotopic description of material M623}
\label{table:material_M623}
\end{longtable}\clearpage

\begin{longtable}[ht!]
{ p{0.3\textwidth} | p{0.3\textwidth} }
\hline
Nuclide & Mass Fraction\\
\hline
H1 & 3.177371e-02\\
H2 & 7.303169e-06\\
B10 & 2.641778e-06\\
B11 & 1.169161e-05\\
C12 & 2.125101e-04\\
N14 & 4.996900e-04\\
N15 & 1.997206e-06\\
O16 & 2.515519e-01\\
Si28 & 3.292091e-03\\
Si29 & 1.731354e-04\\
Si30 & 1.180596e-04\\
P31 & 1.791652e-04\\
S & 7.166797e-05\\
Ti46 & 5.676085e-05\\
Ti47 & 5.230099e-05\\
Ti48 & 5.290034e-04\\
Ti49 & 3.964728e-05\\
Ti50 & 3.873488e-05\\
Cr50 & 5.234496e-03\\
Cr52 & 1.049738e-01\\
Cr53 & 1.213228e-02\\
Cr54 & 3.076909e-03\\
Mn55 & 1.290003e-02\\
Fe54 & 2.619917e-02\\
Fe56 & 4.264834e-01\\
Fe57 & 1.002554e-02\\
Fe58 & 1.357592e-03\\
Co59 & 3.583344e-04\\
Ni58 & 5.899375e-02\\
Ni60 & 2.350704e-02\\
Ni61 & 1.038889e-03\\
Ni62 & 3.366649e-03\\
Ni64 & 8.850723e-04\\
Cu63 & 1.472743e-03\\
Cu65 & 6.772640e-04\\
Nb93 & 7.166716e-04\\
Mo92 & 2.547297e-03\\
Mo94 & 1.622291e-03\\
Mo95 & 2.821851e-03\\
Mo96 & 2.987675e-03\\
Mo97 & 1.728435e-03\\
Mo98 & 4.412270e-03\\
Mo100 & 1.796891e-03\\
Ta181 & 7.166662e-05\\

\caption{Table showing the isotopic description of material M622}
\label{table:material_M622}
\end{longtable}\clearpage

\begin{longtable}[ht!]
{ p{0.3\textwidth} | p{0.3\textwidth} }
\hline
Nuclide & Mass Fraction\\
\hline
H1 & 5.950018e-04\\
H2 & 1.367611e-07\\
B10 & 3.666628e-06\\
B11 & 1.622728e-05\\
C12 & 2.949506e-04\\
N14 & 6.935400e-04\\
N15 & 2.772008e-06\\
O16 & 4.710643e-03\\
Si28 & 4.569239e-03\\
Si29 & 2.403015e-04\\
Si30 & 1.638596e-04\\
P31 & 2.486710e-04\\
S & 9.947128e-05\\
Ti46 & 7.878058e-05\\
Ti47 & 7.259048e-05\\
Ti48 & 7.342242e-04\\
Ti49 & 5.502828e-05\\
Ti50 & 5.376198e-05\\
Cr50 & 7.265198e-03\\
Cr52 & 1.456974e-01\\
Cr53 & 1.683887e-02\\
Cr54 & 4.270568e-03\\
Mn55 & 1.790443e-02\\
Fe54 & 3.636292e-02\\
Fe56 & 5.919362e-01\\
Fe57 & 1.391486e-02\\
Fe58 & 1.884264e-03\\
Co59 & 4.973469e-04\\
Ni58 & 8.188001e-02\\
Ni60 & 3.262635e-02\\
Ni61 & 1.441911e-03\\
Ni62 & 4.672706e-03\\
Ni64 & 1.228426e-03\\
Cu63 & 2.044084e-03\\
Cu65 & 9.400023e-04\\
Nb93 & 9.946949e-04\\
Mo92 & 3.535499e-03\\
Mo94 & 2.251654e-03\\
Mo95 & 3.916571e-03\\
Mo96 & 4.146718e-03\\
Mo97 & 2.398954e-03\\
Mo98 & 6.123969e-03\\
Mo100 & 2.493975e-03\\
Ta181 & 9.946933e-05\\

\caption{Table showing the isotopic description of material M621}
\label{table:material_M621}
\end{longtable}\clearpage

\begin{longtable}[ht!]
{ p{0.3\textwidth} | p{0.3\textwidth} }
\hline
Nuclide & Mass Fraction\\
\hline
\\
B10 & 1.843098e-06\\
B11 & 8.156930e-06\\
C12 & 2.965249e-04\\
N14 & 6.972393e-04\\
N15 & 2.786798e-06\\
Si28 & 4.593614e-03\\
Si29 & 2.415835e-04\\
Si30 & 1.647335e-04\\
P31 & 2.499975e-04\\
S & 1.000016e-04\\
Ti46 & 7.920085e-05\\
Ti47 & 7.297781e-05\\
Ti48 & 7.381413e-04\\
Ti49 & 5.532184e-05\\
Ti50 & 5.404879e-05\\
Cr50 & 7.303946e-03\\
Cr52 & 1.464747e-01\\
Cr53 & 1.692872e-02\\
Cr54 & 4.293349e-03\\
Mn55 & 1.799997e-02\\
Fe54 & 3.660823e-02\\
Fe56 & 5.959297e-01\\
Fe57 & 1.400876e-02\\
Fe58 & 1.896969e-03\\
Co59 & 4.999999e-04\\
Ni58 & 8.231680e-02\\
Ni60 & 3.280046e-02\\
Ni61 & 1.449607e-03\\
Ni62 & 4.697640e-03\\
Ni64 & 1.234979e-03\\
Cu63 & 2.054981e-03\\
Cu65 & 9.450179e-04\\
Nb93 & 1.000001e-04\\
Mo92 & 3.554367e-03\\
Mo94 & 2.263664e-03\\
Mo95 & 3.937466e-03\\
Mo96 & 4.168845e-03\\
Mo97 & 2.411755e-03\\
Mo98 & 6.156638e-03\\
Mo100 & 2.507290e-03\\
Ta181 & 9.999969e-05\\

\caption{Table showing the isotopic description of material EppDT}
\label{table:material_EppDT}
\end{longtable}\clearpage

\begin{longtable}[ht!]
{ p{0.3\textwidth} | p{0.3\textwidth} }
\hline
Nuclide & Mass Fraction\\
\hline
\\
B10 & 1.843098e-06\\
B11 & 8.156930e-06\\
C12 & 2.965249e-04\\
N14 & 6.972393e-04\\
N15 & 2.786798e-06\\
Si28 & 4.593614e-03\\
Si29 & 2.415835e-04\\
Si30 & 1.647335e-04\\
P31 & 2.499975e-04\\
S & 1.000016e-04\\
Ti46 & 7.920085e-05\\
Ti47 & 7.297781e-05\\
Ti48 & 7.381413e-04\\
Ti49 & 5.532184e-05\\
Ti50 & 5.404879e-05\\
Cr50 & 7.303946e-03\\
Cr52 & 1.464747e-01\\
Cr53 & 1.692872e-02\\
Cr54 & 4.293349e-03\\
Mn55 & 1.799997e-02\\
Fe54 & 3.660823e-02\\
Fe56 & 5.959297e-01\\
Fe57 & 1.400876e-02\\
Fe58 & 1.896969e-03\\
Co59 & 4.999999e-04\\
Ni58 & 8.231680e-02\\
Ni60 & 3.280046e-02\\
Ni61 & 1.449607e-03\\
Ni62 & 4.697640e-03\\
Ni64 & 1.234979e-03\\
Cu63 & 2.054981e-03\\
Cu65 & 9.450179e-04\\
Nb93 & 1.000001e-04\\
Mo92 & 3.554367e-03\\
Mo94 & 2.263664e-03\\
Mo95 & 3.937466e-03\\
Mo96 & 4.168845e-03\\
Mo97 & 2.411755e-03\\
Mo98 & 6.156638e-03\\
Mo100 & 2.507290e-03\\
Ta181 & 9.999969e-05\\

\caption{Table showing the isotopic description of material PPWheels}
\label{table:material_PPWheels}
\end{longtable}\clearpage

\begin{longtable}[ht!]
{ p{0.3\textwidth} | p{0.3\textwidth} }
\hline
Nuclide & Mass Fraction\\
\hline
\\
B10 & 1.843098e-06\\
B11 & 8.156930e-06\\
C12 & 2.965249e-04\\
N14 & 6.972393e-04\\
N15 & 2.786798e-06\\
Si28 & 4.593614e-03\\
Si29 & 2.415835e-04\\
Si30 & 1.647335e-04\\
P31 & 2.499975e-04\\
S & 1.000016e-04\\
Ti46 & 7.920085e-05\\
Ti47 & 7.297781e-05\\
Ti48 & 7.381413e-04\\
Ti49 & 5.532184e-05\\
Ti50 & 5.404879e-05\\
Cr50 & 7.303946e-03\\
Cr52 & 1.464747e-01\\
Cr53 & 1.692872e-02\\
Cr54 & 4.293349e-03\\
Mn55 & 1.799997e-02\\
Fe54 & 3.660823e-02\\
Fe56 & 5.959297e-01\\
Fe57 & 1.400876e-02\\
Fe58 & 1.896969e-03\\
Co59 & 4.999999e-04\\
Ni58 & 8.231680e-02\\
Ni60 & 3.280046e-02\\
Ni61 & 1.449607e-03\\
Ni62 & 4.697640e-03\\
Ni64 & 1.234979e-03\\
Cu63 & 2.054981e-03\\
Cu65 & 9.450179e-04\\
Nb93 & 1.000001e-04\\
Mo92 & 3.554367e-03\\
Mo94 & 2.263664e-03\\
Mo95 & 3.937466e-03\\
Mo96 & 4.168845e-03\\
Mo97 & 2.411755e-03\\
Mo98 & 6.156638e-03\\
Mo100 & 2.507290e-03\\
Ta181 & 9.999969e-05\\
\caption{Table showing the isotopic description of material EppMix}
\label{table:material_EppMix}
\end{longtable}\clearpage

\begin{longtable}[ht!]
{ p{0.3\textwidth} | p{0.3\textwidth} }
\hline
Nuclide & Mass Fraction\\
\hline
H1 & 8.251180e-04\\
H2 & 1.896534e-07\\
B10 & 3.659067e-06\\
B11 & 1.619381e-05\\
C12 & 2.943438e-04\\
N14 & 6.921097e-04\\
N15 & 2.766290e-06\\
O16 & 6.532444e-03\\
Si28 & 4.559809e-03\\
Si29 & 2.398056e-04\\
Si30 & 1.635213e-04\\
P31 & 2.481576e-04\\
S & 9.926556e-05\\
Ti46 & 7.861807e-05\\
Ti47 & 7.244078e-05\\
Ti48 & 7.327096e-04\\
Ti49 & 5.491474e-05\\
Ti50 & 5.365103e-05\\
Cr50 & 7.250200e-03\\
Cr52 & 1.453973e-01\\
Cr53 & 1.680419e-02\\
Cr54 & 4.261755e-03\\
Mn55 & 1.786755e-02\\
Fe54 & 3.628794e-02\\
Fe56 & 5.907146e-01\\
Fe57 & 1.388617e-02\\
Fe58 & 1.880371e-03\\
Co59 & 4.963209e-04\\
Ni58 & 8.171106e-02\\
Ni60 & 3.255910e-02\\
Ni61 & 1.438946e-03\\
Ni62 & 4.663070e-03\\
Ni64 & 1.225892e-03\\
Cu63 & 2.039858e-03\\
Cu65 & 9.380624e-04\\
Nb93 & 9.926428e-04\\
Mo92 & 3.528196e-03\\
Mo94 & 2.246995e-03\\
Mo95 & 3.908482e-03\\
Mo96 & 4.138165e-03\\
Mo97 & 2.394004e-03\\
Mo98 & 6.111329e-03\\
Mo100 & 2.488842e-03\\
Ta181 & 9.926390e-05\\
caption{Table showing the isotopic description of material M601}
\label{table:material_M601}
\end{longtable}\clearpage

\begin{longtable}[ht!]
{ p{0.3\textwidth} | p{0.3\textwidth} }
\hline
Nuclide & Mass Fraction\\
\hline
H1 & 3.243870e-03\\
H2 & 7.456037e-07\\
B10 & 1.789785e-06\\
B11 & 7.920970e-06\\
O16 & 2.599157e-02\\
Mg & 3.884290e-04\\
Al27 & 2.913214e-05\\
Si28 & 3.568587e-04\\
Si29 & 1.876763e-05\\
Si30 & 1.279748e-05\\
P31 & 1.359486e-04\\
S & 3.884363e-05\\
Cr50 & 3.039717e-04\\
Cr52 & 6.095905e-03\\
Cr53 & 7.045307e-04\\
Cr54 & 1.786781e-04\\
Mn55 & 1.942150e-05\\
Fe54 & 1.096452e-05\\
Fe56 & 1.784864e-04\\
Fe57 & 4.195764e-06\\
Fe58 & 5.681615e-07\\
Co59 & 4.855377e-04\\
Ni58 & 1.957615e-04\\
Ni60 & 7.800408e-05\\
Ni61 & 3.447379e-06\\
Ni62 & 1.117164e-05\\
Ni64 & 2.936951e-06\\
Cu63 & 6.576720e-01\\
Cu65 & 3.024406e-01\\
Zr & 1.068063e-03\\
Sn & 9.710766e-05\\
Ta181 & 9.710698e-05\\
Pb206 & 2.326315e-05\\
Pb207 & 2.143558e-05\\
Pb208 & 5.107068e-05\\
Bi209 & 2.913224e-05\\

\caption{Table showing the isotopic description of material M603}
\label{table:material_M603}
\end{longtable}\clearpage

\begin{longtable}[ht!]
{ p{0.3\textwidth} | p{0.3\textwidth} }
\hline
Nuclide & Mass Fraction\\
\hline
H1 & 3.090758e-02\\
H2 & 7.104097e-06\\
B10 & 2.670245e-06\\
B11 & 1.181760e-05\\
C12 & 2.147996e-04\\
N14 & 5.050741e-04\\
N15 & 2.018727e-06\\
O16 & 2.446947e-01\\
Si28 & 3.327574e-03\\
Si29 & 1.750009e-04\\
Si30 & 1.193315e-04\\
P31 & 1.810957e-04\\
S & 7.244026e-05\\
Ti46 & 5.737237e-05\\
Ti47 & 5.286433e-05\\
Ti48 & 5.347011e-04\\
Ti49 & 4.007472e-05\\
Ti50 & 3.915229e-05\\
Cr50 & 5.290908e-03\\
Cr52 & 1.061050e-01\\
Cr53 & 1.226301e-02\\
Cr54 & 3.110062e-03\\
Mn55 & 1.303902e-02\\
Fe54 & 2.648149e-02\\
Fe56 & 4.310799e-01\\
Fe57 & 1.013360e-02\\
Fe58 & 1.372222e-03\\
Co59 & 3.621953e-04\\
Ni58 & 5.962947e-02\\
Ni60 & 2.376035e-02\\
Ni61 & 1.050081e-03\\
Ni62 & 3.402923e-03\\
Ni64 & 8.946064e-04\\
Cu63 & 1.488611e-03\\
Cu65 & 6.845615e-04\\
Nb93 & 7.243894e-04\\
Mo92 & 2.574742e-03\\
Mo94 & 1.639770e-03\\
Mo95 & 2.852257e-03\\
Mo96 & 3.019871e-03\\
Mo97 & 1.747054e-03\\
Mo98 & 4.459812e-03\\
Mo100 & 1.816256e-03\\
Ta181 & 7.243891e-05\\

\caption{Table showing the isotopic description of material M602}
\label{table:material_M602}
\end{longtable}\clearpage

\begin{longtable}[ht!]
{ p{0.3\textwidth} | p{0.3\textwidth} }
\hline
Nuclide & Mass Fraction\\
\hline
\\
Al27 & 9.749971e-02\\
Si28 & 1.837444e-03\\
Si29 & 9.663327e-05\\
Si30 & 6.589366e-05\\
Mn55 & 9.999996e-03\\
Fe54 & 2.258229e-03\\
Fe56 & 3.676065e-02\\
Fe57 & 8.641477e-04\\
Fe58 & 1.170170e-04\\
Co59 & 5.000005e-04\\
Ni58 & 3.359869e-02\\
Ni60 & 1.338796e-02\\
Ni61 & 5.916763e-04\\
Ni62 & 1.917401e-03\\
Ni64 & 5.040734e-04\\
Cu63 & 5.461480e-01\\
Cu65 & 2.511552e-01\\
Nb93 & 9.999996e-04\\
Sn & 1.000003e-03\\
Ta181 & 4.999996e-04\\
Pb206 & 4.791233e-05\\
Pb207 & 4.414810e-05\\
Pb208 & 1.051841e-04\\

\caption{Table showing the isotopic description of material EPPDRW}
\label{table:material_EPPDRW}
\end{longtable}\clearpage

\begin{longtable}[ht!]
{ p{0.3\textwidth} | p{0.3\textwidth} }
\hline
Nuclide & Mass Fraction\\
\hline
\\
B10 & 1.843098e-06\\
B11 & 8.156930e-06\\
C12 & 2.965249e-04\\
N14 & 6.972393e-04\\
N15 & 2.786798e-06\\
Si28 & 4.593614e-03\\
Si29 & 2.415835e-04\\
Si30 & 1.647335e-04\\
P31 & 2.499975e-04\\
S & 1.000016e-04\\
Ti46 & 7.920085e-05\\
Ti47 & 7.297781e-05\\
Ti48 & 7.381413e-04\\
Ti49 & 5.532184e-05\\
Ti50 & 5.404879e-05\\
Cr50 & 7.303946e-03\\
Cr52 & 1.464747e-01\\
Cr53 & 1.692872e-02\\
Cr54 & 4.293349e-03\\
Mn55 & 1.799997e-02\\
Fe54 & 3.660823e-02\\
Fe56 & 5.959297e-01\\
Fe57 & 1.400876e-02\\
Fe58 & 1.896969e-03\\
Co59 & 4.999999e-04\\
Ni58 & 8.231680e-02\\
Ni60 & 3.280046e-02\\
Ni61 & 1.449607e-03\\
Ni62 & 4.697640e-03\\
Ni64 & 1.234979e-03\\
Cu63 & 2.054981e-03\\
Cu65 & 9.450179e-04\\
Nb93 & 1.000001e-04\\
Mo92 & 3.554367e-03\\
Mo94 & 2.263664e-03\\
Mo95 & 3.937466e-03\\
Mo96 & 4.168845e-03\\
Mo97 & 2.411755e-03\\
Mo98 & 6.156638e-03\\
Mo100 & 2.507290e-03\\
Ta181 & 9.999969e-05\\

\caption{Table showing the isotopic description of material EPPFW}
\label{table:material_EPPFW}
\end{longtable}\clearpage

\begin{longtable}[ht!]
{ p{0.3\textwidth} | p{0.3\textwidth} }
\hline
Nuclide & Mass Fraction\\
\hline
\\
O16 & 4.986451e-06\\
P31 & 2.999968e-06\\
S & 1.500027e-05\\
Mn55 & 5.000003e-07\\
Fe54 & 5.645562e-07\\
Fe56 & 9.190159e-06\\
Fe57 & 2.160371e-07\\
Fe58 & 2.925431e-08\\
Ni58 & 6.719746e-06\\
Ni60 & 2.677586e-06\\
Ni61 & 1.183357e-07\\
Ni62 & 3.834805e-07\\
Ni64 & 1.008146e-07\\
Cu63 & 6.849595e-01\\
Cu65 & 3.149890e-01\\
Sn & 2.000007e-06\\
Pb206 & 1.197804e-06\\
Pb207 & 1.103704e-06\\
Pb208 & 2.629607e-06\\
Bi209 & 1.000000e-06\\

\caption{Table showing the isotopic description of material M29}
\label{table:material_M29}
\end{longtable}\clearpage

\begin{longtable}[ht!]
{ p{0.3\textwidth} | p{0.3\textwidth} }
\hline
Nuclide & Mass Fraction\\
\hline
\\
B10 & 1.442461e-01\\
B11 & 6.383841e-01\\
C12 & 2.148517e-01\\
C13 & 2.518076e-03\\

\caption{Table showing the isotopic description of material UppExShield}
\label{table:material_UppExShield}
\end{longtable}\clearpage

\begin{longtable}[ht!]
{ p{0.3\textwidth} | p{0.3\textwidth} }
\hline
Nuclide & Mass Fraction\\
\hline
\\
B10 & 1.843098e-06\\
B11 & 8.156930e-06\\
C12 & 2.965249e-04\\
N14 & 6.972393e-04\\
N15 & 2.786798e-06\\
Si28 & 4.593614e-03\\
Si29 & 2.415835e-04\\
Si30 & 1.647335e-04\\
P31 & 2.499975e-04\\
S & 1.000016e-04\\
Ti46 & 7.920085e-05\\
Ti47 & 7.297781e-05\\
Ti48 & 7.381413e-04\\
Ti49 & 5.532184e-05\\
Ti50 & 5.404879e-05\\
Cr50 & 7.303946e-03\\
Cr52 & 1.464747e-01\\
Cr53 & 1.692872e-02\\
Cr54 & 4.293349e-03\\
Mn55 & 1.799997e-02\\
Fe54 & 3.660823e-02\\
Fe56 & 5.959297e-01\\
Fe57 & 1.400876e-02\\
Fe58 & 1.896969e-03\\
Co59 & 4.999999e-04\\
Ni58 & 8.231680e-02\\
Ni60 & 3.280046e-02\\
Ni61 & 1.449607e-03\\
Ni62 & 4.697640e-03\\
Ni64 & 1.234979e-03\\
Cu63 & 2.054981e-03\\
Cu65 & 9.450179e-04\\
Nb93 & 1.000001e-04\\
Mo92 & 3.554367e-03\\
Mo94 & 2.263664e-03\\
Mo95 & 3.937466e-03\\
Mo96 & 4.168845e-03\\
Mo97 & 2.411755e-03\\
Mo98 & 6.156638e-03\\
Mo100 & 2.507290e-03\\
Ta181 & 9.999969e-05\\

\caption{Table showing the isotopic description of material Cryopump}
\label{table:material_Cryopump}
\end{longtable}\clearpage

\begin{longtable}[ht!]
{ p{0.3\textwidth} | p{0.3\textwidth} }
\hline
Nuclide & Mass Fraction\\
\hline
\\
B10 & 1.442742e-01\\
B11 & 6.385081e-01\\
C12 & 2.172177e-01\\

\caption{Table showing the isotopic description of material EPPCN}
\label{table:material_EPPCN}
\end{longtable}\clearpage

\begin{longtable}[ht!]
{ p{0.3\textwidth} | p{0.3\textwidth} }
\hline
Nuclide & Mass Fraction\\
\hline
\\
B10 & 1.843098e-06\\
B11 & 8.156930e-06\\
C12 & 2.965249e-04\\
N14 & 6.972393e-04\\
N15 & 2.786798e-06\\
Si28 & 4.593614e-03\\
Si29 & 2.415835e-04\\
Si30 & 1.647335e-04\\
P31 & 2.499975e-04\\
S & 1.000016e-04\\
Ti46 & 7.920085e-05\\
Ti47 & 7.297781e-05\\
Ti48 & 7.381413e-04\\
Ti49 & 5.532184e-05\\
Ti50 & 5.404879e-05\\
Cr50 & 7.303946e-03\\
Cr52 & 1.464747e-01\\
Cr53 & 1.692872e-02\\
Cr54 & 4.293349e-03\\
Mn55 & 1.799997e-02\\
Fe54 & 3.660823e-02\\
Fe56 & 5.959297e-01\\
Fe57 & 1.400876e-02\\
Fe58 & 1.896969e-03\\
Co59 & 4.999999e-04\\
Ni58 & 8.231680e-02\\
Ni60 & 3.280046e-02\\
Ni61 & 1.449607e-03\\
Ni62 & 4.697640e-03\\
Ni64 & 1.234979e-03\\
Cu63 & 2.054981e-03\\
Cu65 & 9.450179e-04\\
Nb93 & 1.000001e-04\\
Mo92 & 3.554367e-03\\
Mo94 & 2.263664e-03\\
Mo95 & 3.937466e-03\\
Mo96 & 4.168845e-03\\
Mo97 & 2.411755e-03\\
Mo98 & 6.156638e-03\\
Mo100 & 2.507290e-03\\
Ta181 & 9.999969e-05\\

\caption{Table showing the isotopic description of material EMH}
\label{table:material_EMH}
\end{longtable}\clearpage

\begin{longtable}[ht!]
{ p{0.3\textwidth} | p{0.3\textwidth} }
\hline
Nuclide & Mass Fraction\\
\hline
H1 & 1.466991e-03\\
H2 & 3.371874e-07\\
B10 & 1.818988e-06\\
B11 & 8.050226e-06\\
C12 & 2.926459e-04\\
N14 & 6.881184e-04\\
N15 & 2.750343e-06\\
O16 & 1.161415e-02\\
Si28 & 4.533523e-03\\
Si29 & 2.384233e-04\\
Si30 & 1.625786e-04\\
P31 & 2.467271e-04\\
S & 9.869340e-05\\
Ti46 & 7.816479e-05\\
Ti47 & 7.202315e-05\\
Ti48 & 7.284853e-04\\
Ti49 & 5.459815e-05\\
Ti50 & 5.334176e-05\\
Cr50 & 7.208399e-03\\
Cr52 & 1.445586e-01\\
Cr53 & 1.670727e-02\\
Cr54 & 4.237186e-03\\
Mn55 & 1.776451e-02\\
Fe54 & 3.612934e-02\\
Fe56 & 5.881340e-01\\
Fe57 & 1.382551e-02\\
Fe58 & 1.872153e-03\\
Co59 & 4.934592e-04\\
Ni58 & 8.123998e-02\\
Ni60 & 3.237138e-02\\
Ni61 & 1.430644e-03\\
Ni62 & 4.636188e-03\\
Ni64 & 1.218823e-03\\
Cu63 & 2.028099e-03\\
Cu65 & 9.326557e-04\\
Nb93 & 9.869198e-05\\
Mo92 & 3.507870e-03\\
Mo94 & 2.234052e-03\\
Mo95 & 3.885958e-03\\
Mo96 & 4.114311e-03\\
Mo97 & 2.380205e-03\\
Mo98 & 6.076101e-03\\
Mo100 & 2.474491e-03\\
Ta181 & 9.869155e-05\\

\caption{Table showing the isotopic description of material UPDFW3}
\label{table:material_UPDFW3}
\end{longtable}\clearpage

\begin{longtable}[ht!]
{ p{0.3\textwidth} | p{0.3\textwidth} }
\hline
Nuclide & Mass Fraction\\
\hline
H1 & 4.381908e-03\\
H2 & 1.313805e-06\\
Be9 & 4.092101e-03\\
B10 & 1.870951e-06\\
B11 & 7.528208e-06\\
C12 & 2.092166e-04\\
N14 & 6.555165e-04\\
N15 & 2.445979e-06\\
O16 & 3.480110e-02\\
Al27 & 4.699541e-04\\
Si28 & 4.335576e-03\\
Si29 & 2.204930e-04\\
Si30 & 1.452762e-04\\
P31 & 2.349772e-04\\
S & 7.049782e-05\\
K & 4.695553e-06\\
Ti46 & 1.163148e-04\\
Ti47 & 1.048938e-04\\
Ti48 & 1.038917e-03\\
Ti49 & 7.627570e-05\\
Ti50 & 7.303434e-05\\
V & 3.758840e-05\\
Cr50 & 7.152128e-03\\
Cr52 & 1.379272e-01\\
Cr53 & 1.563985e-02\\
Cr54 & 3.893127e-03\\
Mn55 & 1.691837e-02\\
Fe54 & 3.567567e-02\\
Fe56 & 5.600298e-01\\
Fe57 & 1.293363e-02\\
Fe58 & 1.721210e-03\\
Co59 & 4.800693e-04\\
Ni58 & 7.838275e-02\\
Ni60 & 3.019306e-02\\
Ni61 & 1.312469e-03\\
Ni62 & 4.184752e-03\\
Ni64 & 1.065722e-03\\
Cu63 & 1.209131e-02\\
Cu65 & 5.551184e-03\\
Zr & 3.734351e-05\\
Nb93 & 9.399324e-05\\
Mo92 & 3.487083e-03\\
Mo94 & 2.173545e-03\\
Mo95 & 3.740855e-03\\
Mo96 & 3.919447e-03\\
Mo97 & 2.244056e-03\\
Mo98 & 5.670028e-03\\
Mo100 & 2.262832e-03\\
Sn & 1.880066e-05\\
Ta181 & 9.399116e-05\\
W182 & 2.496688e-06\\
W183 & 1.329347e-06\\
W184 & 2.855665e-06\\
W186 & 2.694326e-06\\
Pb206 & 1.812694e-06\\
Pb207 & 1.662714e-06\\
Pb208 & 3.931529e-06\\
Bi209 & 7.516923e-06\\

\caption{Table showing the isotopic description of material FirstWall}
\label{table:material_FirstWall}
\end{longtable}\clearpage

\begin{longtable}[ht!]
{ p{0.3\textwidth} | p{0.3\textwidth} }
\hline
Nuclide & Mass Fraction\\
\hline
\\
B10 & 1.843098e-06\\
B11 & 8.156930e-06\\
C12 & 2.965249e-04\\
N14 & 6.972393e-04\\
N15 & 2.786798e-06\\
Si28 & 4.593614e-03\\
Si29 & 2.415835e-04\\
Si30 & 1.647335e-04\\
P31 & 2.499975e-04\\
S & 1.000016e-04\\
Ti46 & 7.920085e-05\\
Ti47 & 7.297781e-05\\
Ti48 & 7.381413e-04\\
Ti49 & 5.532184e-05\\
Ti50 & 5.404879e-05\\
Cr50 & 7.303946e-03\\
Cr52 & 1.464747e-01\\
Cr53 & 1.692872e-02\\
Cr54 & 4.293349e-03\\
Mn55 & 1.799997e-02\\
Fe54 & 3.660823e-02\\
Fe56 & 5.959297e-01\\
Fe57 & 1.400876e-02\\
Fe58 & 1.896969e-03\\
Co59 & 4.999999e-04\\
Ni58 & 8.231680e-02\\
Ni60 & 3.280046e-02\\
Ni61 & 1.449607e-03\\
Ni62 & 4.697640e-03\\
Ni64 & 1.234979e-03\\
Cu63 & 2.054981e-03\\
Cu65 & 9.450179e-04\\
Nb93 & 1.000001e-04\\
Mo92 & 3.554367e-03\\
Mo94 & 2.263664e-03\\
Mo95 & 3.937466e-03\\
Mo96 & 4.168845e-03\\
Mo97 & 2.411755e-03\\
Mo98 & 6.156638e-03\\
Mo100 & 2.507290e-03\\
Ta181 & 9.999969e-05\\

\caption{Table showing the isotopic description of material UppWater}
\label{table:material_UppWater}
\end{longtable}\clearpage

\begin{longtable}[ht!]
{ p{0.3\textwidth} | p{0.3\textwidth} }
\hline
Nuclide & Mass Fraction\\
\hline
\\
B10 & 1.843098e-06\\
B11 & 8.156930e-06\\
C12 & 2.965249e-04\\
N14 & 6.972393e-04\\
N15 & 2.786798e-06\\
Si28 & 4.593614e-03\\
Si29 & 2.415835e-04\\
Si30 & 1.647335e-04\\
P31 & 2.499975e-04\\
S & 1.000016e-04\\
Ti46 & 7.920085e-05\\
Ti47 & 7.297781e-05\\
Ti48 & 7.381413e-04\\
Ti49 & 5.532184e-05\\
Ti50 & 5.404879e-05\\
Cr50 & 7.303946e-03\\
Cr52 & 1.464747e-01\\
Cr53 & 1.692872e-02\\
Cr54 & 4.293349e-03\\
Mn55 & 1.799997e-02\\
Fe54 & 3.660823e-02\\
Fe56 & 5.959297e-01\\
Fe57 & 1.400876e-02\\
Fe58 & 1.896969e-03\\
Co59 & 4.999999e-04\\
Ni58 & 8.231680e-02\\
Ni60 & 3.280046e-02\\
Ni61 & 1.449607e-03\\
Ni62 & 4.697640e-03\\
Ni64 & 1.234979e-03\\
Cu63 & 2.054981e-03\\
Cu65 & 9.450179e-04\\
Nb93 & 1.000001e-04\\
Mo92 & 3.554367e-03\\
Mo94 & 2.263664e-03\\
Mo95 & 3.937466e-03\\
Mo96 & 4.168845e-03\\
Mo97 & 2.411755e-03\\
Mo98 & 6.156638e-03\\
Mo100 & 2.507290e-03\\
Ta181 & 9.999969e-05\\

\caption{Table showing the isotopic description of material Lenses}
\label{table:material_Lenses}
\end{longtable}\clearpage

\begin{longtable}[ht!]
{ p{0.3\textwidth} | p{0.3\textwidth} }
\hline
Nuclide & Mass Fraction\\
\hline
\\
C12 & 2.965246e-04\\
N14 & 1.693295e-03\\
N15 & 6.767941e-06\\
Si28 & 9.187241e-03\\
Si29 & 4.831665e-04\\
Si30 & 3.294673e-04\\
P31 & 4.499946e-04\\
S & 1.500025e-04\\
Cr50 & 7.721311e-03\\
Cr52 & 1.548449e-01\\
Cr53 & 1.789604e-02\\
Cr54 & 4.538678e-03\\
Mn55 & 2.000003e-02\\
Fe54 & 3.841255e-02\\
Fe56 & 6.253010e-01\\
Fe57 & 1.469916e-02\\
Fe58 & 1.990474e-03\\
Co59 & 9.999998e-04\\
Ni58 & 6.719730e-02\\
Ni60 & 2.677580e-02\\
Ni61 & 1.183353e-03\\
Ni62 & 3.834799e-03\\
Ni64 & 1.008145e-03\\
Nb93 & 1.000000e-03\\

\caption{Table showing the isotopic description of material M102}
\label{table:material_M102}
\end{longtable}\clearpage

\begin{longtable}[ht!]
{ p{0.3\textwidth} | p{0.3\textwidth} }
\hline
Nuclide & Mass Fraction\\
\hline
\\
B10 & 3.317568e-06\\
B11 & 1.468247e-05\\
C12 & 2.965247e-04\\
N14 & 9.960588e-04\\
N15 & 3.981134e-06\\
Si28 & 8.038815e-03\\
Si29 & 4.227711e-04\\
Si30 & 2.882844e-04\\
P31 & 2.999967e-04\\
S & 1.500025e-04\\
Cr50 & 7.930007e-03\\
Cr52 & 1.590294e-01\\
Cr53 & 1.837976e-02\\
Cr54 & 4.661352e-03\\
Mn55 & 2.000002e-02\\
Fe54 & 3.853009e-02\\
Fe56 & 6.272146e-01\\
Fe57 & 1.474418e-02\\
Fe58 & 1.996561e-03\\
Co59 & 1.000000e-03\\
Ni58 & 6.383756e-02\\
Ni60 & 2.543704e-02\\
Ni61 & 1.124186e-03\\
Ni62 & 3.643066e-03\\
Ni64 & 9.577390e-04\\
Nb93 & 1.000001e-03\\

\caption{Table showing the isotopic description of material M103}
\label{table:material_M103}
\end{longtable}\clearpage

\begin{longtable}[ht!]
{ p{0.3\textwidth} | p{0.3\textwidth} }
\hline
Nuclide & Mass Fraction\\
\hline
\\
B10 & 1.843098e-06\\
B11 & 8.156930e-06\\
C12 & 2.965249e-04\\
N14 & 6.972393e-04\\
N15 & 2.786798e-06\\
Si28 & 4.593614e-03\\
Si29 & 2.415835e-04\\
Si30 & 1.647335e-04\\
P31 & 2.499975e-04\\
S & 1.000016e-04\\
Ti46 & 7.920085e-05\\
Ti47 & 7.297781e-05\\
Ti48 & 7.381413e-04\\
Ti49 & 5.532184e-05\\
Ti50 & 5.404879e-05\\
Cr50 & 7.303946e-03\\
Cr52 & 1.464747e-01\\
Cr53 & 1.692872e-02\\
Cr54 & 4.293349e-03\\
Mn55 & 1.799997e-02\\
Fe54 & 3.660823e-02\\
Fe56 & 5.959297e-01\\
Fe57 & 1.400876e-02\\
Fe58 & 1.896969e-03\\
Co59 & 4.999999e-04\\
Ni58 & 8.231680e-02\\
Ni60 & 3.280046e-02\\
Ni61 & 1.449607e-03\\
Ni62 & 4.697640e-03\\
Ni64 & 1.234979e-03\\
Cu63 & 2.054981e-03\\
Cu65 & 9.450179e-04\\
Nb93 & 1.000001e-04\\
Mo92 & 3.554367e-03\\
Mo94 & 2.263664e-03\\
Mo95 & 3.937466e-03\\
Mo96 & 4.168845e-03\\
Mo97 & 2.411755e-03\\
Mo98 & 6.156638e-03\\
Mo100 & 2.507290e-03\\
Ta181 & 9.999969e-05\\

\caption{Table showing the isotopic description of material M100}
\label{table:material_M100}
\end{longtable}\clearpage

\begin{longtable}[ht!]
{ p{0.3\textwidth} | p{0.3\textwidth} }
\hline
Nuclide & Mass Fraction\\
\hline
\\
B10 & 3.686196e-06\\
B11 & 1.631384e-05\\
C12 & 2.965249e-04\\
N14 & 6.972393e-04\\
N15 & 2.786798e-06\\
Si28 & 4.593614e-03\\
Si29 & 2.415835e-04\\
Si30 & 1.647335e-04\\
P31 & 2.499974e-04\\
S & 1.000016e-04\\
Ti46 & 7.920084e-05\\
Ti47 & 7.297780e-05\\
Ti48 & 7.381412e-04\\
Ti49 & 5.532184e-05\\
Ti50 & 5.404879e-05\\
Cr50 & 7.303945e-03\\
Cr52 & 1.464747e-01\\
Cr53 & 1.692872e-02\\
Cr54 & 4.293349e-03\\
Mn55 & 1.799997e-02\\
Fe54 & 3.655695e-02\\
Fe56 & 5.950933e-01\\
Fe57 & 1.398909e-02\\
Fe58 & 1.894312e-03\\
Co59 & 4.999998e-04\\
Ni58 & 8.231679e-02\\
Ni60 & 3.280045e-02\\
Ni61 & 1.449607e-03\\
Ni62 & 4.697639e-03\\
Ni64 & 1.234979e-03\\
Cu63 & 2.054981e-03\\
Cu65 & 9.450178e-04\\
Nb93 & 1.000001e-03\\
Mo92 & 3.554366e-03\\
Mo94 & 2.263664e-03\\
Mo95 & 3.937466e-03\\
Mo96 & 4.168845e-03\\
Mo97 & 2.411754e-03\\
Mo98 & 6.156638e-03\\
Mo100 & 2.507290e-03\\
Ta181 & 9.999968e-05\\

\caption{Table showing the isotopic description of material M101}
\label{table:material_M101}
\end{longtable}\clearpage

\begin{longtable}[ht!]
{ p{0.3\textwidth} | p{0.3\textwidth} }
\hline
Nuclide & Mass Fraction\\
\hline
\\
B10 & 3.317571e-06\\
B11 & 1.468244e-05\\
C12 & 2.965246e-04\\
N14 & 1.095662e-03\\
N15 & 4.379264e-06\\
Si28 & 9.187242e-03\\
Si29 & 4.831665e-04\\
Si30 & 3.294673e-04\\
P31 & 2.999969e-04\\
S & 1.500025e-04\\
Ti46 & 7.920076e-05\\
Ti47 & 7.297772e-05\\
Ti48 & 7.381405e-04\\
Ti49 & 5.532178e-05\\
Ti50 & 5.404873e-05\\
Cr50 & 7.303938e-03\\
Cr52 & 1.464745e-01\\
Cr53 & 1.692870e-02\\
Cr54 & 4.293344e-03\\
Mn55 & 2.000004e-02\\
Fe54 & 3.613923e-02\\
Fe56 & 5.882946e-01\\
Fe57 & 1.382932e-02\\
Fe58 & 1.872668e-03\\
Co59 & 2.000000e-03\\
Ni58 & 7.727699e-02\\
Ni60 & 3.079226e-02\\
Ni61 & 1.360854e-03\\
Ni62 & 4.410023e-03\\
Ni64 & 1.159367e-03\\
Cu63 & 6.849938e-03\\
Cu65 & 3.150061e-03\\
Nb93 & 1.000000e-03\\
Mo92 & 3.198923e-03\\
Mo94 & 2.037293e-03\\
Mo95 & 3.543714e-03\\
Mo96 & 3.751951e-03\\
Mo97 & 2.170581e-03\\
Mo98 & 5.540977e-03\\
Mo100 & 2.256555e-03\\
Ta181 & 1.499994e-03\\

\caption{Table showing the isotopic description of material M106}
\label{table:material_M106}
\end{longtable}\clearpage

\begin{longtable}[ht!]
{ p{0.3\textwidth} | p{0.3\textwidth} }
\hline
Nuclide & Mass Fraction\\
\hline
\\
C12 & 5.930496e-04\\
N14 & 2.988169e-03\\
N15 & 1.194341e-05\\
Si28 & 9.187232e-03\\
Si29 & 4.831671e-04\\
Si30 & 3.294672e-04\\
P31 & 3.999954e-04\\
S & 3.000046e-04\\
V & 1.999904e-03\\
Cr50 & 9.182103e-03\\
Cr52 & 1.841398e-01\\
Cr53 & 2.128179e-02\\
Cr54 & 5.397353e-03\\
Mn55 & 4.999998e-02\\
Fe54 & 3.176235e-02\\
Fe56 & 5.170442e-01\\
Fe57 & 1.215434e-02\\
Fe58 & 1.645869e-03\\
Co59 & 5.000003e-04\\
Ni58 & 8.399666e-02\\
Ni60 & 3.346977e-02\\
Ni61 & 1.479193e-03\\
Ni62 & 4.793502e-03\\
Ni64 & 1.260181e-03\\
Nb93 & 2.999993e-03\\
Mo92 & 3.198924e-03\\
Mo94 & 2.037299e-03\\
Mo95 & 3.543701e-03\\
Mo96 & 3.751951e-03\\
Mo97 & 2.170590e-03\\
Mo98 & 5.540980e-03\\
Mo100 & 2.256557e-03\\
Ta181 & 9.999962e-05\\

\caption{Table showing the isotopic description of material M107}
\label{table:material_M107}
\end{longtable}\clearpage

\begin{longtable}[ht!]
{ p{0.3\textwidth} | p{0.3\textwidth} }
\hline
Nuclide & Mass Fraction\\
\hline
\\
B10 & 3.317572e-06\\
B11 & 1.468245e-05\\
C12 & 6.918901e-04\\
N14 & 1.095662e-03\\
N15 & 4.379265e-06\\
Si28 & 9.187245e-03\\
Si29 & 4.831667e-04\\
Si30 & 3.294674e-04\\
P31 & 2.999970e-04\\
S & 1.500026e-04\\
Ti46 & 7.920079e-05\\
Ti47 & 7.297775e-05\\
Ti48 & 7.381407e-04\\
Ti49 & 5.532180e-05\\
Ti50 & 5.404875e-05\\
Cr50 & 7.616968e-03\\
Cr52 & 1.527524e-01\\
Cr53 & 1.765427e-02\\
Cr54 & 4.477349e-03\\
Mn55 & 2.000004e-02\\
Fe54 & 3.811540e-02\\
Fe56 & 6.204639e-01\\
Fe57 & 1.458555e-02\\
Fe58 & 1.975068e-03\\
Co59 & 4.999995e-04\\
Ni58 & 6.215761e-02\\
Ni60 & 2.476764e-02\\
Ni61 & 1.094601e-03\\
Ni62 & 3.547188e-03\\
Ni64 & 9.325344e-04\\
Cu63 & 6.849941e-03\\
Cu65 & 3.150062e-03\\
Nb93 & 1.000000e-03\\
Mo92 & 7.108708e-04\\
Mo94 & 4.527305e-04\\
Mo95 & 7.874906e-04\\
Mo96 & 8.337684e-04\\
Mo97 & 4.823526e-04\\
Mo98 & 1.231327e-03\\
Mo100 & 5.014577e-04\\
Ta181 & 9.999961e-05\\

\caption{Table showing the isotopic description of material M104}
\label{table:material_M104}
\end{longtable}\clearpage

\begin{longtable}[ht!]
{ p{0.3\textwidth} | p{0.3\textwidth} }
\hline
Nuclide & Mass Fraction\\
\hline
\\
B10 & 3.317573e-06\\
B11 & 1.468245e-05\\
C12 & 2.965248e-04\\
N14 & 1.095662e-03\\
N15 & 4.379266e-06\\
Si28 & 9.187247e-03\\
Si29 & 4.831668e-04\\
Si30 & 3.294675e-04\\
P31 & 2.999971e-04\\
S & 1.500026e-04\\
Ti46 & 7.920081e-05\\
Ti47 & 7.297777e-05\\
Ti48 & 7.381409e-04\\
Ti49 & 5.532181e-05\\
Ti50 & 5.404877e-05\\
Cr50 & 7.721316e-03\\
Cr52 & 1.548450e-01\\
Cr53 & 1.789605e-02\\
Cr54 & 4.538681e-03\\
Mn55 & 2.000005e-02\\
Fe54 & 3.813777e-02\\
Fe56 & 6.208272e-01\\
Fe57 & 1.459402e-02\\
Fe58 & 1.976233e-03\\
Co59 & 4.999996e-04\\
Ni58 & 6.047764e-02\\
Ni60 & 2.409826e-02\\
Ni61 & 1.065019e-03\\
Ni62 & 3.451319e-03\\
Ni64 & 9.073305e-04\\
Cu63 & 6.849942e-03\\
Cu65 & 3.150062e-03\\
Nb93 & 1.000001e-03\\
Mo92 & 7.108710e-04\\
Mo94 & 4.527306e-04\\
Mo95 & 7.874908e-04\\
Mo96 & 8.337687e-04\\
Mo97 & 4.823527e-04\\
Mo98 & 1.231327e-03\\
Mo100 & 5.014578e-04\\
Ta181 & 9.999964e-05\\

\caption{Table showing the isotopic description of material M105}
\label{table:material_M105}
\end{longtable}\clearpage

\begin{longtable}[ht!]
{ p{0.3\textwidth} | p{0.3\textwidth} }
\hline
Nuclide & Mass Fraction\\
\hline
H1 & 6.997078e-04\\
H2 & 1.608276e-07\\
B10 & 1.831598e-06\\
B11 & 8.106036e-06\\
C12 & 2.946747e-04\\
N14 & 6.928888e-04\\
N15 & 2.769410e-06\\
O16 & 5.539577e-03\\
Si28 & 4.564952e-03\\
Si29 & 2.400762e-04\\
Si30 & 1.637057e-04\\
P31 & 2.484376e-04\\
S & 9.937764e-05\\
Ti46 & 7.870667e-05\\
Ti47 & 7.252245e-05\\
Ti48 & 7.335356e-04\\
Ti49 & 5.497666e-05\\
Ti50 & 5.371156e-05\\
Cr50 & 7.258372e-03\\
Cr52 & 1.455608e-01\\
Cr53 & 1.682310e-02\\
Cr54 & 4.266560e-03\\
Mn55 & 1.788766e-02\\
Fe54 & 3.637981e-02\\
Fe56 & 5.922114e-01\\
Fe57 & 1.392136e-02\\
Fe58 & 1.885133e-03\\
Co59 & 4.968802e-04\\
Ni58 & 8.180318e-02\\
Ni60 & 3.259580e-02\\
Ni61 & 1.440563e-03\\
Ni62 & 4.668329e-03\\
Ni64 & 1.227273e-03\\
Cu63 & 2.042160e-03\\
Cu65 & 9.391214e-04\\
Nb93 & 9.937617e-05\\
Mo92 & 3.532189e-03\\
Mo94 & 2.249541e-03\\
Mo95 & 3.912899e-03\\
Mo96 & 4.142835e-03\\
Mo97 & 2.396706e-03\\
Mo98 & 6.118223e-03\\
Mo100 & 2.491646e-03\\
Ta181 & 9.937573e-05\\

\caption{Table showing the isotopic description of material EPPBDY}
\label{table:material_EPPBDY}
\end{longtable}\clearpage

\begin{longtable}[ht!]
{ p{0.3\textwidth} | p{0.3\textwidth} }
\hline
Nuclide & Mass Fraction\\
\hline
\\
B10 & 1.843096e-06\\
B11 & 8.156922e-06\\
C12 & 2.965246e-04\\
N14 & 1.095662e-03\\
N15 & 4.379264e-06\\
Si28 & 9.187242e-03\\
Si29 & 4.831666e-04\\
Si30 & 3.294673e-04\\
P31 & 3.999957e-04\\
S & 1.500025e-04\\
Cr50 & 7.303938e-03\\
Cr52 & 1.464745e-01\\
Cr53 & 1.692870e-02\\
Cr54 & 4.293345e-03\\
Mn55 & 2.000004e-02\\
Fe54 & 3.681150e-02\\
Fe56 & 5.992379e-01\\
Fe57 & 1.408660e-02\\
Fe58 & 1.907509e-03\\
Co59 & 2.000000e-03\\
Ni58 & 7.727700e-02\\
Ni60 & 3.079226e-02\\
Ni61 & 1.360854e-03\\
Ni62 & 4.410023e-03\\
Ni64 & 1.159367e-03\\
Nb93 & 1.000000e-03\\
Mo92 & 3.198923e-03\\
Mo94 & 2.037294e-03\\
Mo95 & 3.543714e-03\\
Mo96 & 3.751951e-03\\
Mo97 & 2.170581e-03\\
Mo98 & 5.540977e-03\\
Mo100 & 2.256555e-03\\
Ta181 & 4.999979e-04\\

\caption{Table showing the isotopic description of material M108}
\label{table:material_M108}
\end{longtable}\clearpage

\begin{longtable}[ht!]
{ p{0.3\textwidth} | p{0.3\textwidth} }
\hline
Nuclide & Mass Fraction\\
\hline
\\
B10 & 1.843112e-05\\
B11 & 8.156993e-05\\
C12 & 7.907392e-04\\
Al27 & 3.500018e-03\\
Si28 & 9.187303e-03\\
Si29 & 4.831706e-04\\
Si30 & 3.294700e-04\\
P31 & 3.999987e-04\\
S & 3.000072e-04\\
Ti46 & 1.683034e-03\\
Ti47 & 1.550790e-03\\
Ti48 & 1.568564e-02\\
Ti49 & 1.175600e-03\\
Ti50 & 1.148546e-03\\
V & 2.999879e-03\\
Cr50 & 6.156236e-03\\
Cr52 & 1.234582e-01\\
Cr53 & 1.426861e-02\\
Cr54 & 3.618704e-03\\
Mn55 & 2.000011e-02\\
Fe54 & 2.947910e-02\\
Fe56 & 4.798757e-01\\
Fe57 & 1.128065e-02\\
Fe58 & 1.527545e-03\\
Co59 & 2.000021e-03\\
Ni58 & 1.713547e-01\\
Ni60 & 6.827897e-02\\
Ni61 & 3.017577e-03\\
Ni62 & 9.778832e-03\\
Ni64 & 2.570789e-03\\
Nb93 & 1.000009e-03\\
Mo92 & 1.777191e-03\\
Mo94 & 1.131836e-03\\
Mo95 & 1.968744e-03\\
Mo96 & 2.084445e-03\\
Mo97 & 1.205889e-03\\
Mo98 & 3.078348e-03\\
Mo100 & 1.253653e-03\\
Ta181 & 5.000035e-04\\

\caption{Table showing the isotopic description of material M109}
\label{table:material_M109}
\end{longtable}\clearpage

\begin{longtable}[ht!]
{ p{0.3\textwidth} | p{0.3\textwidth} }
\hline
Nuclide & Mass Fraction\\
\hline
\\
B10 & 1.843098e-06\\
B11 & 8.156930e-06\\
C12 & 2.965249e-04\\
N14 & 6.972393e-04\\
N15 & 2.786798e-06\\
Si28 & 4.593614e-03\\
Si29 & 2.415835e-04\\
Si30 & 1.647335e-04\\
P31 & 2.499975e-04\\
S & 1.000016e-04\\
Ti46 & 7.920085e-05\\
Ti47 & 7.297781e-05\\
Ti48 & 7.381413e-04\\
Ti49 & 5.532184e-05\\
Ti50 & 5.404879e-05\\
Cr50 & 7.303946e-03\\
Cr52 & 1.464747e-01\\
Cr53 & 1.692872e-02\\
Cr54 & 4.293349e-03\\
Mn55 & 1.799997e-02\\
Fe54 & 3.660823e-02\\
Fe56 & 5.959297e-01\\
Fe57 & 1.400876e-02\\
Fe58 & 1.896969e-03\\
Co59 & 4.999999e-04\\
Ni58 & 8.231680e-02\\
Ni60 & 3.280046e-02\\
Ni61 & 1.449607e-03\\
Ni62 & 4.697640e-03\\
Ni64 & 1.234979e-03\\
Cu63 & 2.054981e-03\\
Cu65 & 9.450179e-04\\
Nb93 & 1.000001e-04\\
Mo92 & 3.554367e-03\\
Mo94 & 2.263664e-03\\
Mo95 & 3.937466e-03\\
Mo96 & 4.168845e-03\\
Mo97 & 2.411755e-03\\
Mo98 & 6.156638e-03\\
Mo100 & 2.507290e-03\\
Ta181 & 9.999969e-05\\

\caption{Table showing the isotopic description of material UppWaterPipes}
\label{table:material_UppWaterPipes}
\end{longtable}\clearpage

\begin{longtable}[ht!]
{ p{0.3\textwidth} | p{0.3\textwidth} }
\hline
Nuclide & Mass Fraction\\
\hline
\\
B10 & 1.843098e-06\\
B11 & 8.156930e-06\\
C12 & 2.965249e-04\\
N14 & 6.972393e-04\\
N15 & 2.786798e-06\\
Si28 & 4.593614e-03\\
Si29 & 2.415835e-04\\
Si30 & 1.647335e-04\\
P31 & 2.499975e-04\\
S & 1.000016e-04\\
Ti46 & 7.920085e-05\\
Ti47 & 7.297781e-05\\
Ti48 & 7.381413e-04\\
Ti49 & 5.532184e-05\\
Ti50 & 5.404879e-05\\
Cr50 & 7.303946e-03\\
Cr52 & 1.464747e-01\\
Cr53 & 1.692872e-02\\
Cr54 & 4.293349e-03\\
Mn55 & 1.799997e-02\\
Fe54 & 3.660823e-02\\
Fe56 & 5.959297e-01\\
Fe57 & 1.400876e-02\\
Fe58 & 1.896969e-03\\
Co59 & 4.999999e-04\\
Ni58 & 8.231680e-02\\
Ni60 & 3.280046e-02\\
Ni61 & 1.449607e-03\\
Ni62 & 4.697640e-03\\
Ni64 & 1.234979e-03\\
Cu63 & 2.054981e-03\\
Cu65 & 9.450179e-04\\
Nb93 & 1.000001e-04\\
Mo92 & 3.554367e-03\\
Mo94 & 2.263664e-03\\
Mo95 & 3.937466e-03\\
Mo96 & 4.168845e-03\\
Mo97 & 2.411755e-03\\
Mo98 & 6.156638e-03\\
Mo100 & 2.507290e-03\\
Ta181 & 9.999969e-05\\

\caption{Table showing the isotopic description of material EppValves}
\label{table:material_EppValves}
\end{longtable}\clearpage

\begin{longtable}[ht!]
{ p{0.3\textwidth} | p{0.3\textwidth} }
\hline
Nuclide & Mass Fraction\\
\hline
H1 & 3.312304e-03\\
H2 & 7.613329e-07\\
B10 & 1.788659e-06\\
B11 & 7.915993e-06\\
O16 & 2.653324e-02\\
Mg & 3.881843e-04\\
Al27 & 2.911380e-05\\
Si28 & 3.566347e-04\\
Si29 & 1.875587e-05\\
Si30 & 1.278942e-05\\
P31 & 1.358633e-04\\
S & 3.881918e-05\\
Cr50 & 3.037802e-04\\
Cr52 & 6.092066e-03\\
Cr53 & 7.040870e-04\\
Cr54 & 1.785663e-04\\
Mn55 & 1.940927e-05\\
Fe54 & 1.095762e-05\\
Fe56 & 1.783737e-04\\
Fe57 & 4.193123e-06\\
Fe58 & 5.678037e-07\\
Co59 & 4.852314e-04\\
Ni58 & 1.956384e-04\\
Ni60 & 7.795500e-05\\
Ni61 & 3.445201e-06\\
Ni62 & 1.116461e-05\\
Ni64 & 2.935106e-06\\
Cu63 & 6.572585e-01\\
Cu65 & 3.022507e-01\\
Zr & 1.067391e-03\\
Sn & 9.704643e-05\\
Ta181 & 9.704585e-05\\
Pb206 & 2.324848e-05\\
Pb207 & 2.142209e-05\\
Pb208 & 5.103877e-05\\
Bi209 & 2.911390e-05\\

\caption{Table showing the isotopic description of material M613}
\label{table:material_M613}
\end{longtable}\clearpage

\begin{longtable}[ht!]
{ p{0.3\textwidth} | p{0.3\textwidth} }
\hline
Nuclide & Mass Fraction\\
\hline
H1 & 7.158264e-04\\
H2 & 1.645325e-07\\
B10 & 3.662661e-06\\
B11 & 1.620970e-05\\
C12 & 2.946323e-04\\
N14 & 6.927907e-04\\
N15 & 2.769006e-06\\
O16 & 5.667191e-03\\
Si28 & 4.564291e-03\\
Si29 & 2.400416e-04\\
Si30 & 1.636824e-04\\
P31 & 2.484018e-04\\
S & 9.936296e-05\\
Ti46 & 7.869531e-05\\
Ti47 & 7.251184e-05\\
Ti48 & 7.334289e-04\\
Ti49 & 5.496867e-05\\
Ti50 & 5.370369e-05\\
Cr50 & 7.257323e-03\\
Cr52 & 1.455394e-01\\
Cr53 & 1.682069e-02\\
Cr54 & 4.265942e-03\\
Mn55 & 1.788512e-02\\
Fe54 & 3.632358e-02\\
Fe56 & 5.912951e-01\\
Fe57 & 1.389977e-02\\
Fe58 & 1.882224e-03\\
Co59 & 4.968081e-04\\
Ni58 & 8.179130e-02\\
Ni60 & 3.259106e-02\\
Ni61 & 1.440349e-03\\
Ni62 & 4.667655e-03\\
Ni64 & 1.227096e-03\\
Cu63 & 2.041861e-03\\
Cu65 & 9.389844e-04\\
Nb93 & 9.936162e-04\\
Mo92 & 3.531668e-03\\
Mo94 & 2.249205e-03\\
Mo95 & 3.912318e-03\\
Mo96 & 4.142226e-03\\
Mo97 & 2.396366e-03\\
Mo98 & 6.117348e-03\\
Mo100 & 2.491290e-03\\
Ta181 & 9.936140e-05\\

\caption{Table showing the isotopic description of material M611}
\label{table:material_M611}
\end{longtable}\clearpage

\begin{longtable}[ht!]
{ p{0.3\textwidth} | p{0.3\textwidth} }
\hline
Nuclide & Mass Fraction\\
\hline
H1 & 1.072682e-03\\
H2 & 2.465560e-07\\
B10 & 9.290169e-08\\
B11 & 4.111511e-07\\
C12 & 2.651897e-05\\
N14 & 8.908000e-06\\
N15 & 3.560435e-08\\
O16 & 8.500081e-03\\
Na23 & 8.943266e-06\\
Mg & 2.463364e-05\\
Al27 & 1.492702e-05\\
Si28 & 3.495610e-05\\
Si29 & 1.838382e-06\\
Si30 & 1.253580e-06\\
P31 & 5.191294e-05\\
S & 7.189985e-06\\
K & 8.942448e-06\\
Ca & 8.943683e-06\\
Ti46 & 7.083168e-07\\
Ti47 & 6.526604e-07\\
Ti48 & 6.601366e-06\\
Ti49 & 4.947581e-07\\
Ti50 & 4.833723e-07\\
Cr50 & 1.615143e-05\\
Cr52 & 3.239032e-04\\
Cr53 & 3.743494e-05\\
Cr54 & 9.493979e-06\\
Mn55 & 5.503136e-06\\
Fe54 & 2.110249e-06\\
Fe56 & 3.435175e-05\\
Fe57 & 8.075214e-07\\
Fe58 & 1.093490e-07\\
Co59 & 3.414588e-05\\
Ni58 & 2.249510e-05\\
Ni60 & 8.963523e-06\\
Ni61 & 3.961413e-07\\
Ni62 & 1.283744e-06\\
Ni64 & 3.374876e-07\\
Cu63 & 6.620143e-02\\
Cu65 & 3.044380e-02\\
Zr & 6.438187e-05\\
Nb93 & 8.943268e-06\\
Mo92 & 1.271505e-05\\
Mo94 & 8.097788e-06\\
Mo95 & 1.408550e-05\\
Mo96 & 1.491317e-05\\
Mo97 & 8.627601e-06\\
Mo98 & 2.202423e-05\\
Mo100 & 8.969334e-06\\
Sn & 5.134141e-06\\
Ta181 & 1.398374e-05\\
W182 & 2.344468e-01\\
W183 & 1.273014e-01\\
W184 & 2.740668e-01\\
W186 & 2.570633e-01\\
Pb206 & 3.406027e-06\\
Pb207 & 3.138446e-06\\
Pb208 & 7.477455e-06\\
Bi209 & 1.558958e-06\\

\caption{Table showing the isotopic description of material M631}
\label{table:material_M631}
\end{longtable}\clearpage

\begin{longtable}[ht!]
{ p{0.3\textwidth} | p{0.3\textwidth} }
\hline
Nuclide & Mass Fraction\\
\hline
H1 & 3.247625e-03\\
H2 & 7.464654e-07\\
B10 & 1.789722e-06\\
B11 & 7.920707e-06\\
C12 & 2.879376e-04\\
N14 & 6.770474e-04\\
N15 & 2.706093e-06\\
O16 & 2.571140e-02\\
Si28 & 4.460584e-03\\
Si29 & 2.345873e-04\\
Si30 & 1.599629e-04\\
P31 & 2.427576e-04\\
S & 9.710555e-05\\
Ti46 & 7.690721e-05\\
Ti47 & 7.086439e-05\\
Ti48 & 7.167649e-04\\
Ti49 & 5.371974e-05\\
Ti50 & 5.248355e-05\\
Cr50 & 7.092425e-03\\
Cr52 & 1.422328e-01\\
Cr53 & 1.643847e-02\\
Cr54 & 4.169015e-03\\
Mn55 & 1.747870e-02\\
Fe54 & 3.554807e-02\\
Fe56 & 5.786717e-01\\
Fe57 & 1.360307e-02\\
Fe58 & 1.842033e-03\\
Co59 & 4.855200e-04\\
Ni58 & 7.993293e-02\\
Ni60 & 3.185056e-02\\
Ni61 & 1.407627e-03\\
Ni62 & 4.561597e-03\\
Ni64 & 1.199214e-03\\
Cu63 & 1.995469e-03\\
Cu65 & 9.176504e-04\\
Nb93 & 9.710414e-05\\
Mo92 & 3.451433e-03\\
Mo94 & 2.198109e-03\\
Mo95 & 3.823438e-03\\
Mo96 & 4.048117e-03\\
Mo97 & 2.341911e-03\\
Mo98 & 5.978344e-03\\
Mo100 & 2.434680e-03\\
Ta181 & 9.710372e-05\\

\caption{Table showing the isotopic description of material EPP2L}
\label{table:material_EPP2L}
\end{longtable}\clearpage

\begin{longtable}[ht!]             
{ p{0.3\textwidth} | p{0.3\textwidth} } 
\hline
Nuclide & Mass Fraction\\
\hline
H1 & 1.087490e-02\\
H2 & 2.499587e-06\\
B10 & 3.078251e-07\\
B11 & 1.362328e-06\\
C12 & 8.848623e-04\\
N14 & 3.061224e-04\\
N15 & 1.223539e-06\\
O16 & 8.609641e-02\\
Si28 & 7.529100e-03\\
Si29 & 3.959642e-04\\
Si30 & 2.700051e-04\\
P31 & 3.393288e-04\\
S & 2.375087e-04\\
Ti46 & 1.322772e-05\\
Ti47 & 1.218838e-05\\
Ti48 & 1.232805e-04\\
Ti49 & 9.239568e-06\\
Ti50 & 9.026949e-06\\
V & 1.269429e-04\\
Cr50 & 6.434173e-03\\
Cr52 & 1.290321e-01\\
Cr53 & 1.491283e-02\\
Cr54 & 3.782086e-03\\
Mn55 & 1.290535e-02\\
Fe54 & 3.869015e-02\\
Fe56 & 6.298195e-01\\
Fe57 & 1.480546e-02\\
Fe58 & 2.004854e-03\\
Co59 & 5.467260e-04\\
Ni58 & 2.246924e-02\\
Ni60 & 8.953234e-03\\
Ni61 & 3.956860e-04\\
Ni62 & 1.282270e-03\\
Ni64 & 3.371005e-04\\
Cu63 & 3.432127e-04\\
Cu65 & 1.578320e-04\\
Nb93 & 2.743787e-04\\
Mo92 & 7.966813e-04\\
Mo94 & 5.073792e-04\\
Mo95 & 8.825497e-04\\
Mo96 & 9.344134e-04\\
Mo97 & 5.405780e-04\\
Mo98 & 1.379961e-03\\
Mo100 & 5.619871e-04\\
Ta181 & 1.670144e-05\\
\caption{Table showing the isotopic description of material M162}
\label{table:material_M162}
\end{longtable}\clearpage                          

\end{centering}

\end{document}
This is never printed
