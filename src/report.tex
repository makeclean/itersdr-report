
\title{Report on the ITER Clite Shutdown Dose rate Calculations}
\author{
  Andrew Davis \\
  Department of Engineering Physics\\
  College of Engineering \\
  The University of Wisconsin-Madison\\
  Madison, Wisconsin, 53706, \underline{USA}
  \and
  Mohamed Sawan \\
  Department of Engineering Physics\\
  College of Engineering \\
  The University of Wisconsin-Madison\\
  Madison, Wisconsin, 53706, \underline{USA}
  \and
  Paul P. H. Wilson \\
  Department of Engineering Physics\\
  College of Engineering \\
  The University of Wisconsin-Madison\\
  Madison, Wisconsin, 53706, \underline{USA}
  \and
  Elliott Biondo \\
  Department of Engineering Physics\\
  College of Engineering \\
  The University of Wisconsin-Madison\\
  Madison, Wisconsin, 53706, \underline{USA}
  \and
  Ahmad Ibrahim \\
  Radiation Transport Group\\
  Oak Ridge National Laboratory \\
  P.O. Box 2008 \\
  Oak Ridge, Tennessee 37831, \underline{USA}
  \and
  Patrick Shriwise \\
  Department of Engineering Physics\\
  College of Engineering \\
  The University of Wisconsin-Madison\\
  Madison, Wisconsin, 53706, \underline{USA}
  \and
  Edward Marriott\\
  Department of Engineering Physics\\
  College of Engineering \\
  The University of Wisconsin-Madison\\
  Madison, Wisconsin, 53706, \underline{USA}
}

\date{\today}

\documentclass[12pt]{article}
%\usepackage[printwatermark]{xwatermark}
%\usepackage{mathptmx}
%\usepackage{fouriernc}
\usepackage{hyperref}
\usepackage{times}
\usepackage{graphicx}
\usepackage{longtable}
\usepackage[acronym]{glossaries}
\usepackage[a4paper, portrait, margin=0.5in]{geometry}
\usepackage[table]{xcolor}
\usepackage[nottoc,numbib]{tocbibind}
\usepackage{subcaption}
\usepackage{multirow}
\usepackage{draftwatermark}
\SetWatermarkText{DRAFT}
\SetWatermarkScale{1}

%% define this page left blank
\newcommand*{\blankpage}{%
\vspace*{\fill}
\begin{center}
 \centering \textbf{This page intentionally left blank}
\end{center}
\vspace{\fill}}

%% to get lof in toc
\renewcommand{\listoffigures}{\begingroup
\tocsection
\tocfile{\listfigurename}{lof}
\endgroup}

%% to get lot in toc
\renewcommand{\listoftables}{\begingroup
\tocsection
\tocfile{\listtablename}{lot}
\endgroup}

%%\makenoidxglossaries
\makeglossaries
%% abreviations
\newacronym{aci}{ACI}{Advanced Computing Initiative}
\newacronym{advantg}{ADVANTG}{AutomateD VAriaNce reducTion Generator}
\newacronym{alara_c}{ALARA}{Analytic and Laplacian Adaptive Radioactivity 
           Analysis}
\newacronym{alara_p}{ALARA - [Principle] - }{As Low As Reasonably Achievable}
\newacronym{bs}{BS}{Bioshield}
\newacronym{cad}{CAD}{Computer Aided Design}
\newacronym{cadis}{CADIS}{Consistent Adjoint Driven Importance Sampling}
\newacronym{chtc}{CHTC}{Centre for High Throughput Computing}
\newacronym{d1s}{D1S}{ Direct One (1) Step}
\newacronym{dag}{DAG}{ Direct Accelerated Geometry}
\newacronym{dagmc}{DAGMC}{ Direct Accelerated Geometry Monte Carlo}
\newacronym{dd}{DD}{ Diagnostics Division}
\newacronym{endf}{ENDF}{Evaluated Nuclear Data File}
\newacronym{ep}{EP}{Equatorial Port}
\newacronym{epp}{EPP}{Equatorial Port Plug}
\newacronym{epi}{EPI}{Equatorial Port Interspace}
\newacronym{fendl}{FENDL}{Fusion Evaluated Nuclear Data Library}
\newacronym{fom}{FOM}{Figure of Merit}
\newacronym{fwcadis}{FW-CADIS}{Forward Weighted - Consistent Adjoint Driven 
            Importance Sampling}
\newacronym{gepp}{GEPP}{Generic Equatorial Port Plug}
\newacronym{icrp}{ICRP}{International Commission on Radiation Protection}
\newacronym{io}{IO}{ITER Organisation}
\newacronym{lp}{LP}{Lower Port}
\newacronym{lpp}{LPP}{Lower Pumping Port}
\newacronym{mcnp}{MCNP}{Monte Carlo N Particle}
\newacronym{moab}{MOAB}{Mesh Oriented datABase}
\newacronym{ornl}{ORNL}{Oak Ridge National Laboratory}
\newacronym{pi}{PI}{Port Interspace}
\newacronym{pfc}{PFC}{Poloidal Field Coil}
\newacronym{pp}{PP}{Port Plug}
\newacronym{pyne}{PyNE}{Python for Nuclear Engineering}
\newacronym{r2s}{R2S}{Rigorous Two [2] Step}
\newacronym{sdr}{SDR}{Shutdown Dose Rate}
\newacronym{ta}{TA}{Task Agreement}
\newacronym{up}{UP}{Upper Port}
\newacronym{upp}{UPP}{Upper Port Plug}
\newacronym{upi}{UPI}{Upper Port Interspace}
\newacronym{uw}{UW}{The University of Wisconsin}
\newacronym{ve}{VE}{Vessel Extension}
\begin{document}
\maketitle
\newpage
\tableofcontents
\newpage
\listoffigures
\newpage
\listoftables
\newpage
\section*{Acronyms}
 \gls{aci} \\ 
 \gls{advantg} \\ \gls{alara_c} \\
 \gls{alara_p} \\ \gls{bs} \\
 \gls{cad} \\
 \gls{cadis} \\  \gls{chtc} \\
 \gls{d1s} \\
 \gls{dag} \\
 \gls{dagmc} \\ \gls{dd} \\
 \gls{endf} \\ \gls{ep} \\
 \gls{epp} \\ \gls{epi} \\
 \gls{fendl} \\ \gls{fom} \\
 \gls{fwcadis} \\
 \gls{gepp} \\ \gls{icrp} \\
 \gls{io} \\ \gls{lp} \\
 \gls{lpp} \\ \gls{mcnp} \\
 \gls{moab} \\ \gls{ornl} \\
 \gls{pi} \\ \gls{pyne} \\
 \gls{pfc} \\ \gls{pp} \\
 \gls{r2s} \\ \gls{sdr} \\
 \gls{ta} \\ \gls{up} \\
 \gls{upp} \\ \gls{upi} \\
 \gls{ve}
% \printnoidxglossaries[type=\acronymtype]
%\printglossary
%\printglossary[type=\acronymtype]
% none of these work, but should

\newpage
\section*{Executive Summary}
The results contained within this report show the results of complex 3D neutron
\& photon transport simulations to determine the shutdown photon dose rate
resulting from the neutron activation of structural materials within a
representative model of the ITER device. There are several ongoing questions
regarding the correct level of \gls{sdr} photon dose in the equatorial and upper
port regions of ITER.
\\
\\
The neutron results showed some effect of the introduction of the B$_4$C liner 
to the bioshield, neutron fluxes were depressed near to the bioshield by an 
order of magnitude. The largest reduction was in the lowest thermal groups, 
which impacts the (n,$\gamma$) capture reactions the most.
\\
\\
The activation results showed the for the first decay time there is little 
difference between the baseline and with B$_4$C. For the second decay time, 
the was an order of magnitude reduction in the photon source density for 
steel components near the bioshield in the B$_4$C case. For the third decay
time there are nominal differences between the two cases.
\\
\\
The photon results showed that the doserate there was some benefit to the 
the B$_4$C liner. Dose rates across all decay times were consistently lower
by factors of 2-4 for the with B$_4$C case for all the ports. The biggest effect
was present in the upper and equatorial ports. However, for decay times greater
than 10$^6$ seconds the dose rate in all the port interspaces was greater than
than 100 $\mu$Sv $hr^{-1}$. 
\\
\\
This analysis focused on only shutdown photon sources, and therefore does not
contain prompt photon doses.
\newpage
\blankpage
\newpage
\section*{Abstract}
This report provides the methology, input data, assumptions and results for a
full analysis of neutron, neutron induced activation and the subsequent
transport of residual decay photons in a reference ITER \gls{cad} model. The 
purpose of the analysis was to (1) establish a baseline result for the shutdown 
photon dose rate around the ITER device and (2) determine the effect upon the 
\gls{sdr} of a thin B$_4$C layer added to the plasma-side of the bioshield. A 
differentiating factor in this analysis relative to others is the level of 
detail present in the ports. The \gls{up} contained a detailed diagnostic model 
along side detailed \gls{pi} equipment up to the bioshield plug. Similarly, the 
\gls{ep} contained the \gls{dd} model of an \gls{epp} with diagnostic drawers 
and significant quantities of internal B$_4$C shielding, the \gls{ep} interspace
contained detailed models of rails, racks and support frames out to the 
bioshield. The \gls{lpp} contained a detailed model of the cryopump. 
\newpage
\section{Purpose}
\subsection{Problem Statement}
The shutdown dose rate in an around the equatorial and upper ports determine the
type and duration of maintenance that can be performed by person access. Thus,
minimization of the dose rate is desirable and indeed encouraged by the 
\gls{alara_p} principles used at ITER. It was recently suggested in the 
Neutronics Task Force that lining the plasma side of the concrete bioshield 
with a thin ($\sim$ mm thickness) layer of boron carbide (B$_4$C). The purpose 
of the layer is to absorb much of the thermal flux which would otherwise lead 
to neutron induced activation, typically (n,$\gamma$) capture reactions. Simple 
scoping calculations have suggested that the thermal neutron flux should be 
depressed significantly and lead to significantly lower shutdown photon dose 
rates, close to an order of magnitude. The main goal of this report was 
therefore to examine if the use of the B$_4$C liner does indeed lead to lower 
dose rates, this would be achieved by modelling a detailed 40$^{\circ}$ sector 
of the ITER device once with the layer included and once without and then 
examining the detailed effect this layer has upon neutron transport, the 
subsequent neutron induced activation and the resultant photon transport.
\subsection{Initiating Documents}
The ITER Task Agreement under which this work was initiated and performed is TA
C74TD21FU. Several documents provided the basis of the materials for components,
and several \gls{cad} models were provided by numerous \gls{io} groups.
\\
\\
This report fulfills deliverable 7 of the task agreement.
\newpage
\section{Solution Methodology}
\subsection{Background}
\subsubsection{Rigorous Two Step (R2S) Method}
The \gls{r2s} method \cite{r2s} is the primary method of generating shutdown
photon dose rates and fluxes for fusion devices. The method consists of 3 main
steps.  Neutron transport simulations generate a detailed three-dimensional
map of multi-group neutron fluxes.  These are used to predict the induced
activity of the structural material, resulting in three-dimensional maps of
the energy-dependent photon emission density for different times after
shutdown.  For each time after shutdown, those photon emission densities are
used as the source for photon transport to determine a three-dimensional map
of the photon dose.  Improvements on the original method rely on mesh based
flux determination methods \cite{mcr2s,r2smesh,r2suned,pyne_r2s} that allow
more fine neutron flux gradients to be represented in resultant activation
source.  Since the \gls{r2s} method performs a complete activation calculation
using a standalone nuclear inventory code, it is capable of handling the
non-linear variations of nuclide density induced by (n,xn) reactions
that \gls{d1s} methods traditionally struggle with.

\subsubsection{Variance Reduction and the FW-CADIS Method}
The neutron and photon transport steps of the above \gls{r2s} methodology are
performed with Monte Carlo radiation transport methods in order to best
capture the effects of streaming through narrow ducts and channels.  At the
same time, problems such as these are characterized by deep penetration for
which Monte Carlo simulation is known to be computationally ineffecient.
Variance reduction techniques can be employed to improve the computational
efficiency without introducing approximations to the transport problem.
\\
\\
For \gls{r2s} problems, it is desirable to predict the neutron flux throughout
the entire domain of the problem with equal statistical accuracy even though
it originates from a relatively small region of the problem in the plasma
chamber.  The \gls{fwcadis} method was originally developed to increase the
efficiency of performing Monte Carlo calculations of spatial distributions
over large domains (e.g., mesh tallies of fluxes or dose rates), as well as
responses at multiple localized detectors and spectra. The basis of this
method is the development of an importance function that represents the
importance of particles to the objective of uniform Monte Carlo particle
density in the desired tally regions.  Implementation of this method utilizes
the results from a forward deterministic calculation to develop a
forward-weighted adjoint source for a deterministic adjoint calculation. The
resulting adjoint function is then used to generate a consistent space- and
energy-dependent source biasing parameters and weight windows that are used in
a forward Monte Carlo calculation to obtain more uniform statistical
uncertainties in the desired tally regions \cite{wagnerNSEFWCADIS}. This
method was used to accelerate the neutron Monte Carlo calculation of
the \gls{r2s} \gls{sdr} analysis in this work.  Since the photon source is
distributed broadly in space, the impact of variance reduction for a broadly
distributed photon dose map is less obvious, and was not used in this problem.

\subsection{Input Description}
\subsubsection{The Model Geometry}
The \gls{cad} model was generated from several sources, ITER Clite \gls{cad}
which was used as input to generate the ITER Clite V1 \gls{mcnp} model, ITER
Diagnostics Division \gls{cad} models of Upper and Equatorial ports, and
previous \gls{mcnp} analysis which were detailed in \cite{cad_origination}. The
model represents several ITER systems in high levels of detail, with all detail
retained in the \gls{epi}, the  \gls{upi}, and the Lower Port Cryopump. Some of
these models were previously used to generate \gls{mcnp} input decks so have
undergone some degree of simplification prior to being provided for use on this 
problem.
\\
\\
Several modifications were nessary for the \gls{cad} to be fully usable. Several
person months of time was invested in cleaning, reparing and building multiple
components multiple times. 
\\
\\
The overall \gls{cad} model in shown in Figure \ref{fig:cad_iter_global}, the
broad details of the model can be seen. 
\begin{figure}[ht!]
  \centering
  \includegraphics[scale=0.8]{../plots/cad/global.png}
  \caption{Section showing the overall model}
  \label{fig:cad_iter_global}
\end{figure}

\begin{figure}[p]
  \centering
  \includegraphics[scale=0.32]{../plots/cad/mats/label_1.png}
  \includegraphics[scale=0.32]{../plots/cad/mats/label_2.png}
  \caption{Section some of the major tokamak materials}
  \label{fig:material_assign_1}
\end{figure}

\begin{figure}[p]
  \centering
  \includegraphics[scale=0.32]{../plots/cad/mats/label_3.png}
  \caption{Section some of the major tokamak materials}
  \label{fig:material_assign_2}
\end{figure}

\newpage
\clearpage
\subsubsection{Materials}
All materials in the problem originate from the Clite \gls{mcnp} model -
CLITE\textunderscore V1\textunderscore REV131031\textunderscore MOD, with the
exception of the \gls{upp}, the \gls{epp}, the
Cryopump, the blanket modules and the contents of the \gls{upi} and \gls{epi}.
The default cross section set used was \gls{fendl}-2.1, with the exception of XX
which was expanded into an isotopic definition and used the \gls{endf}-VIIR1
evaluation. The \gls{epp} material definitions were taken from the \gls{gepp}
\gls{mcnp} model\cite{epp_materials}. The \gls{upp} material definitions were
provided by \cite{bertalot_communication}. The Cryopump materials assignments
came from \cite{cryopump_communication}. The blanket module material
compositions were taken from the Bl-lite \gls{cad}model since the blankets were
homogenized identically in this model. The full evaluated material definitions
can be found in the Appendix.
\newpage
\subsection{Software Programs \& Validation Status}
\subsubsection{DAGMC}
The \gls{dagmc} is a toolkit developed at the \gls{uw}
designed specifically for ray tracing efficiently on complex \gls{cad} based
geometries. The \gls{dagmc} toolkit is distributed as part of \gls{moab} and
is completely open source. Input to \gls{dagmc} is a \gls{moab} mesh file
which contains the faceted representation of the \gls{cad} geometry, this
is loaded and Oriented Bounding Box (OBB) acceleration structures are built
to speed each ray fire query.
\subsubsection{DAG-MCNP5}
DAG-MCNP5 \cite{dagmc} is a version of \gls{mcnp} \cite{mcnp} where the core ray
queries (point in volume, ray fire, next volume, etc) have their \gls{mcnp}
versions replaced with the \gls{dagmc} equivalents. DAG-MCNP5 has been validated
\cite{dagmc_validation} and tested in several ITER analysis in the past.
\subsubsection{ADVANTG}
The \gls{advantg} software automates the 
process of generating variance reduction parameters for continuous-energy Monte 
Carlo simulations of fixed-source neutron, photon, and coupled neutron-photon 
transport problems using \gls{mcnp}5 \cite{advantg}. \gls{advantg} generates 
space and  energy-dependent mesh-based weight-window bounds and biased source 
distributions using three-dimensional discrete ordinates solutions of the 
adjoint transport equation that are calculated by the Denovo package 
\cite{denovo}. 
\\
\\
\gls{advantg} implements the \gls{cadis} method \cite{wagnerNSECADIS} and the 
\gls{fwcadis} method \cite{wagnerNSEFWCADIS} for generating variance reduction 
parameters. The \gls{cadis} and \gls{fwcadis} methods provide a prescription for
generating space and energy-dependent weight-window targets and a consistent 
biased source distribution. The \gls{cadis} method was developed for accelerating
individual tallies, whereas \gls{fwcadis} was developed for multiple tallies and
mesh tallies. The \gls{cadis} method has been demonstrated to provide speed-ups 
in the tally \gls{fom} of $O(10^{1}-10^{4})$ across a broad range of radiation 
detection and shielding problems. The FW-CADIS method has been shown to produce
 relatively uniform statistical uncertainties across multiple cell tallies and 
large space- and energy-dependent mesh tallies in real-world applications.
\\
\\
\gls{advantg} was updated to support DAG-MCNP calculations. This was 
accomplished by linking the \gls{advantg} mapping functions that are used to 
map the Monte Carlo geometry onto the Denovo mesh to the DAG-MCNP 
ray-tracing routines \cite{biondoMC2015}. With this new capability, 
\gls{advantg} was able to map the geometry of the Clite \gls{cad} model onto a 
Cartesean mesh to use Denovo to generate the variance reduction parameters 
for the DAG-MCNP5 calculations of the neutron fluxes.
\subsubsection{PyNE}
The \gls{pyne} \cite{Scopatz2012b} project aims to
make C++ wrapped Python-accessible library of well validated and tested
functions and capabilities available to nuclear engineers across the world.
For this project we used the following capabilities of \gls{pyne}.
\begin{itemize}
  \item{the \texttt{pyne.mesh} class to combine meshes, with an appropriate
        combined statistical error}
  \item{the \texttt{pyne.r2s} class to produce \gls{r2s} inputs and outputs}
  \item{the \texttt{pyne.alara} class to produce alara inputs}
  \item{the \texttt{pyne.mcnp} class to read \gls{mcnp} meshtal files}
\end{itemize}
The PyNE capabilities are interoperable with \gls{dagmc} output and handles
most post-process operations in the analysis.
\subsubsection{ALARA}
\gls{alara_c} \cite{alara} is a nuclear inventory code developed at the \gls{uw}
The unique feature of \gls{alara_c} is the way it handles the
pathways of the reactions, all of the potential pathways are accounted for and
duplicate links linearised. \gls{alara_c} is used in this analysis as the method
of producing gamma ray sources for \gls{r2s} shutdown photon calculations.
\subsection{Validation Status}
\subsubsection*{Analysis Codes}
The 1$^{st}$ rate analsysis codes used in these calculations are shown in Table 
\ref{table:validation}. These codes were used to directly produce results shown
in this report.
\begin{centering}
 \begin{table}[ht!]
  \begin{tabular}{c | c | c | c}
  \hline
  Code & Version & Validation Status & References \\
  \hline 
  DAG-MCNP5 & 1.0 & Production & \cite{dagmc_validation}\\
  ALARA & 2.9.1 & Production & \cite{alara}\\
  ADVANTG & 3.0.2 & Production & \cite{advantg}\\
  Denovo & 5.3 & Production & \cite{denovo} \\
  yt & 3.2 & Production & \\
  \end{tabular}
 \caption{}
 \label{table:validation}
 \end{table}
\end{centering}
\subsubsection*{Auxilliary Codes}
The auxiallary codes used in the this report are shown in Table 
\ref{table:validation_aux}. These codes were not directly used to 
produce results, but were used to either process or produce input
and output data.
\begin{centering}
 \begin{table}[ht!]
  \begin{tabular}{c | c | c}
  \hline
  Code & Validation Status & References \\    
  \hline
  pyne.mesh & Validated by unit tests & - \\
  pyne.r2s & Validated & \cite{pyne_r2s} \\
  pyne.source\_sampling & Validated & \cite{pyne_r2s} \\
 \end{tabular}
 \caption{}
 \label{table:validation_aux}
 \end{table}
\end{centering}
\newpage
\subsection{Solution Methodology}
\label{section:method}
\subsubsection{Neutron Transport}
The neutron transport calculations were performed using version 1.0 of DAG-MCNP5
using the source from version 1.60 of MCNP5, from the develop branch with git
checkout \texttt{e12e2a3}. The two batches of calculations were performed, one
case with the B$_4$C liner and one without.
\\
\\
The level of detail present in the faceted geometry files impacts the amount of
memory required to run the calculation, this combined with the large and
detailed weight window file and the need for 175 energy group neutron spectra
in each voxel means that the calculation requires significant memory resources,
in excess of 12 Gb per CPU. It was therefore determined that in order to perform
the neutron transport in a timely fashion given the memory requirements, this
set of simulations would be considered a good case to run using a high
throughput methodology with domain replication, this procedure is shown in 
Figure \ref{fig:mesh_splitting}.
\\
\\
In this method the spatial domain of the problem is duplicated and a given
meshtally superimposed onto the geometry. Several single core simulations are
launched for that specific meshtally but the random number seed is strided in
such a way that no CPU will share the same seed with another job, this is
performed using the \texttt{rand hist=n} command in \gls{mcnp}.
\begin{figure}[ht!]
  \centering
  \includegraphics[scale=0.3]{../plots/method/neutron_method.png}
  \caption{The high throughput methodology, splitting a given MCNP calculation
           into smaller calculations spanning a contiguous random number seed
           space}
  \label{fig:mesh_spliting}
\end{figure}
The key feature of this method is that the combined sub results are identically
equal to single MCNP calculation covering the whole original range of particles
simulated.
\begin{figure}[ht!]
  \centering
  \includegraphics[scale=0.4]{../plots/transport/job_splits.png}
  \caption{The seven different mesh domains, (left) shown with some transparency
           to indicate where the major boundaries end and (right) showing the
           absolute scale of the mesh elements}
  \label{fig:mesh_domains}
\end{figure}

The calculation was split into seven calculation domains, shown in Figure
\ref{fig:mesh_domains}, which bound the regions of the problem requiring
activation. The physical size and boundaries that were used in the calculation
can be found in Table \ref{table:mesh_sizes}. With \gls{r2s} calculations there 
is a subtle interplay between the size of the mesh elements and the number of mesh
elements; the larger the mesh elements the lower the statistical errors will be
for a given fixed number of particles but the mesh will conform poorly to the
geometry, and vice versa. In this calculation the maximum number of mesh
elements in a given mesh is dictated by the size that the given mesh will
occupy in memory when transferred to an \gls{alara_c}geometry. 

\begin{centering}
 \begin{table}[ht!]
  \begin{tabular}{c | c | c | c | c}
  \hline 
  Mesh & Dimension & Start position (cm) & End position (cm) & Number of bins\\
  \hline 
  \multirow{3}{*}{Mesh 1} & X & 0.0 & 500.0 & 25 \\ & Y & -250.0 & -250.0 & 25 \\
  & Z & -1500.0 & 1900.0 & 170 \\
  \hline
  \multirow{3}{*}{Mesh 2} & X & 500.0 & 1100.0 & 30 \\ & Y & -500.0 & 500.0 & 50\\
  & Z & 400.0 & 1900.0 & 75 \\
  \hline
  \multirow{3}{*}{Mesh 3} & X & 500.0 & 1100.0 & 30 \\ & Y & -500.0 & 500.0 & 50 \\
  & Z & -1500.0 & -300.0 & 60 \\
  \hline
  \multirow{3}{*}{Mesh 4} & X & 500.0 & 1100.0 & 30 \\ & Y & -500.0 & 500.0 & 50 \\
  & Z & -300.0 & 400.0 & 35 \\
  \hline
  \multirow{3}{*}{Mesh 5} & X & 1100.0 & 1700.0 & 30 \\ & Y & -600.0 & 600.0 & 60 \\
  & Z & -1500.0 & -300.0 & 60 \\
  \hline
  \multirow{3}{*}{Mesh 6} & X & 1100.0 & 1700.0 & 60 \\ & Y & -600.0 & 600.0 & 60 \\
  & Z & 400.0 & 1900.0 & 75\\
  \hline
  \multirow{3}{*}{Mesh 7} & X & 1100.0 & 1700.0 & 30 \\ & Y & -600.0 & 600.0 & 60 \\
  & Z & -300.0 & 400.0 & 36 
  \end{tabular}
 \caption{The meshes used in the problem, their start and end coordinate and
          number of divisions}
 \label{table:mesh_sizes}
 \end{table}
\end{centering}
\subsubsection{Activation}
The activation calculations are performed using the spatially dependent neutron
spectra determined in the previous step. The activation calculations proceed as
a set of standard mesh based \gls{r2s} calculations typically proceed since the 
meshes do not overlap, this is shown schematically in Figure
\ref{fig:activation_method}
\begin{figure}[ht!]
  \centering
  \includegraphics[scale=0.3]{../plots/method/activation_method.png}
  \caption{The methodology of having several activation meshes distributed
           across a given problem to generate several spatially discrete
           decay sources}
  \label{fig:activation_method}
\end{figure}
The geometry under each voxel in the mesh is queried to 
determine the average material composition by volume fraction, a material 
description is developed and recorded for running in an inventory code. In the 
case of \texttt{pyne.r2s} a full \gls{alara_c} input deck is produced which can
then be executed and the final shutdown photon sources produced. From a single
inventory calculation the decay photon sources for all the subsequent decay
times can be generated.  
\subsubsection{Photon Transport}
The photon transport proceeds using the same computational model as in the 
neutron calculation, with the exception the photon source and appropriate 
normalisation is determined using the activation step. The calculation is then
performed, but each photon transport will use a common result mesh for each 
calculation, and therefore, since each photon source is determined for a 
specific non overlapping region of the geometry the final result is the 
summation of each mesh, with the statistical error propagated appropriately.
\subsection{Software Risk Assessment}
\clearpage
\newpage
\section{Assumptions and Engineering Judgements}
\subsection{Neutron Source}
As used in typical ITER analysis the reference ITER neutron source with
reflecting boundary conditions is assumed to adequately represent the
true 360$^{\circ}$ neutron source. The ITER supplied MCNP SDEF was used
completely unmodified. The on load neutron data is used un-normalised
in the \gls{r2s} calculation since the neutron source normalisation is scaled
depending on average power for the pulse being modelled. However for reference,
the highest power used in the activation calculation was 500 MW, which is
equivalent to 1.98$\times$10$^{18}$ neutrons per second in the 40$^{\circ}$
sector. Given that this is the IO approved neutron source description, it is
assumed to be adequate for this simulation.
\subsection{Irradiation Scenario}
The irradiation scenario used in this work was the MDRG-2, shown in Table
\ref{tab:irrad_scenario} scenario as opposed to the typical SA-2 scenario.
The MDRG-2 scenario is more aggressive in overall fluence than the SA-2
scenario, representing a full ITER lifetime as opposed to 2 operation
cycles of the SA-2 scenario.
\begin{table}[ht!]
   \begin{tabular}{| l | c |}
      \hline 
      Irradiation Period & Fractional Strength \\
      \hline
      5 Years & 0.0095 \\
      1 Year  & 0.0127 \\
      1 Year  & 0.0190 \\
      1 Year  & 0.0317 \\
      1 Year  & 0.0380 \\
      1 Year  & 0.0380 \\
      6 Days  & 0.1929 \\
      1 Year  & 0.0380 \\
      \cellcolor{blue!25} 400 Seconds & 1.0000 \\
      \cellcolor{blue!25}1673.6 Seconds & 0.0000 \\
      400 Seconds & 1.0000 \\
      \hline
\end{tabular}
\caption{The table shows the MDRG-2 scenario for irradiation, note the
         cells in \textcolor{blue!25}{blue} are repeated 249 times}
\label{tab:irrad_scenario}
\end{table}
It is assumed that the longer fluence operation will lead to greater accumulated
radionuclides, which should provide a conservative estimate of shutdown photon
dose rate.
\\
\\
In this report time periods following machine shutdown are of interest,
1.0$\times$10$^5$ s, 1.0$\times$10$^6$ s and 1.0$\times$10$^6$ s, approximately
corresponding to 27 hours, 14 days, and 116 days post shutdown.
\subsection{Geometry}
The geometry used in this simulation was collected from a number of sources,
some models were generated directly from CAD drawings supplied by the
drawing office of US ITER and the IO, some were intermediate CAD files produced
as part of other simulations, and some were generated by \gls{uw} to meet a
perceived deficiency in the file. Irrespective of this, the \gls{dagmc} method
allows the exact representation of the underlying CAD to be used without
approximation.
\\
\\
The geometry is defined starting at the centre column (x = 0.0 cm) and ends
beyond the bioshield (x = 1800 cm), truncation of details relative to the full
ITER device CAD model begins on the port cell side of the bioshield plugs,
results beyond the bioshield should therefore be not be used for any
analysis purposes. 
\subsection{Variance Reduction}
The weight window parameters were generated using the \gls{ornl} code
\gls{advantg}, specifically using DAG-\gls{advantg}, which allows a
\gls{dagmc} geometry to be read and the geometry and material data handed off to
Denovo. The weight windows used for the problem are shown below in Figure
\ref{fig:wwinp}. The weight windows were produced using the \gls{fwcadis},
which attempts to produce a weight window map which tries to get particles
everywhere in the model, as opposed to \gls{cadis} for example, which
attempts to get results to a small number of specific regions.
\begin{figure}[ht!]
  \centering
  \includegraphics[trim={0cm 9cm 0cm 10cm},clip,scale=0.75]{../plots/wwinp/Neutron_Weight_Windows.png}
  \caption{Figure showing the ADVANTG produced neutron weight windows for y = 0.0 cm,
  z = 500.0 cm, z = 0.0 cm and z = -500.0 cm}
  \label{fig:wwinp}
\end{figure}
Several artifacts are worthy of note, despite being a deterministic code Denovo
has handled the streaming of neutrons along the divertor duct well as shown in
left hand image of Figure \ref{fig:wwinp}. The overall span of the weight window
is seven orders of magnitude, which is largely proportional to the neutron
attenuation from the plasma to the port interspace, previous calculations put
the attenuation in the range of 6 to 8 orders of magnitude. There is some
evidence of ray effects in the weight window solution in the right hand image
of Figure \ref{fig:wwinp} due to the neutron attenuation of some \gls{epi}
shielding around the diagnostic equipment, but this is small deviation in an
otherwise reasonable solution.
\\
\\
It should be noted that \gls{ornl} specifically developed the 360$^{\circ}$
rotation feature for DAG-\gls{advantg} that allows rotationally symmetric
bodies, which is the reason for the weight window values beyond the
40$^{\circ}$ sector.
\subsection{Materials and Nuclear Data}
\subsubsection{Materials}
The full description of material contents can be found in the Appendix. The 
origin of each material is described in Table \ref{tab:material_origin}. 

\begin{centering}
 \begin{longtable}[ht!]{ p{0.2\textwidth} | p{0.3\textwidth} | p{0.3\textwidth} }
  \hline 
  Material Number & Description & Origin \\
  \hline
  M29  & Copper &  CLITE V1 \\
  M74  & Tungsten &  CLITE V1 \\
  M100  & 316L(N)-IG &  CLITE V1 \\
  M101  & &  CLITE V1 \\
  M102  & &  CLITE V1 \\
  M103  & &  CLITE V1 \\
  M104  & &  CLITE V1 \\
  M105  & &  CLITE V1 \\
  M106  & &  CLITE V1 \\
  M107  & &  CLITE V1 \\
  M108  & &  CLITE V1 \\
  M109  & &  CLITE V1 \\
  M110  & &  CLITE V1 \\
  M111  & &  CLITE V1 \\
  M162  & &  CLITE V1 \\
  M170  & &  CLITE V1 \\
  M200  & &  CLITE V1 \\
  M303  & &  CLITE V1 \\
  M400  & &  CLITE V1 \\
  M601  & &  CLITE V1 \\
  M602  & &  CLITE V1 \\
  M603  & &  CLITE V1 \\
  M611  & &  CLITE V1 \\
  M613  & &  CLITE V1 \\
  M621  & &  CLITE V1 \\
  M622  & &  CLITE V1 \\
  M623  & &  CLITE V1 \\
  M631  & &  CLITE V1 \\
  M906  & &  CLITE V1 \\
  M907  & &  CLITE V1 \\
  M908  & &  CLITE V1 \\
  M910 & B$_4$C & Bespoke  \\
  M911 & Al T6061 & Bespoke  \\
  M912 & Al T6061 & Bespoke  \\
  M913 & Copy M100 & CLITE V1\\
  M914 & Copy M100 & CLITE V1\\
  M915 & Copy M910 & Bespoke \\
  M916 & Copy M100 & CLITE V1\\
  M917 & Cryopipes & \cite{cryopump_communication}\\
  M918 & Copy M100 & CLITE V1 \\
  M920 & Copy M100 & CLITE V1 \\
  M921 & Copy M100 & CLITE V1 \\
  M922 & Copy M100 & CLITE V1 \\
  M923 & Copy M100 & CLITE V1 \\
  M924 & Copy M100 & CLITE V1 \\
  M925 & Copy M100 & CLITE V1 \\
  M926 & Copy M100 - Frame Wheels & \cite{bertalot_communication}\\
  M927 & Copy M100 - Frame Wheels Drive & \cite{bertalot_communication}\\
  M928 & Copy M100 - Port Plug Frame & \cite{bertalot_communication}\\
  M929 & Copy M100 - Port Plug Pipes & \cite{bertalot_communication}\\
  M930 & Copy M100 & \cite{cryopump_communication}\\
  M931 & M100:0.7 M400:0.3 & \cite{bertalot_communication}\\
  M932 & M100:0.9 M400:0.1 & \cite{bertalot_communication}\\
  M933 & M100:0.9 M400:0.1 & \cite{bertalot_communication}\\
  M934 & M100:0.97 M400:0.03 & \cite{bertalot_communication}\\
  M937 & Copy M100 & \cite{cryopump_communication}\\
  M938 & Copy M100 & \cite{cryopump_communication}\\
  M939 & Copy M100 & \cite{cryopump_communication}\\
  M935 & First Wall & BL-lite \\
  M936 & First Wall & BL-lite \\
  M940 & Copy M100 & Diagnostics MCNP Model \\
  M941 & Copy M100 & Diagnostics MCNP Model \\
  M942 & Copy M100 & Diagnostics MCNP Model \\
  M943 & Copy M100 & Diagnostics MCNP Model \\
  M944 & EPP Drawers & Diagnostics MCNP Model \\
  M945 & EPP Contents & Diagnostics MCNP Model \\
  M946 & EPP Chunks & Diagnostics MCNP Model \\
 \caption{Table showing the origin of the material descriptions used}
 \label{tab:material_origin}
 \end{longtable}
\end{centering}

\subsubsection*{Nuclear Data}
Neutron Data - The neutron transport data used was \gls{fendl}-2.1/MC, as 
recommended by \gls{io} for ITER analysis. 
\\
\\
Activation Data - The activation data used was \gls{fendl}-3.0/A.
\\
\\
Photon Data - The standard photon data supplied with \gls{mcnp}5 mcplib04 was 
used. This data ultimately falls back onto \gls{endf}/B-VI release 8 photo
atomic data.
\subsection{Transport Software}
\gls{mcnp}
\subsection{Activation Software}
\gls{alara_c}
\newpage
\clearpage
\section{Neutron Transport Results and Conclusions}
\subsection{Baseline}
The baseline results represent a standard C-lite model since there is no B$_4$C
liner present, however relative to the standard C-lite native \gls{mcnp} model
there is much more equipment present in the \gls{upi} \& \gls{epi}, and
the divertor pumping port is unplugged. The open pumping port specifically
leads to higher neutron fluxes beyond the vacuum vessel.
\begin{figure}[ht!]
  \centering
  \includegraphics[trim={0cm 9cm 0cm 10cm},clip,scale=0.75]{../plots/final_model_nob4c/{Neutron_Result_(All_Energy_Groups)_Meshes_1-7}.png}     
  \label{fig:neutrons_baseline}
  \caption{The total neutron flux from the baseline case}
\end{figure}
\begin{figure}[ht!]
  \centering
  \includegraphics[trim={0cm 9cm 0cm 10cm},clip,scale=0.75]{../plots/final_model_nob4c/Neutron_Result_Relative_Error_Meshes_1-7.png}     
  \label{fig:neutrons_baseline_relerr}
  \caption{The relative error in the total neutron flux from the baseline case}
\end{figure}
\newpage
\clearpage
\subsection{Including the B4C Liner}
These results represent a standard C-lite model with improved B$_4$C liner,
however relative to the standard C-lite native \gls{mcnp} model there is much more
equipment present in the \gls{pi} (both upper and lower), and the divertor
pumping port is unplugged. 
\begin{figure}[ht!]
  \centering
  \includegraphics[trim={0cm 9cm 0cm 10cm},clip,scale=0.75]{../plots/final_model/{Neutron_Result_(All_Energy_Groups)_Meshes_1-7}.png}     
  \label{fig:neutrons_no4bc}
  \caption{The total neutron flux from the B$_4$C case}
\end{figure}
\begin{figure}[ht!]
  \centering
  \includegraphics[trim={0cm 9cm 0cm 10cm},clip,scale=0.75]{../plots/final_model/Neutron_Result_Relative_Error_Meshes_1-7.png}     
  \label{fig:neutrons_no4bc_relerr}
  \caption{The relative error in the total neutron flux from the B$_4$C case}
\end{figure}
\subsection{Conclusion}
The final calculations represent only around 10\% of the overall requested 
runtime; this is mostly due to calculations not completing with the 3 day 
allotted runtime on CHTC. When variance reduction parameters that span over
 many orders of magnitude are being used in non-analog Monte Carlo simulations,
 it is difficult to estimate the computational time needed to simulate each 
particle because some of these particles undergo millions of splitting events. 
To overcome this problem, the authors suggest 1) using the computer time cutoff
 (CTME) card in MCNP to stop the MCNP runs after reaching a certain amount of 
running time without losing the results, 2) to increase the frequency of the 
printing and dumping of MCNP using the PRDMP card, 3) and to avoid the use of 
the MPI version of MCNP. The Monte Carlo calculations can still run in parallel
 without using MPI by submitting independent jobs with different random number 
seeds and statistically averaging the results as demonstrated in this report. 
The load balancing of the MPI  version of MCNP does not consider the huge 
differences in the computational times needed to simulate different histories
 that are typical with the use of variance reduction parameters that span many 
orders of magnitude. The MCNP runs will not be stopped even if the computational
 time exceeded the time of the  CTME card until MCNP reaches the printing and 
dumping stage. Without increasing the frequency of these dumping events, the 
time of the CTME card may be ignored for a very long time.

\newpage
\clearpage

\section{Neutron Activation Results and Conclusions}
The \gls{r2s} shutdown dose rate calculations were performed using the
\texttt{pyne.r2s} methods found \gls{pyne}. The setup scripts produce
\gls{alara_c} inputs and the neutron activation calculations were performed
using \gls{alara_c} 2.9.1RC. 
\subsection{Baseline}
\subsubsection{Decay Time 1 - 1$\times$10$^{5}$ seconds}
\begin{figure}[ht!]
\centering
\includegraphics[trim={0cm 9cm 0cm 10cm},clip,scale=0.75]{../plots/final_model_nob4c/{Photon_Source_Density_Decay_Time_1_All_Energy_Groups}.png}
\label{fig:source_dc1_no4bc}
\caption{The total shutdown photon source for the baseline case for decay time 1}
\end{figure}
\clearpage
\subsubsection{Decay Time 2 - 1$\times$10$^{6}$ seconds}
\begin{figure}[ht!]
\centering
\includegraphics[trim={0cm 9cm 0cm 10cm},clip,scale=0.75]{../plots/final_model_nob4c/{Photon_Source_Density_Decay_Time_2_All_Energy_Groups}.png}
\label{fig:source_dc2_no4bc}
\caption{The total shutdown photon source for the baseline case for decay time 2}
\end{figure}
\clearpage
\subsubsection{Decay Time 3 - 1$\times$10$^{7}$ seconds}
\begin{figure}[ht!]
\centering
\includegraphics[trim={0cm 9cm 0cm 10cm},clip,scale=0.75]{../plots/final_model_nob4c/{Photon_Source_Density_Decay_Time_3_All_Energy_Groups}.png}
\label{fig:source_dc3_no4bc}
\caption{The total shutdown photon source for the baseline case for decay time 3}
\end{figure}
\newpage
\clearpage
\subsection{Including the B4C Liner}
\subsubsection{Decay Time 1 - 1$\times$10$^{5}$ seconds}
\begin{figure}[ht!]
\centering
\includegraphics[trim={0cm 9cm 0cm 10cm},clip,scale=0.75]{../plots/final_model/{Photon_Source_Density_Decay_Time_1_All_Energy_Groups}.png}
\label{fig:source_dc1_b4c}
\caption{The total shutdown photon source for the case with B$_4$C for decay time 1}
\end{figure}
\clearpage
\subsubsection{Decay Time 2 - 1$\times$10$^{6}$ seconds}
\begin{figure}[ht!]
\centering
\includegraphics[trim={0cm 9cm 0cm 10cm},clip,scale=0.75]{../plots/final_model/{Photon_Source_Density_Decay_Time_2_All_Energy_Groups}.png}
\label{fig:source_dc2_b4c}
\caption{The total shutdown photon source for the case with B$_4$C for decay time 2}
\end{figure}
\clearpage
\subsubsection{Decay Time 3 - 1$\times$10$^{7}$ seconds}
\begin{figure}[ht!]
\centering
\includegraphics[trim={0cm 9cm 0cm 10cm},clip,scale=0.75]{../plots/final_model/{Photon_Source_Density_Decay_Time_3_All_Energy_Groups}.png}
\label{fig:source_dc3_b4c}
\caption{The total shutdown photon source for the case with B$_4$C for decay time 3}
\end{figure}
\clearpage
\subsection{Comparison}
\subsection{Conclusion}
The shutdown photon source produced for each case and 3 decay times results in
42 individual photon sources that must be used in subsequent photon transport
calculations.
\\
\\
It is worth noting that the lack of gamma source present in the \gls{epp}
regions are not due to poor neutron transport in this region, but due to the
fact that the \gls{gepp} model has a large B$_4$C powder filled containers
in these regions, B$_4$C when irradiated by neutrons undergoes reactions
which produce only low energy beta particle decays with very low associated
photon energies, thus effectively produces no decay photons.
\newpage
\clearpage
\section{Shutdown Photon Dose rates Results and Conclusions}
\subsection{Baseline}
\subsubsection{Decay time 1 - 1.$\times$10$^5$ s}
\begin{figure}[ht!]
\centering
\includegraphics[trim={0cm 9cm 0cm 10cm},clip,scale=0.75]{../plots/final_model_nob4c/Photon_Dose_Rate_Decay_Time_1_Meshes_1-7.png}
\label{fig:photons_dc1_no4bc_total}
\caption{The total dose rate summed over all meshes for decay time 1}
\end{figure}
\begin{figure}[ht!]
\centering
\includegraphics[trim={0cm 9cm 0cm 10cm},clip,scale=0.75]{../plots/final_model_nob4c/Photon_Dose_Relative_Error_Decay_Time_1_Meshes_1-7.png}
\label{fig:photons_dc1_no4bc_total_error}
\caption{The error in the total dose rate combined over all meshes for decay time 1}
\end{figure}
\begin{figure}[ht!]
\centering
\includegraphics[trim={0cm 9cm 0cm 10cm},clip,scale=0.75]{../plots/final_model_nob4c/Photon_Dose_Rate_Decay_Time_1_Meshes_1-7_Zoomed_Y_Views.png}
\label{fig:photons_dc1_no4bc_total_zoomed}
\caption{The total dose rate summed over all meshes for decay time 1, with focus on the port areas}
\end{figure}
\begin{figure}[ht!]
\centering
\includegraphics[trim={0cm 9cm 0cm 10cm},clip,scale=0.75]{../plots/final_model_nob4c/Photon_Dose_Relative_Error_Decay_Time_1_Meshes_1-7_Zoomed_Y_Views.png}
\label{fig:photons_dc1_no4bc_total_error_zoomed}
\caption{The error in the total dose rate combined over all meshes for decay time 1, with focus on the port areas}
\end{figure}

\clearpage
\subsubsection{Decay time 2 - 1.$\times$10$^6$ s}
%% the totals plots
\begin{figure}[ht!]
\centering
\includegraphics[trim={0cm 9cm 0cm 10cm},clip,scale=0.75]{../plots/final_model_nob4c/Photon_Dose_Rate_Decay_Time_2_Meshes_1-7.png}
\label{fig:photons_dc2_no4bc_total}
\caption{The total dose rate summed over all meshes for decay time 2}
\end{figure}
\begin{figure}[ht!]
\centering
\includegraphics[trim={0cm 9cm 0cm 10cm},clip,scale=0.75]{../plots/final_model_nob4c/Photon_Dose_Relative_Error_Decay_Time_2_Meshes_1-7.png}
\label{fig:photons_dc2_no4bc_total_error}
\caption{The error in the total dose rate combined over all meshes for decay time 2}
\end{figure}
\begin{figure}[ht!]
\centering
\includegraphics[trim={0cm 9cm 0cm 10cm},clip,scale=0.75]{../plots/final_model_nob4c/Photon_Dose_Rate_Decay_Time_2_Meshes_1-7_Zoomed_Y_Views.png}
\label{fig:photons_dc2_no4bc_total_zoomed}
\caption{The total dose rate summed over all meshes for decay time 2, with focus on the port areas}
\end{figure}
\begin{figure}[ht!]
\centering
\includegraphics[trim={0cm 9cm 0cm 10cm},clip,scale=0.75]{../plots/final_model_nob4c/Photon_Dose_Relative_Error_Decay_Time_2_Meshes_1-7_Zoomed_Y_Views.png}
\label{fig:photons_dc2_no4bc_total_error_zoomed}
\caption{The error in the total dose rate combined over all meshes for decay time 2, with focus on the port areas}
\end{figure}
\clearpage

\subsubsection{Decay time 3 - 1.$\times$10$^7$ s}
%% the totals plots
\begin{figure}[ht!]
\centering
\includegraphics[trim={0cm 9cm 0cm 10cm},clip,scale=0.75]{../plots/final_model_nob4c/Photon_Dose_Rate_Decay_Time_3_Meshes_1-7.png}
\label{fig:photons_dc3_no4bc_total}
\caption{The total dose rate summed over all meshes for decay time 3}
\end{figure}
\begin{figure}[ht!]
\centering
\includegraphics[trim={0cm 9cm 0cm 10cm},clip,scale=0.75]{../plots/final_model_nob4c/Photon_Dose_Relative_Error_Decay_Time_3_Meshes_1-7.png}
\label{fig:photons_dc3_no4bc_total_error}
\caption{The error in the total dose rate combined over all meshes for decay time 3}
\end{figure}
\begin{figure}[ht!]
\centering
\includegraphics[trim={0cm 9cm 0cm 10cm},clip,scale=0.75]{../plots/final_model_nob4c/Photon_Dose_Rate_Decay_Time_3_Meshes_1-7_Zoomed_Y_Views.png}
\label{fig:photons_dc3_no4bc_total_zoomed}
\caption{The total dose rate summed over all meshes for decay time 3, with focus on the port areas}
\end{figure}
\begin{figure}[ht!]
\centering
\includegraphics[trim={0cm 9cm 0cm 10cm},clip,scale=0.75]{../plots/final_model_nob4c/Photon_Dose_Relative_Error_Decay_Time_3_Meshes_1-7_Zoomed_Y_Views.png}
\label{fig:photons_dc3_no4bc_total_error_zoomed}
\caption{The error in the total dose rate combined over all meshes for decay time 3, with focus on the port areas}
\end{figure}
\clearpage
\subsection{Including the B$_4$C liner}
\subsubsection{Decay time 1 - 1.$\times$10$^5$ s}
\begin{figure}[ht!]
\centering
\includegraphics[trim={0cm 9cm 0cm 10cm},clip,scale=0.75]{../plots/final_model/Photon_Dose_Rate_Decay_Time_1_Meshes_1-7.png}
\label{fig:photons_dc1_b4c_total}
\caption{The total dose rate summed over all meshes for decay time 1}
\end{figure}
\begin{figure}[ht!]
\centering
\includegraphics[trim={0cm 9cm 0cm 10cm},clip,scale=0.75]{../plots/final_model/Photon_Dose_Relative_Error_Decay_Time_1_Meshes_1-7.png}
\label{fig:photons_dc1_b4c_total_error}
\caption{The error in the total dose rate combined over all meshes for decay time 1}
\end{figure}
\begin{figure}[ht!]
\centering
\includegraphics[trim={0cm 9cm 0cm 10cm},clip,scale=0.75]{../plots/final_model/Photon_Dose_Rate_Decay_Time_1_Meshes_1-7_Zoomed_Y_Views.png}
\label{fig:photons_dc1_b4c_total_zoomed}
\caption{The total dose rate summed over all meshes for decay time 1, with focus on the port areas}
\end{figure}
\begin{figure}[ht!]
\centering
\includegraphics[trim={0cm 9cm 0cm 10cm},clip,scale=0.75]{../plots/final_model/Photon_Dose_Relative_Error_Decay_Time_1_Meshes_1-7_Zoomed_Y_Views.png}
\label{fig:photons_dc1_b4c_total_error_zoomed}
\caption{The error in the total dose rate combined over all meshes for decay time 1, with focus on the port areas}
\end{figure}

\clearpage
\subsubsection{Decay time 2 - 1.$\times$10$^6$ s}

%% the totals plots
\begin{figure}[ht!]
\centering
\includegraphics[trim={0cm 9cm 0cm 10cm},clip,scale=0.75]{../plots/final_model/Photon_Dose_Rate_Decay_Time_2_Meshes_1-7.png}
\label{fig:photons_dc2_b4c_total}
\caption{The total dose rate summed over all meshes for decay time 2}
\end{figure}
\begin{figure}[ht!]
\centering
\includegraphics[trim={0cm 9cm 0cm 10cm},clip,scale=0.75]{../plots/final_model/Photon_Dose_Relative_Error_Decay_Time_2_Meshes_1-7.png}
\label{fig:photons_dc2_b4c_total_error}
\caption{The error in the total dose rate combined over all meshes for decay time 2}
\end{figure}
\begin{figure}[ht!]
\centering
\includegraphics[trim={0cm 9cm 0cm 10cm},clip,scale=0.75]{../plots/final_model/Photon_Dose_Rate_Decay_Time_2_Meshes_1-7_Zoomed_Y_Views.png}
\label{fig:photons_dc2_b4c_total_zoomed}
\caption{The total dose rate summed over all meshes for decay time 2, with focus on the port areas}
\end{figure}
\begin{figure}[ht!]
\centering
\includegraphics[trim={0cm 9cm 0cm 10cm},clip,scale=0.75]{../plots/final_model/Photon_Dose_Relative_Error_Decay_Time_2_Meshes_1-7_Zoomed_Y_Views.png}
\label{fig:photons_dc2_b4c_total_error_zoomed}
\caption{The error in the total dose rate combined over all meshes for decay time 2, with focus on the port areas}
\end{figure}
\clearpage

\subsubsection{Decay time 3 - 1.$\times$10$^7$ s}
%% the totals plots
\begin{figure}[ht!]
\centering
\includegraphics[trim={0cm 9cm 0cm 10cm},clip,scale=0.75]{../plots/final_model/Photon_Dose_Rate_Decay_Time_3_Meshes_1-7.png}
\label{fig:photons_dc3_b4c_total}
\caption{The total dose rate summed over all meshes for decay time 3}
\end{figure}
\begin{figure}[ht!]
\centering
\includegraphics[trim={0cm 9cm 0cm 10cm},clip,scale=0.75]{../plots/final_model/Photon_Dose_Relative_Error_Decay_Time_3_Meshes_1-7.png}
\label{fig:photons_dc3_b4c_total_error}
\caption{The error in the total dose rate combined over all meshes for decay time 3}
\end{figure}
\begin{figure}[ht!]
\centering
\includegraphics[trim={0cm 9cm 0cm 10cm},clip,scale=0.75]{../plots/final_model/Photon_Dose_Rate_Decay_Time_3_Meshes_1-7_Zoomed_Y_Views.png}
\label{fig:photons_dc3_b4c_total_zoomed}
\caption{The total dose rate summed over all meshes for decay time 3, with focus on the port areas}
\end{figure}
\begin{figure}[ht!]
\centering
\includegraphics[trim={0cm 9cm 0cm 10cm},clip,scale=0.75]{../plots/final_model/Photon_Dose_Relative_Error_Decay_Time_3_Meshes_1-7_Zoomed_Y_Views.png}
\label{fig:photons_dc3_b4c_total_error_zoomed}
\caption{The error in the total dose rate combined over all meshes for decay time 3, with focus on the port areas}
\end{figure}
\clearpage
\subsection{Photon Cross Talk}
The impact of decay photons born in areas far from a region of interest and 
streaming to it are known as cross talk. Cross talk makes assigning dose 
budgets to components difficult as there is the possibility that a given 
component will meet its dose budget, but due to leaky adjacent components
may mean that the dose is higher than would otherwise be tolerated. There
have been studies previously done to estimate the degree of cross talk for
a number of ITER \gls{pp} systems, but those calculations were for an older
model with fewer components present in the interspaces. Below are the 
detailed contributions for each mesh shown in Figures 
\ref{fig:ct_photons_dc2_no4bc_m7_flux} to \ref{fig:ct_photons_dc2_no4bc_m7_flux}.

\begin{figure}[ht!]
\centering
\includegraphics[trim={0cm 9cm 0cm 10cm},clip,scale=0.75]{../plots/final_model_nob4c/Photon_Dose_Rate_Decay_Time_2_Mesh_1.png}
\label{fig:ct_photons_dc2_no4bc_m1_flux}
\caption{The total photon dose from the B$_4$C case for decay time 2 for mesh 1}
\end{figure}

\begin{figure}[ht!]
\centering
\includegraphics[trim={0cm 9cm 0cm 10cm},clip,scale=0.75]{../plots/final_model_nob4c/Photon_Dose_Rate_Decay_Time_2_Mesh_2.png}
\label{fig:ct_photons_dc2_no4bc_m2_flux}
\caption{The total photon dose from the B$_4$C case for decay time 2 for mesh 2}
\end{figure}

\begin{figure}[ht!]
\centering
\includegraphics[trim={0cm 9cm 0cm 10cm},clip,scale=0.75]{../plots/final_model_nob4c/Photon_Dose_Rate_Decay_Time_2_Mesh_3.png}
\label{fig:ct_photons_dc2_no4bc_m3_flux}
\caption{The total photon dose from the B$_4$C case for decay time 2 for mesh 3}
\end{figure}

\begin{figure}[ht!]
\centering
\includegraphics[trim={0cm 9cm 0cm 10cm},clip,scale=0.75]{../plots/final_model_nob4c/Photon_Dose_Rate_Decay_Time_2_Mesh_4.png}
\label{fig:ct_photons_dc2_no4bc_m4_flux}
\caption{The total photon dose from the B$_4$C case for decay time 2 for mesh 4}
\end{figure}

\begin{figure}[ht!]
\centering
\includegraphics[trim={0cm 9cm 0cm 10cm},clip,scale=0.75]{../plots/final_model_nob4c/Photon_Dose_Rate_Decay_Time_2_Mesh_5.png}
\label{fig:ct_photons_dc2_no4bc_m5_flux}
\caption{The total photon dose from the B$_4$C case for decay time 2 for mesh 5}
\end{figure}

\begin{figure}[ht!]
\centering
\includegraphics[trim={0cm 9cm 0cm 10cm},clip,scale=0.75]{../plots/final_model_nob4c/Photon_Dose_Rate_Decay_Time_2_Mesh_6.png}
\label{fig:ct_photons_dc2_no4bc_m6_flux}
\caption{The total photon dose from the B$_4$C case for decay time 2 for mesh 6}
\end{figure}

\begin{figure}[ht!]
\centering
\includegraphics[trim={0cm 9cm 0cm 10cm},clip,scale=0.75]{../plots/final_model_nob4c/Photon_Dose_Rate_Decay_Time_2_Mesh_7.png}
\label{fig:ct_photons_dc2_no4bc_m7_flux}
\caption{The total photon dose from the B$_4$C case for decay time 2 for mesh 7}
\end{figure}
\newpage
\clearpage
\subsection{Conclusion}
It is clear from Figures \ref{fig:photons_dc1_no4bc_total} to 
\ref{fig:photons_dc3_b4c_total_error_zoomed} that the shutdown photon dose rate 
in the port interspace is slightly affected by the addition of boron carbide 
to the plasma side surface of the bioshield. The reason for the this is 
significantly degraded thermal flux which impacts the importance of (n,$\gamma$)
 reactions and also the transport of neutrons back from the bioshield. The true 
benefit appears only to be very close to the B$_4$C layer, where is reduces the
activation of steel components near to the bioshield, those components away
from the bioshield appear to have their activation dominated by the neutron
transported via long paths around other ports. This is  likely due to the fact
that most neutron induced activation in fusion devices are due to high energy
capture reactions like (n,p), (n,2n) and (n,xn) reactions.
\\
\\
What is clear is that the shutdown photon dose rate at the second decay time is
significantly higher than the desired target goal of 100 $\mu$Sv hr$^{-1}$,
previous analysis has shown ranges of \gls{sdr} in the \gls{epi} of anywhere
from 200 to 400 $\mu$Sv hr$^{-1}$. It is surmised that the reason for the
significantly higher \gls{sdr} in this calculation is due to the increased
mass of components modelled in the \gls{epi} which leads to higher \gls{sdr}.
The \gls{pfc}3 \& 4 coils also are strong sources near the \gls{epi}, but being
beyond the \gls{ve} are somewhat shielded by it. The \gls{bs} is als a
significant source, unshielded due to is proximity to the \gls{epi}. The ever
present photon cross talk is also present, the \gls{lp} being unplugged leads
to strong sources in the cryopump, which has roughly the same contribution
as the sources within the \gls{epi}.
\\
\\
It is also apparent that the utility of the B$_4$C present in the \gls{gepp}
design is not as high as may be imagined. The B$_4$C powder is contained within
a stainless steel lined box, which itself becomes activated under the presence
of neutrons. The largest components in the region which do become activated are
the back plate and attachment flange, both of which are the dominant source of
activation in the region. It is suggested that the front 50 cm of the B$_4$C
drawers could be removed, reducing the weight of the port plug without
significantly impacting the shielding capability.
\newpage
\clearpage
\section{Calculation Details}
\subsection{Neutron Transport}
Using the methodology shown in Section \ref{section:method}, the neutron meshes
were split into 7 calculations. These jobs were submitted to the \gls{uw}
\gls{chtc} system in batches of 200 for each mesh, with seven meshes used to
determine the neutron flux, making a
total of 2800 individual \gls{mcnp} simulations. These results were then
combined using \gls{pyne} to result in the statistically averaged final results.
Initially calculations were batched into groups of $5\times10^7$ particles per
batch, however these calculations did not finish in a timely enough manner and
were instead reduced to groups of $5\times10^6$ particles per batch. With the
smaller batching system calculations did finish, however, only 10\% of all
calculations for each mesh finished, but this represents $5\times10^7$ total
source histories for each mesh.
\\
\\
\subsubsection{Baseline}
There are no nominal differences between the Baseline and the B$_4$C calculation
with the excpetion
\subsubsection{With B$_4$C}
\subsection{Activation Calculations}
There was an activation calculation for each neutron mesh and case, therefore
there were a total of 14 activation calculations performed. The activation
calculations used a uniform `truncation tolerance', a parameter in \gls{alara_c}
which details when the linearised transmutation chains should be terminated. The
materials used in the activation calculation were taken directly from the 
associated MCNP input decks.
\subsubsection{Baseline}
\subsubsection{With B$_4$C}
\subsection{Photon Transport}
The shutdown photon sources were developed in several separate meshes and
therefore any individual mesh does not represent the entire problem. The
results of the previous \gls{alara_c} calculations are then processed further
to make shutdown sources to be read into a sampling subroutine (also distributed
with \gls{pyne}). Each photon source is run independently, and the photon dose
is recorded on a mesh common to each photon calculation and then the meshes
summed together to create the final result. The final mesh has a uniform size of
side 2 cm striding from x = {0,1500}, y = {-500,500}, z = {-1900,1500} cm.
The meshes used the\gls{icrp}-74 dose response coefficients as a dose multiplier
as recommended in \cite{iter_sdr_coeffs}.
\\
\\
Each photon transport calculation was performed for a fixed number of particles,
1$\times$10$^8$, which nominally completed in less than 15 hours on 40 cores of
\gls{aci}. Each photon simulation used the same random number seed, but given
that each photon source is unique, there are no concerns regarding the resuse
of the same seed.
\subsubsection{Baseline}
\subsubsection{With B$_4$C}
\newpage
\clearpage
\section{Acceptance Criteria}
\newpage
\clearpage
\section{Lessons learned and future work}
\subsection*{CAD}
One of the biggest reasons for delay with the deliver of the calculation results
in this task was the cleaning and preparation of the CAD geometry. Several
months were expended cleaning, repairing and merging components together into a
final model. Ultimately due to time contraints, the decision was taken to
proceed with the analysis with the CAD in its current state. It is important
for CAD based workflows to have access to the unmodified, unsimplified CAD
models that have been used to prepare the IO geometries. A specific issue
was the use of the IO supplied Clite CAD geometry which was unprepared for
such an analysis, several stages of simplification and fixing has been performed
which lead to problems downstream in the final CAD preparation.
\subsection*{High Throughput Methodology}
The splitting of a given MCNP calculation into subruns spanning the same random
number space proved to be a successful way of performing these calculations with
access to the large of amounts of CPU-time on \gls{uw}'s \gls{chtc} system. In
the future, we expect to make increasing use of this method for MCNP
calculations. In this specific case long histories were a pervasive problem and
limited the total number of runs that could be completed, however the underlying
method was reliable. One of the issues associated with such a method is the
quantity of storage space required to store the intermediate split results prior
to merging, in this batch of simulations a total sum of 1.8 Tb of data was
produced, consider that this represents only 10\% of the completed runs,
therefore this simulation should have consumed some 20 Tb of space.
\subsection*{Variance Reduction}
The neutron section of this work was impacted by long histories, which appear to
be a pervasive problem in the ITER geometry due to the area available for
neutrons to stream through, combined with the strong neutron attenuation though
bulk shielding. Ultimately this problem needs a solution, there are several
potential solutions but each must be appropriately weighed and considered to
ensure that a fair game is being played when rouletting particles. A future
work could be to investigate several potential solutions to the ``long history
problem'' and then provide recommendations to IO.
\\
\\
The authors will also like to note that the use of use of global MC 
variance-reduction techniques, including the FW-CADIS method does not 
provide optimum variance reduction parameters for SDR calculations. 
The goal of these global MC variance-reduction techniques is to uniformly 
distribute the Monte Carlo computational efforts throughout numerous 
phase-space segments to calculate many Monte Carlo tallies with nearly 
uniform relative uncertainties. Even though SDR analysis requires the 
calculation of space- and energy-dependent neutron fluxes, the goal of 
the analysis is the accurate assessment of the final \gls{sdr}. Global 
Monte Carlo variance-reduction techniques do not focus MC computational 
efforts on calculating the production rates of radioisotopes that will 
ultimately contribute to the \gls{sdr}. On the contrary, the Multi-Step 
Consistent Adjoint Driven Importance Sampling (MS-CADIS) hybrid 
MC/deterministic method is currently under development to speed up the \gls{sdr}
 Monte Carlo neutron transport calculation using an importance function that 
represents the neutron importance to the final \gls{sdr} \cite{mscadis}. 
The MS-CADIS method uses the CADIS method to develop consistent source biasing 
and weight window variance-reduction parameters. However, because the MS-CADIS 
method focuses on multistep shielding calculations such as the \gls{r2s} 
calculations of \gls{sdr}, it develops an importance function for the initial 
radiation transport calculation (e.g., the neutron calculations in \gls{sdr} 
simulations) that represents the importance of particles to the final response
 of the overall simulation. This paper explains the theory of the MS-CADIS 
method and provides some insights into the physical interpretations of the 
MS-CADIS adjoint neutron source and the MS-CADIS adjoint neutron flux.
\section{Acknowledgements}
\newpage
\clearpage
\bibliographystyle{unsrt}
\bibliography{bibliography}
\newpage
\clearpage
\section{Appendix - A - Materials used in the problem}
\begin{centering}
\begin{longtable}[ht!]
{ p{0.3\textwidth} | p{0.3\textwidth} }
\hline
Nuclide & Atom Fraction\\
\hline
H1 & 6.997078e-04\\
H2 & 1.608276e-07\\
B10 & 1.831598e-06\\
B11 & 8.106036e-06\\
C12 & 2.946747e-04\\
N14 & 6.928888e-04\\
N15 & 2.769410e-06\\
O16 & 5.539577e-03\\
Si28 & 4.564952e-03\\
Si29 & 2.400762e-04\\
Si30 & 1.637057e-04\\
P31 & 2.484376e-04\\
S & 9.937764e-05\\
Ti46 & 7.870667e-05\\
Ti47 & 7.252245e-05\\
Ti48 & 7.335356e-04\\
Ti49 & 5.497666e-05\\
Ti50 & 5.371156e-05\\
Cr50 & 7.258372e-03\\
Cr52 & 1.455608e-01\\
Cr53 & 1.682310e-02\\
Cr54 & 4.266560e-03\\
Mn55 & 1.788766e-02\\
Fe54 & 3.637981e-02\\
Fe56 & 5.922114e-01\\
Fe57 & 1.392136e-02\\
Fe58 & 1.885133e-03\\
Co59 & 4.968802e-04\\
Ni58 & 8.180318e-02\\
Ni60 & 3.259580e-02\\
Ni61 & 1.440563e-03\\
Ni62 & 4.668329e-03\\
Ni64 & 1.227273e-03\\
Cu63 & 2.042160e-03\\
Cu65 & 9.391214e-04\\
Nb93 & 9.937617e-05\\
Mo92 & 3.532189e-03\\
Mo94 & 2.249541e-03\\
Mo95 & 3.912899e-03\\
Mo96 & 4.142835e-03\\
Mo97 & 2.396706e-03\\
Mo98 & 6.118223e-03\\
Mo100 & 2.491646e-03\\
Ta181 & 9.937573e-05\\
\caption{Table showing the isotopic description of material EPP3L}
\label{table:material_EPP3L}
\end{longtable}
\clearpage
\begin{longtable}[ht!]
{ p{0.3\textwidth} | p{0.3\textwidth} }
\hline
Nuclide & Mass Fraction\\
\hline
He4 & 3.355146e-03\\
B10 & 1.802792e-04\\
B11 & 7.978553e-04\\
C12 & 1.970612e-04\\
N14 & 9.160305e-04\\
N15 & 3.624343e-06\\
O16 & 1.378506e-02\\
Al27 & 6.199031e-04\\
Si28 & 1.295973e-02\\
Si29 & 6.826361e-04\\
Si30 & 4.652330e-04\\
P31 & 2.955888e-04\\
S & 1.970642e-04\\
K & 7.614452e-05\\
Ti46 & 1.096136e-04\\
Ti47 & 1.010008e-04\\
Ti48 & 1.022010e-03\\
Ti49 & 7.656520e-05\\
Ti50 & 7.480316e-05\\
V & 2.627356e-05\\
Cr50 & 4.743983e-03\\
Cr52 & 9.513602e-02\\
Cr53 & 1.099539e-02\\
Cr54 & 2.788589e-03\\
Mn55 & 1.313740e-02\\
Fe54 & 2.407534e-02\\
Fe56 & 3.919107e-01\\
Fe57 & 9.212806e-03\\
Fe58 & 1.247545e-03\\
Co59 & 3.284351e-04\\
Ni58 & 5.296795e-02\\
Ni60 & 2.110594e-02\\
Ni61 & 9.327718e-04\\
Ni62 & 3.022769e-03\\
Ni64 & 7.946654e-04\\
Cu63 & 1.898430e-01\\
Cu65 & 8.730242e-02\\
Zr & 1.313594e-05\\
Nb93 & 2.401275e-02\\
Mo92 & 2.334761e-03\\
Mo94 & 1.486929e-03\\
Mo95 & 2.586387e-03\\
Mo96 & 2.738387e-03\\
Mo97 & 1.584213e-03\\
Mo98 & 4.044120e-03\\
Mo100 & 1.646961e-03\\
Sn & 7.995497e-03\\
Ta181 & 6.052439e-03\\
W182 & 1.724655e-06\\
W183 & 9.367748e-07\\
W184 & 2.016320e-06\\
W186 & 1.890873e-06\\
Pb206 & 1.276447e-06\\
Pb207 & 1.176209e-06\\
Pb208 & 2.802325e-06\\
Bi209 & 5.254959e-06\\
\caption{Table showing the isotopic description of material M908}
\label{table:material_M908}
\end{longtable}\clearpage

\begin{longtable}[ht!]
{ p{0.3\textwidth} | p{0.3\textwidth} }
\hline
Nuclide & Mass Fraction\\
\hline
\\
B10 & 3.318107e-06\\
B11 & 1.468481e-05\\
C12 & 2.965728e-04\\
C13 & 3.475851e-06\\
N14 & 1.095890e-03\\
N15 & 4.288673e-06\\
Si28 & 9.188137e-03\\
Si29 & 4.834412e-04\\
Si30 & 3.300419e-04\\
P31 & 3.000486e-04\\
S32 & 1.420960e-04\\
S33 & 1.156998e-06\\
S34 & 6.754457e-06\\
S36 & 1.682823e-08\\
Ti46 & 7.921378e-05\\
Ti47 & 7.298965e-05\\
Ti48 & 7.385702e-04\\
Ti49 & 5.533087e-05\\
Ti50 & 5.405755e-05\\
Cr50 & 8.348725e-03\\
Cr52 & 1.674258e-01\\
Cr53 & 1.935031e-02\\
Cr54 & 4.907523e-03\\
Mn55 & 2.000324e-02\\
Fe54 & 3.613009e-02\\
Fe56 & 5.881456e-01\\
Fe57 & 1.382579e-02\\
Fe58 & 1.872207e-03\\
Co59 & 1.000162e-03\\
Ni58 & 8.065044e-02\\
Ni60 & 3.213624e-02\\
Ni61 & 1.420261e-03\\
Ni62 & 4.602659e-03\\
Ni64 & 1.209846e-03\\
Nb93 & 1.000162e-03\\
Mo92 & 6.959281e-04\\
Mo94 & 4.477765e-04\\
Mo95 & 7.834282e-04\\
Mo96 & 8.331563e-04\\
Mo97 & 4.848117e-04\\
Mo98 & 1.244427e-03\\
Mo100 & 5.112820e-04\\
Ta180 & 1.194554e-08\\
Ta181 & 1.000043e-04\\
\caption{Table showing the isotopic description of material CryoPipes}
\label{table:material_CryoPipes}
\end{longtable}\clearpage

\begin{longtable}[ht!]
{ p{0.3\textwidth} | p{0.3\textwidth} }
\hline
Nuclide & Mass Fraction\\
\hline
H1 & 5.004784e-06\\
H2 & 1.150349e-09\\
C12 & 2.968774e-05\\
N14 & 9.972407e-06\\
N15 & 3.985867e-08\\
O16 & 2.995422e-05\\
Na23 & 1.001186e-05\\
Mg & 5.005930e-06\\
Al27 & 1.501778e-05\\
Si28 & 1.839627e-05\\
Si29 & 9.674816e-07\\
Si30 & 6.597175e-07\\
P31 & 5.005883e-05\\
S & 5.006025e-06\\
K & 1.001096e-05\\
Ca & 1.001235e-05\\
Ti46 & 7.929504e-07\\
Ti47 & 7.306453e-07\\
Ti48 & 7.390152e-06\\
Ti49 & 5.538772e-07\\
Ti50 & 5.411284e-07\\
Cr50 & 4.178642e-07\\
Cr52 & 8.379937e-06\\
Cr53 & 9.685066e-07\\
Cr54 & 2.456255e-07\\
Mn55 & 5.005959e-06\\
Fe54 & 1.695680e-06\\
Fe56 & 2.760323e-05\\
Fe57 & 6.488791e-07\\
Fe58 & 8.786698e-08\\
Co59 & 1.001190e-05\\
Ni58 & 1.345544e-05\\
Ni60 & 5.361537e-06\\
Ni61 & 2.369519e-07\\
Ni62 & 7.678737e-07\\
Ni64 & 2.018687e-07\\
Cu63 & 6.858056e-06\\
Cu65 & 3.153799e-06\\
Zr & 1.001074e-05\\
Nb93 & 1.001189e-05\\
Mo92 & 1.423431e-05\\
Mo94 & 9.065397e-06\\
Mo95 & 1.576854e-05\\
Mo96 & 1.669515e-05\\
Mo97 & 9.658474e-06\\
Mo98 & 2.465584e-05\\
Mo100 & 1.004101e-05\\
Ta181 & 1.001185e-05\\
W182 & 2.624604e-01\\
W183 & 1.425122e-01\\
W184 & 3.068133e-01\\
W186 & 2.877791e-01\\
Pb206 & 2.398459e-06\\
Pb207 & 2.210035e-06\\
Pb208 & 5.265464e-06\\
\caption{Table showing the isotopic description of material M74}
\label{table:material_M74}
\end{longtable}\clearpage

\begin{longtable}[ht!]
{ p{0.3\textwidth} | p{0.3\textwidth} }
\hline
Nuclide & Mass Fraction\\
\hline
\\
B10 & 1.843098e-06\\
B11 & 8.156930e-06\\
C12 & 2.965249e-04\\
N14 & 6.972393e-04\\
N15 & 2.786798e-06\\
Si28 & 4.593614e-03\\
Si29 & 2.415835e-04\\
Si30 & 1.647335e-04\\
P31 & 2.499975e-04\\
S & 1.000016e-04\\
Ti46 & 7.920085e-05\\
Ti47 & 7.297781e-05\\
Ti48 & 7.381413e-04\\
Ti49 & 5.532184e-05\\
Ti50 & 5.404879e-05\\
Cr50 & 7.303946e-03\\
Cr52 & 1.464747e-01\\
Cr53 & 1.692872e-02\\
Cr54 & 4.293349e-03\\
Mn55 & 1.799997e-02\\
Fe54 & 3.660823e-02\\
Fe56 & 5.959297e-01\\
Fe57 & 1.400876e-02\\
Fe58 & 1.896969e-03\\
Co59 & 4.999999e-04\\
Ni58 & 8.231680e-02\\
Ni60 & 3.280046e-02\\
Ni61 & 1.449607e-03\\
Ni62 & 4.697640e-03\\
Ni64 & 1.234979e-03\\
Cu63 & 2.054981e-03\\
Cu65 & 9.450179e-04\\
Nb93 & 1.000001e-04\\
Mo92 & 3.554367e-03\\
Mo94 & 2.263664e-03\\
Mo95 & 3.937466e-03\\
Mo96 & 4.168845e-03\\
Mo97 & 2.411755e-03\\
Mo98 & 6.156638e-03\\
Mo100 & 2.507290e-03\\
Ta181 & 9.999969e-05\\

\caption{Table showing the isotopic description of material EppDucts}
\label{table:material_EppDucts}
\end{longtable}\clearpage

\begin{longtable}[ht!]
{ p{0.3\textwidth} | p{0.3\textwidth} }
\hline
Nuclide & Mass Fraction\\
\hline
H1 & 6.386261e-04\\
He4 & 3.220044e-03\\
B10 & 1.154247e-06\\
B11 & 5.108293e-06\\
C12 & 6.839086e-03\\
N14 & 1.718415e-03\\
N15 & 3.455416e-06\\
O16 & 1.269236e-02\\
Mg & 8.473257e-04\\
Al27 & 3.419589e-03\\
Si28 & 1.035880e-02\\
Si29 & 5.471327e-04\\
Si30 & 3.740661e-04\\
P31 & 2.818109e-04\\
S & 6.669566e-04\\
K & 3.130979e-06\\
Ti46 & 9.536886e-04\\
Ti47 & 8.787528e-04\\
Ti48 & 8.891961e-03\\
Ti49 & 6.661519e-04\\
Ti50 & 6.508222e-04\\
V & 2.504891e-05\\
Cr50 & 4.443454e-03\\
Cr52 & 8.910932e-02\\
Cr53 & 1.029885e-02\\
Cr54 & 2.611948e-03\\
Mn55 & 1.252507e-02\\
Fe54 & 2.295321e-02\\
Fe56 & 3.736446e-01\\
Fe57 & 8.783409e-03\\
Fe58 & 1.189398e-03\\
Co59 & 3.131276e-04\\
Ni58 & 5.317565e-02\\
Ni60 & 2.119099e-02\\
Ni61 & 9.412247e-04\\
Ni62 & 3.036404e-03\\
Ni64 & 8.081852e-04\\
Cu63 & 2.102419e-01\\
Cu65 & 9.700064e-02\\
Zr & 1.252369e-05\\
Nb93 & 1.828886e-02\\
Mo92 & 2.225932e-03\\
Mo94 & 1.417623e-03\\
Mo95 & 2.465848e-03\\
Mo96 & 2.610754e-03\\
Mo97 & 1.510374e-03\\
Mo98 & 3.855621e-03\\
Mo100 & 1.570197e-03\\
Sn & 1.252512e-05\\
Ta181 & 6.262504e-05\\
W182 & 1.644269e-06\\
W183 & 8.931100e-07\\
W184 & 1.922345e-06\\
W186 & 1.802741e-06\\
Pb206 & 1.216953e-06\\
Pb207 & 1.121388e-06\\
Pb208 & 2.671709e-06\\
Bi209 & 5.010036e-06]\\

\caption{Table showing the isotopic description of material M906}
\label{table:material_M906}
\end{longtable}\clearpage

\begin{longtable}[ht!]
{ p{0.3\textwidth} | p{0.3\textwidth} }
\hline
Nuclide & Mass Fraction\\
\hline
\\
He4 & 3.098619e-03\\
B10 & 1.699068e-04\\
B11 & 7.519497e-04\\
C12 & 2.047632e-04\\
N14 & 9.518301e-04\\
N15 & 3.765986e-06\\
O16 & 1.298416e-02\\
Al27 & 6.157792e-04\\
Si28 & 1.264583e-02\\
Si29 & 6.661027e-04\\
Si30 & 4.539647e-04\\
P31 & 3.071415e-04\\
S & 2.047662e-04\\
K & 7.203318e-05\\
Ti46 & 1.102498e-04\\
Ti47 & 1.015870e-04\\
Ti48 & 1.027943e-03\\
Ti49 & 7.700943e-05\\
Ti50 & 7.523736e-05\\
V & 2.730040e-05\\
Cr50 & 4.919782e-03\\
Cr52 & 9.866164e-02\\
Cr53 & 1.140286e-02\\
Cr54 & 2.891929e-03\\
Mn55 & 1.365088e-02\\
Fe54 & 2.501631e-02\\
Fe56 & 4.072277e-01\\
Fe57 & 9.572885e-03\\
Fe58 & 1.296304e-03\\
Co59 & 3.412721e-04\\
Ni58 & 5.503822e-02\\
Ni60 & 2.193083e-02\\
Ni61 & 9.692286e-04\\
Ni62 & 3.140908e-03\\
Ni64 & 8.257244e-04\\
Cu63 & 1.755623e-01\\
Cu65 & 8.073510e-02\\
Zr & 1.364937e-05\\
Nb93 & 2.218681e-02\\
Mo92 & 2.426000e-03\\
Mo94 & 1.545044e-03\\
Mo95 & 2.687481e-03\\
Mo96 & 2.845421e-03\\
Mo97 & 1.646130e-03\\
Mo98 & 4.202163e-03\\
Mo100 & 1.711333e-03\\
Sn & 7.386505e-03\\
Ta181 & 5.597890e-03\\
W182 & 1.792061e-06\\
W183 & 9.733890e-07\\
W184 & 2.095129e-06\\
W186 & 1.964781e-06\\
Pb206 & 1.326335e-06\\
Pb207 & 1.222183e-06\\
Pb208 & 2.911851e-06\\
Bi209 & 5.460353e-06\\

\caption{Table showing the isotopic description of material M907}
\label{table:material_M907}
\end{longtable}\clearpage

\begin{longtable}[ht!]
{ p{0.3\textwidth} | p{0.3\textwidth} }
\hline
Nuclide & Mass Fraction\\
\hline
H1 & 4.122292e-04\\
H2 & 9.475075e-08\\
B10 & 1.836323e-06\\
B11 & 8.126946e-06\\
C12 & 2.954349e-04\\
N14 & 6.946763e-04\\
N15 & 2.776554e-06\\
O16 & 3.263613e-03\\
Si28 & 4.576728e-03\\
Si29 & 2.406955e-04\\
Si30 & 1.641280e-04\\
P31 & 2.490785e-04\\
S & 9.963397e-05\\
Ti46 & 7.890971e-05\\
Ti47 & 7.270954e-05\\
Ti48 & 7.354280e-04\\
Ti49 & 5.511848e-05\\
Ti50 & 5.385011e-05\\
Cr50 & 7.277097e-03\\
Cr52 & 1.459363e-01\\
Cr53 & 1.686649e-02\\
Cr54 & 4.277567e-03\\
Mn55 & 1.793380e-02\\
Fe54 & 3.647366e-02\\
Fe56 & 5.937391e-01\\
Fe57 & 1.395727e-02\\
Fe58 & 1.889996e-03\\
Co59 & 4.981619e-04\\
Ni58 & 8.201421e-02\\
Ni60 & 3.267989e-02\\
Ni61 & 1.444278e-03\\
Ni62 & 4.680372e-03\\
Ni64 & 1.230439e-03\\
Cu63 & 2.047427e-03\\
Cu65 & 9.415441e-04\\
Nb93 & 9.963253e-05\\
Mo92 & 3.541301e-03\\
Mo94 & 2.255343e-03\\
Mo95 & 3.922992e-03\\
Mo96 & 4.153521e-03\\
Mo97 & 2.402889e-03\\
Mo98 & 6.134007e-03\\
Mo100 & 2.498074e-03\\
Ta181 & 9.963210e-05\\

\caption{Table showing the isotopic description of material UPDSM}
\label{table:material_UPDSM}
\end{longtable}\clearpage

\begin{longtable}[ht!]
{ p{0.3\textwidth} | p{0.3\textwidth} }
\hline
Nuclide & Mass Fraction\\
\hline
\\
B10 & 1.843098e-06\\
B11 & 8.156930e-06\\
C12 & 2.965249e-04\\
N14 & 6.972393e-04\\
N15 & 2.786798e-06\\
Si28 & 4.593614e-03\\
Si29 & 2.415835e-04\\
Si30 & 1.647335e-04\\
P31 & 2.499975e-04\\
S & 1.000016e-04\\
Ti46 & 7.920085e-05\\
Ti47 & 7.297781e-05\\
Ti48 & 7.381413e-04\\
Ti49 & 5.532184e-05\\
Ti50 & 5.404879e-05\\
Cr50 & 7.303946e-03\\
Cr52 & 1.464747e-01\\
Cr53 & 1.692872e-02\\
Cr54 & 4.293349e-03\\
Mn55 & 1.799997e-02\\
Fe54 & 3.660823e-02\\
Fe56 & 5.959297e-01\\
Fe57 & 1.400876e-02\\
Fe58 & 1.896969e-03\\
Co59 & 4.999999e-04\\
Ni58 & 8.231680e-02\\
Ni60 & 3.280046e-02\\
Ni61 & 1.449607e-03\\
Ni62 & 4.697640e-03\\
Ni64 & 1.234979e-03\\
Cu63 & 2.054981e-03\\
Cu65 & 9.450179e-04\\
Nb93 & 1.000001e-04\\
Mo92 & 3.554367e-03\\
Mo94 & 2.263664e-03\\
Mo95 & 3.937466e-03\\
Mo96 & 4.168845e-03\\
Mo97 & 2.411755e-03\\
Mo98 & 6.156638e-03\\
Mo100 & 2.507290e-03\\
Ta181 & 9.999969e-05\\

\caption{Table showing the isotopic description of material EPTRAP}
\label{table:material_EPTRAP}
\end{longtable}\clearpage

\begin{longtable}[ht!]
{ p{0.3\textwidth} | p{0.3\textwidth} }
\hline
Nuclide & Mass Fraction\\
\hline
\\
B10 & 1.843098e-06\\
B11 & 8.156930e-06\\
C12 & 2.965249e-04\\
N14 & 6.972393e-04\\
N15 & 2.786798e-06\\
Si28 & 4.593614e-03\\
Si29 & 2.415835e-04\\
Si30 & 1.647335e-04\\
P31 & 2.499975e-04\\
S & 1.000016e-04\\
Ti46 & 7.920085e-05\\
Ti47 & 7.297781e-05\\
Ti48 & 7.381413e-04\\
Ti49 & 5.532184e-05\\
Ti50 & 5.404879e-05\\
Cr50 & 7.303946e-03\\
Cr52 & 1.464747e-01\\
Cr53 & 1.692872e-02\\
Cr54 & 4.293349e-03\\
Mn55 & 1.799997e-02\\
Fe54 & 3.660823e-02\\
Fe56 & 5.959297e-01\\
Fe57 & 1.400876e-02\\
Fe58 & 1.896969e-03\\
Co59 & 4.999999e-04\\
Ni58 & 8.231680e-02\\
Ni60 & 3.280046e-02\\
Ni61 & 1.449607e-03\\
Ni62 & 4.697640e-03\\
Ni64 & 1.234979e-03\\
Cu63 & 2.054981e-03\\
Cu65 & 9.450179e-04\\
Nb93 & 1.000001e-04\\
Mo92 & 3.554367e-03\\
Mo94 & 2.263664e-03\\
Mo95 & 3.937466e-03\\
Mo96 & 4.168845e-03\\
Mo97 & 2.411755e-03\\
Mo98 & 6.156638e-03\\
Mo100 & 2.507290e-03\\
Ta181 & 9.999969e-05\\

\caption{Table showing the isotopic description of material PPWater}
\label{table:material_PPWater}
\end{longtable}\clearpage

\begin{longtable}[ht!]
{ p{0.3\textwidth} | p{0.3\textwidth} }
\hline
Nuclide & Mass Fraction\\
\hline
\\
Al27 & 9.749971e-02\\
Si28 & 1.837444e-03\\
Si29 & 9.663327e-05\\
Si30 & 6.589366e-05\\
Mn55 & 9.999996e-03\\
Fe54 & 2.258229e-03\\
Fe56 & 3.676065e-02\\
Fe57 & 8.641477e-04\\
Fe58 & 1.170170e-04\\
Co59 & 5.000005e-04\\
Ni58 & 3.359869e-02\\
Ni60 & 1.338796e-02\\
Ni61 & 5.916763e-04\\
Ni62 & 1.917401e-03\\
Ni64 & 5.040734e-04\\
Cu63 & 5.461480e-01\\
Cu65 & 2.511552e-01\\
Nb93 & 9.999996e-04\\
Sn & 1.000003e-03\\
Ta181 & 4.999996e-04\\
Pb206 & 4.791233e-05\\
Pb207 & 4.414810e-05\\
Pb208 & 1.051841e-04\\

\caption{Table showing the isotopic description of material M303}
\label{table:material_M303}
\end{longtable}\clearpage

\begin{longtable}[ht!]
{ p{0.3\textwidth} | p{0.3\textwidth} }
\hline
Nuclide & Mass Fraction\\
\hline
\\
B10 & 1.843112e-05\\
B11 & 8.156993e-05\\
C12 & 7.907392e-04\\
Al27 & 3.500018e-03\\
Si28 & 9.187303e-03\\
Si29 & 4.831706e-04\\
Si30 & 3.294700e-04\\
P31 & 3.999987e-04\\
S & 3.000072e-04\\
Ti46 & 1.683034e-03\\
Ti47 & 1.550790e-03\\
Ti48 & 1.568564e-02\\
Ti49 & 1.175600e-03\\
Ti50 & 1.148546e-03\\
V & 2.999879e-03\\
Cr50 & 6.156236e-03\\
Cr52 & 1.234582e-01\\
Cr53 & 1.426861e-02\\
Cr54 & 3.618704e-03\\
Mn55 & 2.000011e-02\\
Fe54 & 2.947910e-02\\
Fe56 & 4.798757e-01\\
Fe57 & 1.128065e-02\\
Fe58 & 1.527545e-03\\
Co59 & 2.000021e-03\\
Ni58 & 1.713547e-01\\
Ni60 & 6.827897e-02\\
Ni61 & 3.017577e-03\\
Ni62 & 9.778832e-03\\
Ni64 & 2.570789e-03\\
Nb93 & 1.000009e-03\\
Mo92 & 1.777191e-03\\
Mo94 & 1.131836e-03\\
Mo95 & 1.968744e-03\\
Mo96 & 2.084445e-03\\
Mo97 & 1.205889e-03\\
Mo98 & 3.078348e-03\\
Mo100 & 1.253653e-03\\
Ta181 & 5.000035e-04\\

\caption{Table showing the isotopic description of material EPPCH}
\label{table:material_EPPCH}
\end{longtable}\clearpage

\begin{longtable}[ht!]
{ p{0.3\textwidth} | p{0.3\textwidth} }
\hline
Nuclide & Mass Fraction\\
\hline
\\
B10 & 1.442461e-01\\
B11 & 6.383841e-01\\
C12 & 2.148517e-01\\
C13 & 2.518076e-03\\

\caption{Table showing the isotopic description of material B4C}
\label{table:material_B4C}
\end{longtable}\clearpage

\begin{longtable}[ht!]
{ p{0.3\textwidth} | p{0.3\textwidth} }
\hline
Nuclide & Mass Fraction\\
\hline
H1 & 1.847040e-03\\
H2 & 5.536889e-07\\
B10 & 2.078773e-06\\
B11 & 8.447382e-06\\
C12 & 2.246913e-04\\
N14 & 6.589756e-04\\
N15 & 2.459362e-06\\
O16 & 1.468420e-02\\
Al27 & 5.622494e-04\\
Si28 & 4.416591e-03\\
Si29 & 2.247218e-04\\
Si30 & 1.481326e-04\\
P31 & 2.389155e-04\\
S & 7.355948e-05\\
K & 4.723997e-06\\
Ti46 & 1.297295e-04\\
Ti47 & 1.172395e-04\\
Ti48 & 1.163744e-03\\
Ti49 & 8.562027e-05\\
Ti50 & 8.214717e-05\\
V & 3.778995e-05\\
Cr50 & 7.333487e-03\\
Cr52 & 1.415351e-01\\
Cr53 & 1.605524e-02\\
Cr54 & 3.998101e-03\\
Mn55 & 1.707051e-02\\
Fe54 & 3.604376e-02\\
Fe56 & 5.659124e-01\\
Fe57 & 1.307070e-02\\
Fe58 & 1.739600e-03\\
Co59 & 5.028159e-04\\
Ni58 & 8.513362e-02\\
Ni60 & 3.287757e-02\\
Ni61 & 1.430997e-03\\
Ni62 & 4.568483e-03\\
Ni64 & 1.166425e-03\\
Cu63 & 1.472533e-02\\
Cu65 & 6.762427e-03\\
Zr & 4.164644e-05\\
Nb93 & 1.010637e-03\\
Mo92 & 3.582076e-03\\
Mo94 & 2.233810e-03\\
Mo95 & 3.845468e-03\\
Mo96 & 4.029974e-03\\
Mo97 & 2.307890e-03\\
Mo98 & 5.832650e-03\\
Mo100 & 2.328829e-03\\
Sn & 1.890806e-05\\
Ta181 & 1.034368e-04\\
W182 & 2.505661e-06\\
W183 & 1.331036e-06\\
W184 & 2.915618e-06\\
W186 & 2.705784e-06\\
Pb206 & 1.819663e-06\\
Pb207 & 1.667171e-06\\
Pb208 & 3.944902e-06\\
Bi209 & 7.547778e-06\\

\caption{Table showing the isotopic description of material ShieldBlock}
\label{table:material_ShieldBlock}
\end{longtable}\clearpage

\begin{longtable}[ht!]
{ p{0.3\textwidth} | p{0.3\textwidth} }
\hline
Nuclide & Mass Fraction\\
\hline
\\
B10 & 1.843098e-06\\
B11 & 8.156930e-06\\
C12 & 2.965249e-04\\
N14 & 6.972393e-04\\
N15 & 2.786798e-06\\
Si28 & 4.593614e-03\\
Si29 & 2.415835e-04\\
Si30 & 1.647335e-04\\
P31 & 2.499975e-04\\
S & 1.000016e-04\\
Ti46 & 7.920085e-05\\
Ti47 & 7.297781e-05\\
Ti48 & 7.381413e-04\\
Ti49 & 5.532184e-05\\
Ti50 & 5.404879e-05\\
Cr50 & 7.303946e-03\\
Cr52 & 1.464747e-01\\
Cr53 & 1.692872e-02\\
Cr54 & 4.293349e-03\\
Mn55 & 1.799997e-02\\
Fe54 & 3.660823e-02\\
Fe56 & 5.959297e-01\\
Fe57 & 1.400876e-02\\
Fe58 & 1.896969e-03\\
Co59 & 4.999999e-04\\
Ni58 & 8.231680e-02\\
Ni60 & 3.280046e-02\\
Ni61 & 1.449607e-03\\
Ni62 & 4.697640e-03\\
Ni64 & 1.234979e-03\\
Cu63 & 2.054981e-03\\
Cu65 & 9.450179e-04\\
Nb93 & 1.000001e-04\\
Mo92 & 3.554367e-03\\
Mo94 & 2.263664e-03\\
Mo95 & 3.937466e-03\\
Mo96 & 4.168845e-03\\
Mo97 & 2.411755e-03\\
Mo98 & 6.156638e-03\\
Mo100 & 2.507290e-03\\
Ta181 & 9.999969e-05\\

\caption{Table showing the isotopic description of material UppDucts}
\label{table:material_UppDucts}
\end{longtable}\clearpage

\begin{longtable}[ht!]
{ p{0.3\textwidth} | p{0.3\textwidth} }
\hline
Nuclide & Mass Fraction\\
\hline
H1 & 5.556235e-03\\
H2 & 1.277099e-06\\
O16 & 4.968034e-01\\
Na23 & 1.710308e-02\\
Mg & 2.563458e-03\\
Al27 & 4.697334e-02\\
Si28 & 2.895215e-01\\
Si29 & 1.522632e-02\\
Si30 & 1.038271e-02\\
S & 1.281754e-03\\
K & 1.924422e-02\\
Ca & 8.294590e-02\\
Fe54 & 6.998641e-04\\
Fe56 & 1.139283e-02\\
Fe57 & 2.678156e-04\\
Fe58 & 3.626579e-05\\

\caption{Table showing the isotopic description of material M200}
\label{table:material_M200}
\end{longtable}\clearpage

\begin{longtable}[ht!]
{ p{0.3\textwidth} | p{0.3\textwidth} }
\hline
Nuclide & Mass Fraction\\
\hline
\\
B10 & 1.843098e-06\\
B11 & 8.156930e-06\\
C12 & 2.965249e-04\\
N14 & 6.972393e-04\\
N15 & 2.786798e-06\\
Si28 & 4.593614e-03\\
Si29 & 2.415835e-04\\
Si30 & 1.647335e-04\\
P31 & 2.499975e-04\\
S & 1.000016e-04\\
Ti46 & 7.920085e-05\\
Ti47 & 7.297781e-05\\
Ti48 & 7.381413e-04\\
Ti49 & 5.532184e-05\\
Ti50 & 5.404879e-05\\
Cr50 & 7.303946e-03\\
Cr52 & 1.464747e-01\\
Cr53 & 1.692872e-02\\
Cr54 & 4.293349e-03\\
Mn55 & 1.799997e-02\\
Fe54 & 3.660823e-02\\
Fe56 & 5.959297e-01\\
Fe57 & 1.400876e-02\\
Fe58 & 1.896969e-03\\
Co59 & 4.999999e-04\\
Ni58 & 8.231680e-02\\
Ni60 & 3.280046e-02\\
Ni61 & 1.449607e-03\\
Ni62 & 4.697640e-03\\
Ni64 & 1.234979e-03\\
Cu63 & 2.054981e-03\\
Cu65 & 9.450179e-04\\
Nb93 & 1.000001e-04\\
Mo92 & 3.554367e-03\\
Mo94 & 2.263664e-03\\
Mo95 & 3.937466e-03\\
Mo96 & 4.168845e-03\\
Mo97 & 2.411755e-03\\
Mo98 & 6.156638e-03\\
Mo100 & 2.507290e-03\\
Ta181 & 9.999969e-05\\

\caption{Table showing the isotopic description of material EppDiagPipes}
\label{table:material_EppDiagPipes}
\end{longtable}\clearpage

\begin{longtable}[ht!]
{ p{0.3\textwidth} | p{0.3\textwidth} }
\hline
Nuclide & Mass Fraction\\
\hline
\\
B10 & 1.843098e-06\\
B11 & 8.156930e-06\\
C12 & 2.965249e-04\\
N14 & 6.972393e-04\\
N15 & 2.786798e-06\\
Si28 & 4.593614e-03\\
Si29 & 2.415835e-04\\
Si30 & 1.647335e-04\\
P31 & 2.499975e-04\\
S & 1.000016e-04\\
Ti46 & 7.920085e-05\\
Ti47 & 7.297781e-05\\
Ti48 & 7.381413e-04\\
Ti49 & 5.532184e-05\\
Ti50 & 5.404879e-05\\
Cr50 & 7.303946e-03\\
Cr52 & 1.464747e-01\\
Cr53 & 1.692872e-02\\
Cr54 & 4.293349e-03\\
Mn55 & 1.799997e-02\\
Fe54 & 3.660823e-02\\
Fe56 & 5.959297e-01\\
Fe57 & 1.400876e-02\\
Fe58 & 1.896969e-03\\
Co59 & 4.999999e-04\\
Ni58 & 8.231680e-02\\
Ni60 & 3.280046e-02\\
Ni61 & 1.449607e-03\\
Ni62 & 4.697640e-03\\
Ni64 & 1.234979e-03\\
Cu63 & 2.054981e-03\\
Cu65 & 9.450179e-04\\
Nb93 & 1.000001e-04\\
Mo92 & 3.554367e-03\\
Mo94 & 2.263664e-03\\
Mo95 & 3.937466e-03\\
Mo96 & 4.168845e-03\\
Mo97 & 2.411755e-03\\
Mo98 & 6.156638e-03\\
Mo100 & 2.507290e-03\\
Ta181 & 9.999969e-05\\

\caption{Table showing the isotopic description of material EppWaterPipes}
\label{table:material_EppWaterPipes}
\end{longtable}\clearpage

\begin{longtable}[ht!]
{ p{0.3\textwidth} | p{0.3\textwidth} }
\hline
Nuclide & Mass Fraction\\
\hline
\\
B10 & 1.843098e-06\\
B11 & 8.156930e-06\\
C12 & 2.965249e-04\\
N14 & 6.972393e-04\\
N15 & 2.786798e-06\\
Si28 & 4.593614e-03\\
Si29 & 2.415835e-04\\
Si30 & 1.647335e-04\\
P31 & 2.499975e-04\\
S & 1.000016e-04\\
Ti46 & 7.920085e-05\\
Ti47 & 7.297781e-05\\
Ti48 & 7.381413e-04\\
Ti49 & 5.532184e-05\\
Ti50 & 5.404879e-05\\
Cr50 & 7.303946e-03\\
Cr52 & 1.464747e-01\\
Cr53 & 1.692872e-02\\
Cr54 & 4.293349e-03\\
Mn55 & 1.799997e-02\\
Fe54 & 3.660823e-02\\
Fe56 & 5.959297e-01\\
Fe57 & 1.400876e-02\\
Fe58 & 1.896969e-03\\
Co59 & 4.999999e-04\\
Ni58 & 8.231680e-02\\
Ni60 & 3.280046e-02\\
Ni61 & 1.449607e-03\\
Ni62 & 4.697640e-03\\
Ni64 & 1.234979e-03\\
Cu63 & 2.054981e-03\\
Cu65 & 9.450179e-04\\
Nb93 & 1.000001e-04\\
Mo92 & 3.554367e-03\\
Mo94 & 2.263664e-03\\
Mo95 & 3.937466e-03\\
Mo96 & 4.168845e-03\\
Mo97 & 2.411755e-03\\
Mo98 & 6.156638e-03\\
Mo100 & 2.507290e-03\\
Ta181 & 9.999969e-05\\

\caption{Table showing the isotopic description of material EppDIagBox}
\label{table:material_EppDIagBox}
\end{longtable}\clearpage

\begin{longtable}[ht!]
{ p{0.3\textwidth} | p{0.3\textwidth} }
\hline
Nuclide & Mass Fraction\\
\hline
H1 & 5.454527e-03\\
H2 & 1.253721e-06\\
B10 & 1.753451e-06\\
B11 & 7.760184e-06\\
C12 & 2.821022e-04\\
N14 & 6.633262e-04\\
N15 & 2.651251e-06\\
O16 & 4.318342e-02\\
Si28 & 4.370185e-03\\
Si29 & 2.298331e-04\\
Si30 & 1.567210e-04\\
P31 & 2.378378e-04\\
S & 9.513758e-05\\
Ti46 & 7.534858e-05\\
Ti47 & 6.942822e-05\\
Ti48 & 7.022387e-04\\
Ti49 & 5.263103e-05\\
Ti50 & 5.141990e-05\\
Cr50 & 6.948688e-03\\
Cr52 & 1.393503e-01\\
Cr53 & 1.610532e-02\\
Cr54 & 4.084524e-03\\
Mn55 & 1.712447e-02\\
Fe54 & 3.482764e-02\\
Fe56 & 5.669441e-01\\
Fe57 & 1.332739e-02\\
Fe58 & 1.804702e-03\\
Co59 & 4.756803e-04\\
Ni58 & 7.831298e-02\\
Ni60 & 3.120507e-02\\
Ni61 & 1.379099e-03\\
Ni62 & 4.469151e-03\\
Ni64 & 1.174910e-03\\
Cu63 & 1.955029e-03\\
Cu65 & 8.990530e-04\\
Nb93 & 9.513620e-05\\
Mo92 & 3.381485e-03\\
Mo94 & 2.153561e-03\\
Mo95 & 3.745951e-03\\
Mo96 & 3.966076e-03\\
Mo97 & 2.294449e-03\\
Mo98 & 5.857184e-03\\
Mo100 & 2.385338e-03\\
Ta181 & 9.513578e-05\\

\caption{Table showing the isotopic description of material UPDFW2}
\label{table:material_UPDFW2}
\end{longtable}\clearpage

\begin{longtable}[ht!]
{ p{0.3\textwidth} | p{0.3\textwidth} }
\hline
Nuclide & Mass Fraction\\
\hline
\\
B10 & 1.843098e-06\\
B11 & 8.156930e-06\\
C12 & 2.965249e-04\\
N14 & 6.972393e-04\\
N15 & 2.786798e-06\\
Si28 & 4.593614e-03\\
Si29 & 2.415835e-04\\
Si30 & 1.647335e-04\\
P31 & 2.499975e-04\\
S & 1.000016e-04\\
Ti46 & 7.920085e-05\\
Ti47 & 7.297781e-05\\
Ti48 & 7.381413e-04\\
Ti49 & 5.532184e-05\\
Ti50 & 5.404879e-05\\
Cr50 & 7.303946e-03\\
Cr52 & 1.464747e-01\\
Cr53 & 1.692872e-02\\
Cr54 & 4.293349e-03\\
Mn55 & 1.799997e-02\\
Fe54 & 3.660823e-02\\
Fe56 & 5.959297e-01\\
Fe57 & 1.400876e-02\\
Fe58 & 1.896969e-03\\
Co59 & 4.999999e-04\\
Ni58 & 8.231680e-02\\
Ni60 & 3.280046e-02\\
Ni61 & 1.449607e-03\\
Ni62 & 4.697640e-03\\
Ni64 & 1.234979e-03\\
Cu63 & 2.054981e-03\\
Cu65 & 9.450179e-04\\
Nb93 & 1.000001e-04\\
Mo92 & 3.554367e-03\\
Mo94 & 2.263664e-03\\
Mo95 & 3.937466e-03\\
Mo96 & 4.168845e-03\\
Mo97 & 2.411755e-03\\
Mo98 & 6.156638e-03\\
Mo100 & 2.507290e-03\\
Ta181 & 9.999969e-05\\

\caption{Table showing the isotopic description of material PPWheelsDrives}
\label{table:material_PPWheelsDrives}
\end{longtable}\clearpage

\begin{longtable}[ht!]
{ p{0.3\textwidth} | p{0.3\textwidth} }
\hline
Nuclide & Mass Fraction\\
\hline
\\
Mg24 & 1.886648e-01\\
Mg25 & 2.488124e-02\\
Mg26 & 2.848708e-02\\
Al27 & 1.932231e-01\\
Si28 & 1.482315e-01\\
Si29 & 7.799320e-03\\
Si30 & 5.324541e-03\\
Ti46 & 2.396156e-03\\
Ti47 & 2.207881e-03\\
Ti48 & 2.234118e-02\\
Ti49 & 1.673716e-03\\
Ti50 & 1.635200e-03\\
Cr50 & 2.946330e-03\\
Cr52 & 5.908588e-02\\
Cr53 & 6.828873e-03\\
Cr54 & 1.731903e-03\\
Mn55 & 3.025413e-02\\
Fe54 & 7.970736e-03\\
Fe56 & 1.297521e-01\\
Fe57 & 3.050137e-03\\
Fe58 & 4.130316e-04\\
Cu63 & 5.524743e-02\\
Cu65 & 2.543026e-02\\
Zn64 & 2.424389e-02\\
Zn66 & 1.409971e-02\\
Zn67 & 2.085388e-03\\
Zn68 & 9.665589e-03\\
Zn70 & 3.289786e-04\\

\caption{Table showing the isotopic description of material UppExFrames}
\label{table:material_UppExFrames}
\end{longtable}\clearpage

\begin{longtable}[ht!]
{ p{0.3\textwidth} | p{0.3\textwidth} }
\hline
Nuclide & Mass Fraction\\
\hline
\\
B10 & 1.843098e-06\\
B11 & 8.156930e-06\\
C12 & 2.965249e-04\\
N14 & 6.972393e-04\\
N15 & 2.786798e-06\\
Si28 & 4.593614e-03\\
Si29 & 2.415835e-04\\
Si30 & 1.647335e-04\\
P31 & 2.499975e-04\\
S & 1.000016e-04\\
Ti46 & 7.920085e-05\\
Ti47 & 7.297781e-05\\
Ti48 & 7.381413e-04\\
Ti49 & 5.532184e-05\\
Ti50 & 5.404879e-05\\
Cr50 & 7.303946e-03\\
Cr52 & 1.464747e-01\\
Cr53 & 1.692872e-02\\
Cr54 & 4.293349e-03\\
Mn55 & 1.799997e-02\\
Fe54 & 3.660823e-02\\
Fe56 & 5.959297e-01\\
Fe57 & 1.400876e-02\\
Fe58 & 1.896969e-03\\
Co59 & 4.999999e-04\\
Ni58 & 8.231680e-02\\
Ni60 & 3.280046e-02\\
Ni61 & 1.449607e-03\\
Ni62 & 4.697640e-03\\
Ni64 & 1.234979e-03\\
Cu63 & 2.054981e-03\\
Cu65 & 9.450179e-04\\
Nb93 & 1.000001e-04\\
Mo92 & 3.554367e-03\\
Mo94 & 2.263664e-03\\
Mo95 & 3.937466e-03\\
Mo96 & 4.168845e-03\\
Mo97 & 2.411755e-03\\
Mo98 & 6.156638e-03\\
Mo100 & 2.507290e-03\\
Ta181 & 9.999969e-05\\

\caption{Table showing the isotopic description of material PPF}
\label{table:material_PPF}
\end{longtable}\clearpage

\begin{longtable}[ht!]
{ p{0.3\textwidth} | p{0.3\textwidth} }
\hline
Nuclide & Mass Fraction\\
\hline
\\
B10 & 1.843098e-06\\
B11 & 8.156930e-06\\
C12 & 2.965249e-04\\
N14 & 6.972393e-04\\
N15 & 2.786798e-06\\
Si28 & 4.593614e-03\\
Si29 & 2.415835e-04\\
Si30 & 1.647335e-04\\
P31 & 2.499975e-04\\
S & 1.000016e-04\\
Ti46 & 7.920085e-05\\
Ti47 & 7.297781e-05\\
Ti48 & 7.381413e-04\\
Ti49 & 5.532184e-05\\
Ti50 & 5.404879e-05\\
Cr50 & 7.303946e-03\\
Cr52 & 1.464747e-01\\
Cr53 & 1.692872e-02\\
Cr54 & 4.293349e-03\\
Mn55 & 1.799997e-02\\
Fe54 & 3.660823e-02\\
Fe56 & 5.959297e-01\\
Fe57 & 1.400876e-02\\
Fe58 & 1.896969e-03\\
Co59 & 4.999999e-04\\
Ni58 & 8.231680e-02\\
Ni60 & 3.280046e-02\\
Ni61 & 1.449607e-03\\
Ni62 & 4.697640e-03\\
Ni64 & 1.234979e-03\\
Cu63 & 2.054981e-03\\
Cu65 & 9.450179e-04\\
Nb93 & 1.000001e-04\\
Mo92 & 3.554367e-03\\
Mo94 & 2.263664e-03\\
Mo95 & 3.937466e-03\\
Mo96 & 4.168845e-03\\
Mo97 & 2.411755e-03\\
Mo98 & 6.156638e-03\\
Mo100 & 2.507290e-03\\
Ta181 & 9.999969e-05\\

\caption{Table showing the isotopic description of material UPPFW}
\label{table:material_UPPFW}
\end{longtable}\clearpage

\begin{longtable}[ht!]
{ p{0.3\textwidth} | p{0.3\textwidth} }
\hline
Nuclide & Mass Fraction\\
\hline
H1 & 1.195209e-02\\
H2 & 2.747184e-06\\
B10 & 1.646661e-06\\
B11 & 7.287539e-06\\
C12 & 2.649211e-04\\
N14 & 6.229271e-04\\
N15 & 2.489779e-06\\
O16 & 9.462485e-02\\
Si28 & 4.104019e-03\\
Si29 & 2.158360e-04\\
Si30 & 1.471767e-04\\
P31 & 2.233526e-04\\
S & 8.934346e-05\\
Ti46 & 7.075966e-05\\
Ti47 & 6.519978e-05\\
Ti48 & 6.594697e-04\\
Ti49 & 4.942570e-05\\
Ti50 & 4.828815e-05\\
Cr50 & 6.525487e-03\\
Cr52 & 1.308633e-01\\
Cr53 & 1.512446e-02\\
Cr54 & 3.835761e-03\\
Mn55 & 1.608154e-02\\
Fe54 & 3.270650e-02\\
Fe56 & 5.324162e-01\\
Fe57 & 1.251572e-02\\
Fe58 & 1.694793e-03\\
Co59 & 4.467107e-04\\
Ni58 & 7.354343e-02\\
Ni60 & 2.930445e-02\\
Ni61 & 1.295109e-03\\
Ni62 & 4.196951e-03\\
Ni64 & 1.103354e-03\\
Cu63 & 1.835962e-03\\
Cu65 & 8.442973e-04\\
Nb93 & 8.934187e-05\\
Mo92 & 3.175533e-03\\
Mo94 & 2.022394e-03\\
Mo95 & 3.517801e-03\\
Mo96 & 3.724509e-03\\
Mo97 & 2.154712e-03\\
Mo98 & 5.500460e-03\\
Mo100 & 2.240058e-03\\
Ta181 & 8.934166e-05\\

\caption{Table showing the isotopic description of material M170}
\label{table:material_M170}
\end{longtable}\clearpage

\begin{longtable}[ht!]
{ p{0.3\textwidth} | p{0.3\textwidth} }
\hline
Nuclide & Mass Fraction\\
\hline
\\
B10 & 5.529302e-05\\
B11 & 2.447075e-04\\
C12 & 2.965248e-04\\
N14 & 9.960573e-04\\
N15 & 3.981142e-06\\
Si28 & 9.187247e-03\\
Si29 & 4.831668e-04\\
Si30 & 3.294675e-04\\
P31 & 2.999971e-04\\
S & 2.000030e-04\\
Cr50 & 7.199606e-03\\
Cr52 & 1.443820e-01\\
Cr53 & 1.668693e-02\\
Cr54 & 4.232016e-03\\
Mn55 & 2.000005e-02\\
Fe54 & 3.663419e-02\\
Fe56 & 5.963522e-01\\
Fe57 & 1.401877e-02\\
Fe58 & 1.898314e-03\\
Co59 & 1.000000e-03\\
Ni58 & 8.063689e-02\\
Ni60 & 3.213105e-02\\
Ni61 & 1.420018e-03\\
Ni62 & 4.601767e-03\\
Ni64 & 1.209775e-03\\
Nb93 & 4.999994e-04\\
Mo92 & 3.554365e-03\\
Mo94 & 2.263663e-03\\
Mo95 & 3.937464e-03\\
Mo96 & 4.168843e-03\\
Mo97 & 2.411753e-03\\
Mo98 & 6.156635e-03\\
Mo100 & 2.507289e-03\\

\caption{Table showing the isotopic description of material M111}
\label{table:material_M111}
\end{longtable}\clearpage

\begin{longtable}[ht!]
{ p{0.3\textwidth} | p{0.3\textwidth} }
\hline
Nuclide & Mass Fraction\\
\hline
\\
B10 & 5.529303e-05\\
B11 & 2.447076e-04\\
C12 & 2.965248e-04\\
N14 & 1.593692e-03\\
N15 & 6.369823e-06\\
Si28 & 9.187249e-03\\
Si29 & 4.831669e-04\\
Si30 & 3.294676e-04\\
P31 & 2.999971e-04\\
S & 2.000031e-04\\
Cr50 & 7.199608e-03\\
Cr52 & 1.443820e-01\\
Cr53 & 1.668693e-02\\
Cr54 & 4.232017e-03\\
Mn55 & 2.000005e-02\\
Fe54 & 3.660032e-02\\
Fe56 & 5.958007e-01\\
Fe57 & 1.400578e-02\\
Fe58 & 1.896568e-03\\
Co59 & 1.000001e-03\\
Ni58 & 8.063691e-02\\
Ni60 & 3.213106e-02\\
Ni61 & 1.420018e-03\\
Ni62 & 4.601768e-03\\
Ni64 & 1.209776e-03\\
Nb93 & 4.999995e-04\\
Mo92 & 3.554366e-03\\
Mo94 & 2.263663e-03\\
Mo95 & 3.937465e-03\\
Mo96 & 4.168844e-03\\
Mo97 & 2.411754e-03\\
Mo98 & 6.156637e-03\\
Mo100 & 2.507290e-03\\

\caption{Table showing the isotopic description of material M110}
\label{table:material_M110}
\end{longtable}\clearpage

\begin{longtable}[ht!]
{ p{0.3\textwidth} | p{0.3\textwidth} }
\hline
Nuclide & Mass Fraction\\
\hline
H1 & 1.121426e-01\\
H2 & 2.577594e-05\\
O16 & 8.878316e-01\\

\caption{Table showing the isotopic description of material M400}
\label{table:material_M400}
\end{longtable}\clearpage

\begin{longtable}[ht!]
{ p{0.3\textwidth} | p{0.3\textwidth} }
\hline
Nuclide & Mass Fraction\\
\hline
H1 & 1.466991e-03\\
H2 & 3.371874e-07\\
B10 & 1.818988e-06\\
B11 & 8.050226e-06\\
C12 & 2.926459e-04\\
N14 & 6.881184e-04\\
N15 & 2.750343e-06\\
O16 & 1.161415e-02\\
Si28 & 4.533523e-03\\
Si29 & 2.384233e-04\\
Si30 & 1.625786e-04\\
P31 & 2.467271e-04\\
S & 9.869340e-05\\
Ti46 & 7.816479e-05\\
Ti47 & 7.202315e-05\\
Ti48 & 7.284853e-04\\
Ti49 & 5.459815e-05\\
Ti50 & 5.334176e-05\\
Cr50 & 7.208399e-03\\
Cr52 & 1.445586e-01\\
Cr53 & 1.670727e-02\\
Cr54 & 4.237186e-03\\
Mn55 & 1.776451e-02\\
Fe54 & 3.612934e-02\\
Fe56 & 5.881340e-01\\
Fe57 & 1.382551e-02\\
Fe58 & 1.872153e-03\\
Co59 & 4.934592e-04\\
Ni58 & 8.123998e-02\\
Ni60 & 3.237138e-02\\
Ni61 & 1.430644e-03\\
Ni62 & 4.636188e-03\\
Ni64 & 1.218823e-03\\
Cu63 & 2.028099e-03\\
Cu65 & 9.326557e-04\\
Nb93 & 9.869198e-05\\
Mo92 & 3.507870e-03\\
Mo94 & 2.234052e-03\\
Mo95 & 3.885958e-03\\
Mo96 & 4.114311e-03\\
Mo97 & 2.380205e-03\\
Mo98 & 6.076101e-03\\
Mo100 & 2.474491e-03\\
Ta181 & 9.869155e-05\\

\caption{Table showing the isotopic description of material UPTRAP}
\label{table:material_UPTRAP}
\end{longtable}\clearpage

\begin{longtable}[ht!]
{ p{0.3\textwidth} | p{0.3\textwidth} }
\hline
Nuclide & Mass Fraction\\
\hline
\\
Mg24 & 1.886648e-01\\
Mg25 & 2.488124e-02\\
Mg26 & 2.848708e-02\\
Al27 & 1.932231e-01\\
Si28 & 1.482315e-01\\
Si29 & 7.799320e-03\\
Si30 & 5.324541e-03\\
Ti46 & 2.396156e-03\\
Ti47 & 2.207881e-03\\
Ti48 & 2.234118e-02\\
Ti49 & 1.673716e-03\\
Ti50 & 1.635200e-03\\
Cr50 & 2.946330e-03\\
Cr52 & 5.908588e-02\\
Cr53 & 6.828873e-03\\
Cr54 & 1.731903e-03\\
Mn55 & 3.025413e-02\\
Fe54 & 7.970736e-03\\
Fe56 & 1.297521e-01\\
Fe57 & 3.050137e-03\\
Fe58 & 4.130316e-04\\
Cu63 & 5.524743e-02\\
Cu65 & 2.543026e-02\\
Zn64 & 2.424389e-02\\
Zn66 & 1.409971e-02\\
Zn67 & 2.085388e-03\\
Zn68 & 9.665589e-03\\
Zn70 & 3.289786e-04\\

\caption{Table showing the isotopic description of material EppExFrames}
\label{table:material_EppExFrames}
\end{longtable}\clearpage

\begin{longtable}[ht!]
{ p{0.3\textwidth} | p{0.3\textwidth} }
\hline
Nuclide & Mass Fraction\\
\hline
H1 & 3.212605e-03\\
H2 & 7.384167e-07\\
B10 & 1.790300e-06\\
B11 & 7.923249e-06\\
O16 & 2.574419e-02\\
Mg & 3.885405e-04\\
Al27 & 2.914059e-05\\
Si28 & 3.569616e-04\\
Si29 & 1.877306e-05\\
Si30 & 1.280119e-05\\
P31 & 1.359879e-04\\
S & 3.885477e-05\\
Cr50 & 3.040592e-04\\
Cr52 & 6.097658e-03\\
Cr53 & 7.047341e-04\\
Cr54 & 1.787298e-04\\
Mn55 & 1.942713e-05\\
Fe54 & 1.096767e-05\\
Fe56 & 1.785373e-04\\
Fe57 & 4.196966e-06\\
Fe58 & 5.683249e-07\\
Co59 & 4.856775e-04\\
Ni58 & 1.958178e-04\\
Ni60 & 7.802652e-05\\
Ni61 & 3.448359e-06\\
Ni62 & 1.117486e-05\\
Ni64 & 2.937796e-06\\
Cu63 & 6.578606e-01\\
Cu65 & 3.025276e-01\\
Zr & 1.068370e-03\\
Sn & 9.713576e-05\\
Ta181 & 9.713515e-05\\
Pb206 & 2.326982e-05\\
Pb207 & 2.144172e-05\\
Pb208 & 5.108568e-05\\
Bi209 & 2.914063e-05\\

\caption{Table showing the isotopic description of material M623}
\label{table:material_M623}
\end{longtable}\clearpage

\begin{longtable}[ht!]
{ p{0.3\textwidth} | p{0.3\textwidth} }
\hline
Nuclide & Mass Fraction\\
\hline
H1 & 3.177371e-02\\
H2 & 7.303169e-06\\
B10 & 2.641778e-06\\
B11 & 1.169161e-05\\
C12 & 2.125101e-04\\
N14 & 4.996900e-04\\
N15 & 1.997206e-06\\
O16 & 2.515519e-01\\
Si28 & 3.292091e-03\\
Si29 & 1.731354e-04\\
Si30 & 1.180596e-04\\
P31 & 1.791652e-04\\
S & 7.166797e-05\\
Ti46 & 5.676085e-05\\
Ti47 & 5.230099e-05\\
Ti48 & 5.290034e-04\\
Ti49 & 3.964728e-05\\
Ti50 & 3.873488e-05\\
Cr50 & 5.234496e-03\\
Cr52 & 1.049738e-01\\
Cr53 & 1.213228e-02\\
Cr54 & 3.076909e-03\\
Mn55 & 1.290003e-02\\
Fe54 & 2.619917e-02\\
Fe56 & 4.264834e-01\\
Fe57 & 1.002554e-02\\
Fe58 & 1.357592e-03\\
Co59 & 3.583344e-04\\
Ni58 & 5.899375e-02\\
Ni60 & 2.350704e-02\\
Ni61 & 1.038889e-03\\
Ni62 & 3.366649e-03\\
Ni64 & 8.850723e-04\\
Cu63 & 1.472743e-03\\
Cu65 & 6.772640e-04\\
Nb93 & 7.166716e-04\\
Mo92 & 2.547297e-03\\
Mo94 & 1.622291e-03\\
Mo95 & 2.821851e-03\\
Mo96 & 2.987675e-03\\
Mo97 & 1.728435e-03\\
Mo98 & 4.412270e-03\\
Mo100 & 1.796891e-03\\
Ta181 & 7.166662e-05\\

\caption{Table showing the isotopic description of material M622}
\label{table:material_M622}
\end{longtable}\clearpage

\begin{longtable}[ht!]
{ p{0.3\textwidth} | p{0.3\textwidth} }
\hline
Nuclide & Mass Fraction\\
\hline
H1 & 5.950018e-04\\
H2 & 1.367611e-07\\
B10 & 3.666628e-06\\
B11 & 1.622728e-05\\
C12 & 2.949506e-04\\
N14 & 6.935400e-04\\
N15 & 2.772008e-06\\
O16 & 4.710643e-03\\
Si28 & 4.569239e-03\\
Si29 & 2.403015e-04\\
Si30 & 1.638596e-04\\
P31 & 2.486710e-04\\
S & 9.947128e-05\\
Ti46 & 7.878058e-05\\
Ti47 & 7.259048e-05\\
Ti48 & 7.342242e-04\\
Ti49 & 5.502828e-05\\
Ti50 & 5.376198e-05\\
Cr50 & 7.265198e-03\\
Cr52 & 1.456974e-01\\
Cr53 & 1.683887e-02\\
Cr54 & 4.270568e-03\\
Mn55 & 1.790443e-02\\
Fe54 & 3.636292e-02\\
Fe56 & 5.919362e-01\\
Fe57 & 1.391486e-02\\
Fe58 & 1.884264e-03\\
Co59 & 4.973469e-04\\
Ni58 & 8.188001e-02\\
Ni60 & 3.262635e-02\\
Ni61 & 1.441911e-03\\
Ni62 & 4.672706e-03\\
Ni64 & 1.228426e-03\\
Cu63 & 2.044084e-03\\
Cu65 & 9.400023e-04\\
Nb93 & 9.946949e-04\\
Mo92 & 3.535499e-03\\
Mo94 & 2.251654e-03\\
Mo95 & 3.916571e-03\\
Mo96 & 4.146718e-03\\
Mo97 & 2.398954e-03\\
Mo98 & 6.123969e-03\\
Mo100 & 2.493975e-03\\
Ta181 & 9.946933e-05\\

\caption{Table showing the isotopic description of material M621}
\label{table:material_M621}
\end{longtable}\clearpage

\begin{longtable}[ht!]
{ p{0.3\textwidth} | p{0.3\textwidth} }
\hline
Nuclide & Mass Fraction\\
\hline
\\
B10 & 1.843098e-06\\
B11 & 8.156930e-06\\
C12 & 2.965249e-04\\
N14 & 6.972393e-04\\
N15 & 2.786798e-06\\
Si28 & 4.593614e-03\\
Si29 & 2.415835e-04\\
Si30 & 1.647335e-04\\
P31 & 2.499975e-04\\
S & 1.000016e-04\\
Ti46 & 7.920085e-05\\
Ti47 & 7.297781e-05\\
Ti48 & 7.381413e-04\\
Ti49 & 5.532184e-05\\
Ti50 & 5.404879e-05\\
Cr50 & 7.303946e-03\\
Cr52 & 1.464747e-01\\
Cr53 & 1.692872e-02\\
Cr54 & 4.293349e-03\\
Mn55 & 1.799997e-02\\
Fe54 & 3.660823e-02\\
Fe56 & 5.959297e-01\\
Fe57 & 1.400876e-02\\
Fe58 & 1.896969e-03\\
Co59 & 4.999999e-04\\
Ni58 & 8.231680e-02\\
Ni60 & 3.280046e-02\\
Ni61 & 1.449607e-03\\
Ni62 & 4.697640e-03\\
Ni64 & 1.234979e-03\\
Cu63 & 2.054981e-03\\
Cu65 & 9.450179e-04\\
Nb93 & 1.000001e-04\\
Mo92 & 3.554367e-03\\
Mo94 & 2.263664e-03\\
Mo95 & 3.937466e-03\\
Mo96 & 4.168845e-03\\
Mo97 & 2.411755e-03\\
Mo98 & 6.156638e-03\\
Mo100 & 2.507290e-03\\
Ta181 & 9.999969e-05\\

\caption{Table showing the isotopic description of material EppDT}
\label{table:material_EppDT}
\end{longtable}\clearpage

\begin{longtable}[ht!]
{ p{0.3\textwidth} | p{0.3\textwidth} }
\hline
Nuclide & Mass Fraction\\
\hline
\\
B10 & 1.843098e-06\\
B11 & 8.156930e-06\\
C12 & 2.965249e-04\\
N14 & 6.972393e-04\\
N15 & 2.786798e-06\\
Si28 & 4.593614e-03\\
Si29 & 2.415835e-04\\
Si30 & 1.647335e-04\\
P31 & 2.499975e-04\\
S & 1.000016e-04\\
Ti46 & 7.920085e-05\\
Ti47 & 7.297781e-05\\
Ti48 & 7.381413e-04\\
Ti49 & 5.532184e-05\\
Ti50 & 5.404879e-05\\
Cr50 & 7.303946e-03\\
Cr52 & 1.464747e-01\\
Cr53 & 1.692872e-02\\
Cr54 & 4.293349e-03\\
Mn55 & 1.799997e-02\\
Fe54 & 3.660823e-02\\
Fe56 & 5.959297e-01\\
Fe57 & 1.400876e-02\\
Fe58 & 1.896969e-03\\
Co59 & 4.999999e-04\\
Ni58 & 8.231680e-02\\
Ni60 & 3.280046e-02\\
Ni61 & 1.449607e-03\\
Ni62 & 4.697640e-03\\
Ni64 & 1.234979e-03\\
Cu63 & 2.054981e-03\\
Cu65 & 9.450179e-04\\
Nb93 & 1.000001e-04\\
Mo92 & 3.554367e-03\\
Mo94 & 2.263664e-03\\
Mo95 & 3.937466e-03\\
Mo96 & 4.168845e-03\\
Mo97 & 2.411755e-03\\
Mo98 & 6.156638e-03\\
Mo100 & 2.507290e-03\\
Ta181 & 9.999969e-05\\

\caption{Table showing the isotopic description of material PPWheels}
\label{table:material_PPWheels}
\end{longtable}\clearpage

\begin{longtable}[ht!]
{ p{0.3\textwidth} | p{0.3\textwidth} }
\hline
Nuclide & Mass Fraction\\
\hline
\\
B10 & 1.843098e-06\\
B11 & 8.156930e-06\\
C12 & 2.965249e-04\\
N14 & 6.972393e-04\\
N15 & 2.786798e-06\\
Si28 & 4.593614e-03\\
Si29 & 2.415835e-04\\
Si30 & 1.647335e-04\\
P31 & 2.499975e-04\\
S & 1.000016e-04\\
Ti46 & 7.920085e-05\\
Ti47 & 7.297781e-05\\
Ti48 & 7.381413e-04\\
Ti49 & 5.532184e-05\\
Ti50 & 5.404879e-05\\
Cr50 & 7.303946e-03\\
Cr52 & 1.464747e-01\\
Cr53 & 1.692872e-02\\
Cr54 & 4.293349e-03\\
Mn55 & 1.799997e-02\\
Fe54 & 3.660823e-02\\
Fe56 & 5.959297e-01\\
Fe57 & 1.400876e-02\\
Fe58 & 1.896969e-03\\
Co59 & 4.999999e-04\\
Ni58 & 8.231680e-02\\
Ni60 & 3.280046e-02\\
Ni61 & 1.449607e-03\\
Ni62 & 4.697640e-03\\
Ni64 & 1.234979e-03\\
Cu63 & 2.054981e-03\\
Cu65 & 9.450179e-04\\
Nb93 & 1.000001e-04\\
Mo92 & 3.554367e-03\\
Mo94 & 2.263664e-03\\
Mo95 & 3.937466e-03\\
Mo96 & 4.168845e-03\\
Mo97 & 2.411755e-03\\
Mo98 & 6.156638e-03\\
Mo100 & 2.507290e-03\\
Ta181 & 9.999969e-05\\
\caption{Table showing the isotopic description of material EppMix}
\label{table:material_EppMix}
\end{longtable}\clearpage

\begin{longtable}[ht!]
{ p{0.3\textwidth} | p{0.3\textwidth} }
\hline
Nuclide & Mass Fraction\\
\hline
H1 & 8.251180e-04\\
H2 & 1.896534e-07\\
B10 & 3.659067e-06\\
B11 & 1.619381e-05\\
C12 & 2.943438e-04\\
N14 & 6.921097e-04\\
N15 & 2.766290e-06\\
O16 & 6.532444e-03\\
Si28 & 4.559809e-03\\
Si29 & 2.398056e-04\\
Si30 & 1.635213e-04\\
P31 & 2.481576e-04\\
S & 9.926556e-05\\
Ti46 & 7.861807e-05\\
Ti47 & 7.244078e-05\\
Ti48 & 7.327096e-04\\
Ti49 & 5.491474e-05\\
Ti50 & 5.365103e-05\\
Cr50 & 7.250200e-03\\
Cr52 & 1.453973e-01\\
Cr53 & 1.680419e-02\\
Cr54 & 4.261755e-03\\
Mn55 & 1.786755e-02\\
Fe54 & 3.628794e-02\\
Fe56 & 5.907146e-01\\
Fe57 & 1.388617e-02\\
Fe58 & 1.880371e-03\\
Co59 & 4.963209e-04\\
Ni58 & 8.171106e-02\\
Ni60 & 3.255910e-02\\
Ni61 & 1.438946e-03\\
Ni62 & 4.663070e-03\\
Ni64 & 1.225892e-03\\
Cu63 & 2.039858e-03\\
Cu65 & 9.380624e-04\\
Nb93 & 9.926428e-04\\
Mo92 & 3.528196e-03\\
Mo94 & 2.246995e-03\\
Mo95 & 3.908482e-03\\
Mo96 & 4.138165e-03\\
Mo97 & 2.394004e-03\\
Mo98 & 6.111329e-03\\
Mo100 & 2.488842e-03\\
Ta181 & 9.926390e-05\\
caption{Table showing the isotopic description of material M601}
\label{table:material_M601}
\end{longtable}\clearpage

\begin{longtable}[ht!]
{ p{0.3\textwidth} | p{0.3\textwidth} }
\hline
Nuclide & Mass Fraction\\
\hline
H1 & 3.243870e-03\\
H2 & 7.456037e-07\\
B10 & 1.789785e-06\\
B11 & 7.920970e-06\\
O16 & 2.599157e-02\\
Mg & 3.884290e-04\\
Al27 & 2.913214e-05\\
Si28 & 3.568587e-04\\
Si29 & 1.876763e-05\\
Si30 & 1.279748e-05\\
P31 & 1.359486e-04\\
S & 3.884363e-05\\
Cr50 & 3.039717e-04\\
Cr52 & 6.095905e-03\\
Cr53 & 7.045307e-04\\
Cr54 & 1.786781e-04\\
Mn55 & 1.942150e-05\\
Fe54 & 1.096452e-05\\
Fe56 & 1.784864e-04\\
Fe57 & 4.195764e-06\\
Fe58 & 5.681615e-07\\
Co59 & 4.855377e-04\\
Ni58 & 1.957615e-04\\
Ni60 & 7.800408e-05\\
Ni61 & 3.447379e-06\\
Ni62 & 1.117164e-05\\
Ni64 & 2.936951e-06\\
Cu63 & 6.576720e-01\\
Cu65 & 3.024406e-01\\
Zr & 1.068063e-03\\
Sn & 9.710766e-05\\
Ta181 & 9.710698e-05\\
Pb206 & 2.326315e-05\\
Pb207 & 2.143558e-05\\
Pb208 & 5.107068e-05\\
Bi209 & 2.913224e-05\\

\caption{Table showing the isotopic description of material M603}
\label{table:material_M603}
\end{longtable}\clearpage

\begin{longtable}[ht!]
{ p{0.3\textwidth} | p{0.3\textwidth} }
\hline
Nuclide & Mass Fraction\\
\hline
H1 & 3.090758e-02\\
H2 & 7.104097e-06\\
B10 & 2.670245e-06\\
B11 & 1.181760e-05\\
C12 & 2.147996e-04\\
N14 & 5.050741e-04\\
N15 & 2.018727e-06\\
O16 & 2.446947e-01\\
Si28 & 3.327574e-03\\
Si29 & 1.750009e-04\\
Si30 & 1.193315e-04\\
P31 & 1.810957e-04\\
S & 7.244026e-05\\
Ti46 & 5.737237e-05\\
Ti47 & 5.286433e-05\\
Ti48 & 5.347011e-04\\
Ti49 & 4.007472e-05\\
Ti50 & 3.915229e-05\\
Cr50 & 5.290908e-03\\
Cr52 & 1.061050e-01\\
Cr53 & 1.226301e-02\\
Cr54 & 3.110062e-03\\
Mn55 & 1.303902e-02\\
Fe54 & 2.648149e-02\\
Fe56 & 4.310799e-01\\
Fe57 & 1.013360e-02\\
Fe58 & 1.372222e-03\\
Co59 & 3.621953e-04\\
Ni58 & 5.962947e-02\\
Ni60 & 2.376035e-02\\
Ni61 & 1.050081e-03\\
Ni62 & 3.402923e-03\\
Ni64 & 8.946064e-04\\
Cu63 & 1.488611e-03\\
Cu65 & 6.845615e-04\\
Nb93 & 7.243894e-04\\
Mo92 & 2.574742e-03\\
Mo94 & 1.639770e-03\\
Mo95 & 2.852257e-03\\
Mo96 & 3.019871e-03\\
Mo97 & 1.747054e-03\\
Mo98 & 4.459812e-03\\
Mo100 & 1.816256e-03\\
Ta181 & 7.243891e-05\\

\caption{Table showing the isotopic description of material M602}
\label{table:material_M602}
\end{longtable}\clearpage

\begin{longtable}[ht!]
{ p{0.3\textwidth} | p{0.3\textwidth} }
\hline
Nuclide & Mass Fraction\\
\hline
\\
Al27 & 9.749971e-02\\
Si28 & 1.837444e-03\\
Si29 & 9.663327e-05\\
Si30 & 6.589366e-05\\
Mn55 & 9.999996e-03\\
Fe54 & 2.258229e-03\\
Fe56 & 3.676065e-02\\
Fe57 & 8.641477e-04\\
Fe58 & 1.170170e-04\\
Co59 & 5.000005e-04\\
Ni58 & 3.359869e-02\\
Ni60 & 1.338796e-02\\
Ni61 & 5.916763e-04\\
Ni62 & 1.917401e-03\\
Ni64 & 5.040734e-04\\
Cu63 & 5.461480e-01\\
Cu65 & 2.511552e-01\\
Nb93 & 9.999996e-04\\
Sn & 1.000003e-03\\
Ta181 & 4.999996e-04\\
Pb206 & 4.791233e-05\\
Pb207 & 4.414810e-05\\
Pb208 & 1.051841e-04\\

\caption{Table showing the isotopic description of material EPPDRW}
\label{table:material_EPPDRW}
\end{longtable}\clearpage

\begin{longtable}[ht!]
{ p{0.3\textwidth} | p{0.3\textwidth} }
\hline
Nuclide & Mass Fraction\\
\hline
\\
B10 & 1.843098e-06\\
B11 & 8.156930e-06\\
C12 & 2.965249e-04\\
N14 & 6.972393e-04\\
N15 & 2.786798e-06\\
Si28 & 4.593614e-03\\
Si29 & 2.415835e-04\\
Si30 & 1.647335e-04\\
P31 & 2.499975e-04\\
S & 1.000016e-04\\
Ti46 & 7.920085e-05\\
Ti47 & 7.297781e-05\\
Ti48 & 7.381413e-04\\
Ti49 & 5.532184e-05\\
Ti50 & 5.404879e-05\\
Cr50 & 7.303946e-03\\
Cr52 & 1.464747e-01\\
Cr53 & 1.692872e-02\\
Cr54 & 4.293349e-03\\
Mn55 & 1.799997e-02\\
Fe54 & 3.660823e-02\\
Fe56 & 5.959297e-01\\
Fe57 & 1.400876e-02\\
Fe58 & 1.896969e-03\\
Co59 & 4.999999e-04\\
Ni58 & 8.231680e-02\\
Ni60 & 3.280046e-02\\
Ni61 & 1.449607e-03\\
Ni62 & 4.697640e-03\\
Ni64 & 1.234979e-03\\
Cu63 & 2.054981e-03\\
Cu65 & 9.450179e-04\\
Nb93 & 1.000001e-04\\
Mo92 & 3.554367e-03\\
Mo94 & 2.263664e-03\\
Mo95 & 3.937466e-03\\
Mo96 & 4.168845e-03\\
Mo97 & 2.411755e-03\\
Mo98 & 6.156638e-03\\
Mo100 & 2.507290e-03\\
Ta181 & 9.999969e-05\\

\caption{Table showing the isotopic description of material EPPFW}
\label{table:material_EPPFW}
\end{longtable}\clearpage

\begin{longtable}[ht!]
{ p{0.3\textwidth} | p{0.3\textwidth} }
\hline
Nuclide & Mass Fraction\\
\hline
\\
O16 & 4.986451e-06\\
P31 & 2.999968e-06\\
S & 1.500027e-05\\
Mn55 & 5.000003e-07\\
Fe54 & 5.645562e-07\\
Fe56 & 9.190159e-06\\
Fe57 & 2.160371e-07\\
Fe58 & 2.925431e-08\\
Ni58 & 6.719746e-06\\
Ni60 & 2.677586e-06\\
Ni61 & 1.183357e-07\\
Ni62 & 3.834805e-07\\
Ni64 & 1.008146e-07\\
Cu63 & 6.849595e-01\\
Cu65 & 3.149890e-01\\
Sn & 2.000007e-06\\
Pb206 & 1.197804e-06\\
Pb207 & 1.103704e-06\\
Pb208 & 2.629607e-06\\
Bi209 & 1.000000e-06\\

\caption{Table showing the isotopic description of material M29}
\label{table:material_M29}
\end{longtable}\clearpage

\begin{longtable}[ht!]
{ p{0.3\textwidth} | p{0.3\textwidth} }
\hline
Nuclide & Mass Fraction\\
\hline
\\
B10 & 1.442461e-01\\
B11 & 6.383841e-01\\
C12 & 2.148517e-01\\
C13 & 2.518076e-03\\

\caption{Table showing the isotopic description of material UppExShield}
\label{table:material_UppExShield}
\end{longtable}\clearpage

\begin{longtable}[ht!]
{ p{0.3\textwidth} | p{0.3\textwidth} }
\hline
Nuclide & Mass Fraction\\
\hline
\\
B10 & 1.843098e-06\\
B11 & 8.156930e-06\\
C12 & 2.965249e-04\\
N14 & 6.972393e-04\\
N15 & 2.786798e-06\\
Si28 & 4.593614e-03\\
Si29 & 2.415835e-04\\
Si30 & 1.647335e-04\\
P31 & 2.499975e-04\\
S & 1.000016e-04\\
Ti46 & 7.920085e-05\\
Ti47 & 7.297781e-05\\
Ti48 & 7.381413e-04\\
Ti49 & 5.532184e-05\\
Ti50 & 5.404879e-05\\
Cr50 & 7.303946e-03\\
Cr52 & 1.464747e-01\\
Cr53 & 1.692872e-02\\
Cr54 & 4.293349e-03\\
Mn55 & 1.799997e-02\\
Fe54 & 3.660823e-02\\
Fe56 & 5.959297e-01\\
Fe57 & 1.400876e-02\\
Fe58 & 1.896969e-03\\
Co59 & 4.999999e-04\\
Ni58 & 8.231680e-02\\
Ni60 & 3.280046e-02\\
Ni61 & 1.449607e-03\\
Ni62 & 4.697640e-03\\
Ni64 & 1.234979e-03\\
Cu63 & 2.054981e-03\\
Cu65 & 9.450179e-04\\
Nb93 & 1.000001e-04\\
Mo92 & 3.554367e-03\\
Mo94 & 2.263664e-03\\
Mo95 & 3.937466e-03\\
Mo96 & 4.168845e-03\\
Mo97 & 2.411755e-03\\
Mo98 & 6.156638e-03\\
Mo100 & 2.507290e-03\\
Ta181 & 9.999969e-05\\

\caption{Table showing the isotopic description of material Cryopump}
\label{table:material_Cryopump}
\end{longtable}\clearpage

\begin{longtable}[ht!]
{ p{0.3\textwidth} | p{0.3\textwidth} }
\hline
Nuclide & Mass Fraction\\
\hline
\\
B10 & 1.442742e-01\\
B11 & 6.385081e-01\\
C12 & 2.172177e-01\\

\caption{Table showing the isotopic description of material EPPCN}
\label{table:material_EPPCN}
\end{longtable}\clearpage

\begin{longtable}[ht!]
{ p{0.3\textwidth} | p{0.3\textwidth} }
\hline
Nuclide & Mass Fraction\\
\hline
\\
B10 & 1.843098e-06\\
B11 & 8.156930e-06\\
C12 & 2.965249e-04\\
N14 & 6.972393e-04\\
N15 & 2.786798e-06\\
Si28 & 4.593614e-03\\
Si29 & 2.415835e-04\\
Si30 & 1.647335e-04\\
P31 & 2.499975e-04\\
S & 1.000016e-04\\
Ti46 & 7.920085e-05\\
Ti47 & 7.297781e-05\\
Ti48 & 7.381413e-04\\
Ti49 & 5.532184e-05\\
Ti50 & 5.404879e-05\\
Cr50 & 7.303946e-03\\
Cr52 & 1.464747e-01\\
Cr53 & 1.692872e-02\\
Cr54 & 4.293349e-03\\
Mn55 & 1.799997e-02\\
Fe54 & 3.660823e-02\\
Fe56 & 5.959297e-01\\
Fe57 & 1.400876e-02\\
Fe58 & 1.896969e-03\\
Co59 & 4.999999e-04\\
Ni58 & 8.231680e-02\\
Ni60 & 3.280046e-02\\
Ni61 & 1.449607e-03\\
Ni62 & 4.697640e-03\\
Ni64 & 1.234979e-03\\
Cu63 & 2.054981e-03\\
Cu65 & 9.450179e-04\\
Nb93 & 1.000001e-04\\
Mo92 & 3.554367e-03\\
Mo94 & 2.263664e-03\\
Mo95 & 3.937466e-03\\
Mo96 & 4.168845e-03\\
Mo97 & 2.411755e-03\\
Mo98 & 6.156638e-03\\
Mo100 & 2.507290e-03\\
Ta181 & 9.999969e-05\\

\caption{Table showing the isotopic description of material EMH}
\label{table:material_EMH}
\end{longtable}\clearpage

\begin{longtable}[ht!]
{ p{0.3\textwidth} | p{0.3\textwidth} }
\hline
Nuclide & Mass Fraction\\
\hline
H1 & 1.466991e-03\\
H2 & 3.371874e-07\\
B10 & 1.818988e-06\\
B11 & 8.050226e-06\\
C12 & 2.926459e-04\\
N14 & 6.881184e-04\\
N15 & 2.750343e-06\\
O16 & 1.161415e-02\\
Si28 & 4.533523e-03\\
Si29 & 2.384233e-04\\
Si30 & 1.625786e-04\\
P31 & 2.467271e-04\\
S & 9.869340e-05\\
Ti46 & 7.816479e-05\\
Ti47 & 7.202315e-05\\
Ti48 & 7.284853e-04\\
Ti49 & 5.459815e-05\\
Ti50 & 5.334176e-05\\
Cr50 & 7.208399e-03\\
Cr52 & 1.445586e-01\\
Cr53 & 1.670727e-02\\
Cr54 & 4.237186e-03\\
Mn55 & 1.776451e-02\\
Fe54 & 3.612934e-02\\
Fe56 & 5.881340e-01\\
Fe57 & 1.382551e-02\\
Fe58 & 1.872153e-03\\
Co59 & 4.934592e-04\\
Ni58 & 8.123998e-02\\
Ni60 & 3.237138e-02\\
Ni61 & 1.430644e-03\\
Ni62 & 4.636188e-03\\
Ni64 & 1.218823e-03\\
Cu63 & 2.028099e-03\\
Cu65 & 9.326557e-04\\
Nb93 & 9.869198e-05\\
Mo92 & 3.507870e-03\\
Mo94 & 2.234052e-03\\
Mo95 & 3.885958e-03\\
Mo96 & 4.114311e-03\\
Mo97 & 2.380205e-03\\
Mo98 & 6.076101e-03\\
Mo100 & 2.474491e-03\\
Ta181 & 9.869155e-05\\

\caption{Table showing the isotopic description of material UPDFW3}
\label{table:material_UPDFW3}
\end{longtable}\clearpage

\begin{longtable}[ht!]
{ p{0.3\textwidth} | p{0.3\textwidth} }
\hline
Nuclide & Mass Fraction\\
\hline
H1 & 4.381908e-03\\
H2 & 1.313805e-06\\
Be9 & 4.092101e-03\\
B10 & 1.870951e-06\\
B11 & 7.528208e-06\\
C12 & 2.092166e-04\\
N14 & 6.555165e-04\\
N15 & 2.445979e-06\\
O16 & 3.480110e-02\\
Al27 & 4.699541e-04\\
Si28 & 4.335576e-03\\
Si29 & 2.204930e-04\\
Si30 & 1.452762e-04\\
P31 & 2.349772e-04\\
S & 7.049782e-05\\
K & 4.695553e-06\\
Ti46 & 1.163148e-04\\
Ti47 & 1.048938e-04\\
Ti48 & 1.038917e-03\\
Ti49 & 7.627570e-05\\
Ti50 & 7.303434e-05\\
V & 3.758840e-05\\
Cr50 & 7.152128e-03\\
Cr52 & 1.379272e-01\\
Cr53 & 1.563985e-02\\
Cr54 & 3.893127e-03\\
Mn55 & 1.691837e-02\\
Fe54 & 3.567567e-02\\
Fe56 & 5.600298e-01\\
Fe57 & 1.293363e-02\\
Fe58 & 1.721210e-03\\
Co59 & 4.800693e-04\\
Ni58 & 7.838275e-02\\
Ni60 & 3.019306e-02\\
Ni61 & 1.312469e-03\\
Ni62 & 4.184752e-03\\
Ni64 & 1.065722e-03\\
Cu63 & 1.209131e-02\\
Cu65 & 5.551184e-03\\
Zr & 3.734351e-05\\
Nb93 & 9.399324e-05\\
Mo92 & 3.487083e-03\\
Mo94 & 2.173545e-03\\
Mo95 & 3.740855e-03\\
Mo96 & 3.919447e-03\\
Mo97 & 2.244056e-03\\
Mo98 & 5.670028e-03\\
Mo100 & 2.262832e-03\\
Sn & 1.880066e-05\\
Ta181 & 9.399116e-05\\
W182 & 2.496688e-06\\
W183 & 1.329347e-06\\
W184 & 2.855665e-06\\
W186 & 2.694326e-06\\
Pb206 & 1.812694e-06\\
Pb207 & 1.662714e-06\\
Pb208 & 3.931529e-06\\
Bi209 & 7.516923e-06\\

\caption{Table showing the isotopic description of material FirstWall}
\label{table:material_FirstWall}
\end{longtable}\clearpage

\begin{longtable}[ht!]
{ p{0.3\textwidth} | p{0.3\textwidth} }
\hline
Nuclide & Mass Fraction\\
\hline
\\
B10 & 1.843098e-06\\
B11 & 8.156930e-06\\
C12 & 2.965249e-04\\
N14 & 6.972393e-04\\
N15 & 2.786798e-06\\
Si28 & 4.593614e-03\\
Si29 & 2.415835e-04\\
Si30 & 1.647335e-04\\
P31 & 2.499975e-04\\
S & 1.000016e-04\\
Ti46 & 7.920085e-05\\
Ti47 & 7.297781e-05\\
Ti48 & 7.381413e-04\\
Ti49 & 5.532184e-05\\
Ti50 & 5.404879e-05\\
Cr50 & 7.303946e-03\\
Cr52 & 1.464747e-01\\
Cr53 & 1.692872e-02\\
Cr54 & 4.293349e-03\\
Mn55 & 1.799997e-02\\
Fe54 & 3.660823e-02\\
Fe56 & 5.959297e-01\\
Fe57 & 1.400876e-02\\
Fe58 & 1.896969e-03\\
Co59 & 4.999999e-04\\
Ni58 & 8.231680e-02\\
Ni60 & 3.280046e-02\\
Ni61 & 1.449607e-03\\
Ni62 & 4.697640e-03\\
Ni64 & 1.234979e-03\\
Cu63 & 2.054981e-03\\
Cu65 & 9.450179e-04\\
Nb93 & 1.000001e-04\\
Mo92 & 3.554367e-03\\
Mo94 & 2.263664e-03\\
Mo95 & 3.937466e-03\\
Mo96 & 4.168845e-03\\
Mo97 & 2.411755e-03\\
Mo98 & 6.156638e-03\\
Mo100 & 2.507290e-03\\
Ta181 & 9.999969e-05\\

\caption{Table showing the isotopic description of material UppWater}
\label{table:material_UppWater}
\end{longtable}\clearpage

\begin{longtable}[ht!]
{ p{0.3\textwidth} | p{0.3\textwidth} }
\hline
Nuclide & Mass Fraction\\
\hline
\\
B10 & 1.843098e-06\\
B11 & 8.156930e-06\\
C12 & 2.965249e-04\\
N14 & 6.972393e-04\\
N15 & 2.786798e-06\\
Si28 & 4.593614e-03\\
Si29 & 2.415835e-04\\
Si30 & 1.647335e-04\\
P31 & 2.499975e-04\\
S & 1.000016e-04\\
Ti46 & 7.920085e-05\\
Ti47 & 7.297781e-05\\
Ti48 & 7.381413e-04\\
Ti49 & 5.532184e-05\\
Ti50 & 5.404879e-05\\
Cr50 & 7.303946e-03\\
Cr52 & 1.464747e-01\\
Cr53 & 1.692872e-02\\
Cr54 & 4.293349e-03\\
Mn55 & 1.799997e-02\\
Fe54 & 3.660823e-02\\
Fe56 & 5.959297e-01\\
Fe57 & 1.400876e-02\\
Fe58 & 1.896969e-03\\
Co59 & 4.999999e-04\\
Ni58 & 8.231680e-02\\
Ni60 & 3.280046e-02\\
Ni61 & 1.449607e-03\\
Ni62 & 4.697640e-03\\
Ni64 & 1.234979e-03\\
Cu63 & 2.054981e-03\\
Cu65 & 9.450179e-04\\
Nb93 & 1.000001e-04\\
Mo92 & 3.554367e-03\\
Mo94 & 2.263664e-03\\
Mo95 & 3.937466e-03\\
Mo96 & 4.168845e-03\\
Mo97 & 2.411755e-03\\
Mo98 & 6.156638e-03\\
Mo100 & 2.507290e-03\\
Ta181 & 9.999969e-05\\

\caption{Table showing the isotopic description of material Lenses}
\label{table:material_Lenses}
\end{longtable}\clearpage

\begin{longtable}[ht!]
{ p{0.3\textwidth} | p{0.3\textwidth} }
\hline
Nuclide & Mass Fraction\\
\hline
\\
C12 & 2.965246e-04\\
N14 & 1.693295e-03\\
N15 & 6.767941e-06\\
Si28 & 9.187241e-03\\
Si29 & 4.831665e-04\\
Si30 & 3.294673e-04\\
P31 & 4.499946e-04\\
S & 1.500025e-04\\
Cr50 & 7.721311e-03\\
Cr52 & 1.548449e-01\\
Cr53 & 1.789604e-02\\
Cr54 & 4.538678e-03\\
Mn55 & 2.000003e-02\\
Fe54 & 3.841255e-02\\
Fe56 & 6.253010e-01\\
Fe57 & 1.469916e-02\\
Fe58 & 1.990474e-03\\
Co59 & 9.999998e-04\\
Ni58 & 6.719730e-02\\
Ni60 & 2.677580e-02\\
Ni61 & 1.183353e-03\\
Ni62 & 3.834799e-03\\
Ni64 & 1.008145e-03\\
Nb93 & 1.000000e-03\\

\caption{Table showing the isotopic description of material M102}
\label{table:material_M102}
\end{longtable}\clearpage

\begin{longtable}[ht!]
{ p{0.3\textwidth} | p{0.3\textwidth} }
\hline
Nuclide & Mass Fraction\\
\hline
\\
B10 & 3.317568e-06\\
B11 & 1.468247e-05\\
C12 & 2.965247e-04\\
N14 & 9.960588e-04\\
N15 & 3.981134e-06\\
Si28 & 8.038815e-03\\
Si29 & 4.227711e-04\\
Si30 & 2.882844e-04\\
P31 & 2.999967e-04\\
S & 1.500025e-04\\
Cr50 & 7.930007e-03\\
Cr52 & 1.590294e-01\\
Cr53 & 1.837976e-02\\
Cr54 & 4.661352e-03\\
Mn55 & 2.000002e-02\\
Fe54 & 3.853009e-02\\
Fe56 & 6.272146e-01\\
Fe57 & 1.474418e-02\\
Fe58 & 1.996561e-03\\
Co59 & 1.000000e-03\\
Ni58 & 6.383756e-02\\
Ni60 & 2.543704e-02\\
Ni61 & 1.124186e-03\\
Ni62 & 3.643066e-03\\
Ni64 & 9.577390e-04\\
Nb93 & 1.000001e-03\\

\caption{Table showing the isotopic description of material M103}
\label{table:material_M103}
\end{longtable}\clearpage

\begin{longtable}[ht!]
{ p{0.3\textwidth} | p{0.3\textwidth} }
\hline
Nuclide & Mass Fraction\\
\hline
\\
B10 & 1.843098e-06\\
B11 & 8.156930e-06\\
C12 & 2.965249e-04\\
N14 & 6.972393e-04\\
N15 & 2.786798e-06\\
Si28 & 4.593614e-03\\
Si29 & 2.415835e-04\\
Si30 & 1.647335e-04\\
P31 & 2.499975e-04\\
S & 1.000016e-04\\
Ti46 & 7.920085e-05\\
Ti47 & 7.297781e-05\\
Ti48 & 7.381413e-04\\
Ti49 & 5.532184e-05\\
Ti50 & 5.404879e-05\\
Cr50 & 7.303946e-03\\
Cr52 & 1.464747e-01\\
Cr53 & 1.692872e-02\\
Cr54 & 4.293349e-03\\
Mn55 & 1.799997e-02\\
Fe54 & 3.660823e-02\\
Fe56 & 5.959297e-01\\
Fe57 & 1.400876e-02\\
Fe58 & 1.896969e-03\\
Co59 & 4.999999e-04\\
Ni58 & 8.231680e-02\\
Ni60 & 3.280046e-02\\
Ni61 & 1.449607e-03\\
Ni62 & 4.697640e-03\\
Ni64 & 1.234979e-03\\
Cu63 & 2.054981e-03\\
Cu65 & 9.450179e-04\\
Nb93 & 1.000001e-04\\
Mo92 & 3.554367e-03\\
Mo94 & 2.263664e-03\\
Mo95 & 3.937466e-03\\
Mo96 & 4.168845e-03\\
Mo97 & 2.411755e-03\\
Mo98 & 6.156638e-03\\
Mo100 & 2.507290e-03\\
Ta181 & 9.999969e-05\\

\caption{Table showing the isotopic description of material M100}
\label{table:material_M100}
\end{longtable}\clearpage

\begin{longtable}[ht!]
{ p{0.3\textwidth} | p{0.3\textwidth} }
\hline
Nuclide & Mass Fraction\\
\hline
\\
B10 & 3.686196e-06\\
B11 & 1.631384e-05\\
C12 & 2.965249e-04\\
N14 & 6.972393e-04\\
N15 & 2.786798e-06\\
Si28 & 4.593614e-03\\
Si29 & 2.415835e-04\\
Si30 & 1.647335e-04\\
P31 & 2.499974e-04\\
S & 1.000016e-04\\
Ti46 & 7.920084e-05\\
Ti47 & 7.297780e-05\\
Ti48 & 7.381412e-04\\
Ti49 & 5.532184e-05\\
Ti50 & 5.404879e-05\\
Cr50 & 7.303945e-03\\
Cr52 & 1.464747e-01\\
Cr53 & 1.692872e-02\\
Cr54 & 4.293349e-03\\
Mn55 & 1.799997e-02\\
Fe54 & 3.655695e-02\\
Fe56 & 5.950933e-01\\
Fe57 & 1.398909e-02\\
Fe58 & 1.894312e-03\\
Co59 & 4.999998e-04\\
Ni58 & 8.231679e-02\\
Ni60 & 3.280045e-02\\
Ni61 & 1.449607e-03\\
Ni62 & 4.697639e-03\\
Ni64 & 1.234979e-03\\
Cu63 & 2.054981e-03\\
Cu65 & 9.450178e-04\\
Nb93 & 1.000001e-03\\
Mo92 & 3.554366e-03\\
Mo94 & 2.263664e-03\\
Mo95 & 3.937466e-03\\
Mo96 & 4.168845e-03\\
Mo97 & 2.411754e-03\\
Mo98 & 6.156638e-03\\
Mo100 & 2.507290e-03\\
Ta181 & 9.999968e-05\\

\caption{Table showing the isotopic description of material M101}
\label{table:material_M101}
\end{longtable}\clearpage

\begin{longtable}[ht!]
{ p{0.3\textwidth} | p{0.3\textwidth} }
\hline
Nuclide & Mass Fraction\\
\hline
\\
B10 & 3.317571e-06\\
B11 & 1.468244e-05\\
C12 & 2.965246e-04\\
N14 & 1.095662e-03\\
N15 & 4.379264e-06\\
Si28 & 9.187242e-03\\
Si29 & 4.831665e-04\\
Si30 & 3.294673e-04\\
P31 & 2.999969e-04\\
S & 1.500025e-04\\
Ti46 & 7.920076e-05\\
Ti47 & 7.297772e-05\\
Ti48 & 7.381405e-04\\
Ti49 & 5.532178e-05\\
Ti50 & 5.404873e-05\\
Cr50 & 7.303938e-03\\
Cr52 & 1.464745e-01\\
Cr53 & 1.692870e-02\\
Cr54 & 4.293344e-03\\
Mn55 & 2.000004e-02\\
Fe54 & 3.613923e-02\\
Fe56 & 5.882946e-01\\
Fe57 & 1.382932e-02\\
Fe58 & 1.872668e-03\\
Co59 & 2.000000e-03\\
Ni58 & 7.727699e-02\\
Ni60 & 3.079226e-02\\
Ni61 & 1.360854e-03\\
Ni62 & 4.410023e-03\\
Ni64 & 1.159367e-03\\
Cu63 & 6.849938e-03\\
Cu65 & 3.150061e-03\\
Nb93 & 1.000000e-03\\
Mo92 & 3.198923e-03\\
Mo94 & 2.037293e-03\\
Mo95 & 3.543714e-03\\
Mo96 & 3.751951e-03\\
Mo97 & 2.170581e-03\\
Mo98 & 5.540977e-03\\
Mo100 & 2.256555e-03\\
Ta181 & 1.499994e-03\\

\caption{Table showing the isotopic description of material M106}
\label{table:material_M106}
\end{longtable}\clearpage

\begin{longtable}[ht!]
{ p{0.3\textwidth} | p{0.3\textwidth} }
\hline
Nuclide & Mass Fraction\\
\hline
\\
C12 & 5.930496e-04\\
N14 & 2.988169e-03\\
N15 & 1.194341e-05\\
Si28 & 9.187232e-03\\
Si29 & 4.831671e-04\\
Si30 & 3.294672e-04\\
P31 & 3.999954e-04\\
S & 3.000046e-04\\
V & 1.999904e-03\\
Cr50 & 9.182103e-03\\
Cr52 & 1.841398e-01\\
Cr53 & 2.128179e-02\\
Cr54 & 5.397353e-03\\
Mn55 & 4.999998e-02\\
Fe54 & 3.176235e-02\\
Fe56 & 5.170442e-01\\
Fe57 & 1.215434e-02\\
Fe58 & 1.645869e-03\\
Co59 & 5.000003e-04\\
Ni58 & 8.399666e-02\\
Ni60 & 3.346977e-02\\
Ni61 & 1.479193e-03\\
Ni62 & 4.793502e-03\\
Ni64 & 1.260181e-03\\
Nb93 & 2.999993e-03\\
Mo92 & 3.198924e-03\\
Mo94 & 2.037299e-03\\
Mo95 & 3.543701e-03\\
Mo96 & 3.751951e-03\\
Mo97 & 2.170590e-03\\
Mo98 & 5.540980e-03\\
Mo100 & 2.256557e-03\\
Ta181 & 9.999962e-05\\

\caption{Table showing the isotopic description of material M107}
\label{table:material_M107}
\end{longtable}\clearpage

\begin{longtable}[ht!]
{ p{0.3\textwidth} | p{0.3\textwidth} }
\hline
Nuclide & Mass Fraction\\
\hline
\\
B10 & 3.317572e-06\\
B11 & 1.468245e-05\\
C12 & 6.918901e-04\\
N14 & 1.095662e-03\\
N15 & 4.379265e-06\\
Si28 & 9.187245e-03\\
Si29 & 4.831667e-04\\
Si30 & 3.294674e-04\\
P31 & 2.999970e-04\\
S & 1.500026e-04\\
Ti46 & 7.920079e-05\\
Ti47 & 7.297775e-05\\
Ti48 & 7.381407e-04\\
Ti49 & 5.532180e-05\\
Ti50 & 5.404875e-05\\
Cr50 & 7.616968e-03\\
Cr52 & 1.527524e-01\\
Cr53 & 1.765427e-02\\
Cr54 & 4.477349e-03\\
Mn55 & 2.000004e-02\\
Fe54 & 3.811540e-02\\
Fe56 & 6.204639e-01\\
Fe57 & 1.458555e-02\\
Fe58 & 1.975068e-03\\
Co59 & 4.999995e-04\\
Ni58 & 6.215761e-02\\
Ni60 & 2.476764e-02\\
Ni61 & 1.094601e-03\\
Ni62 & 3.547188e-03\\
Ni64 & 9.325344e-04\\
Cu63 & 6.849941e-03\\
Cu65 & 3.150062e-03\\
Nb93 & 1.000000e-03\\
Mo92 & 7.108708e-04\\
Mo94 & 4.527305e-04\\
Mo95 & 7.874906e-04\\
Mo96 & 8.337684e-04\\
Mo97 & 4.823526e-04\\
Mo98 & 1.231327e-03\\
Mo100 & 5.014577e-04\\
Ta181 & 9.999961e-05\\

\caption{Table showing the isotopic description of material M104}
\label{table:material_M104}
\end{longtable}\clearpage

\begin{longtable}[ht!]
{ p{0.3\textwidth} | p{0.3\textwidth} }
\hline
Nuclide & Mass Fraction\\
\hline
\\
B10 & 3.317573e-06\\
B11 & 1.468245e-05\\
C12 & 2.965248e-04\\
N14 & 1.095662e-03\\
N15 & 4.379266e-06\\
Si28 & 9.187247e-03\\
Si29 & 4.831668e-04\\
Si30 & 3.294675e-04\\
P31 & 2.999971e-04\\
S & 1.500026e-04\\
Ti46 & 7.920081e-05\\
Ti47 & 7.297777e-05\\
Ti48 & 7.381409e-04\\
Ti49 & 5.532181e-05\\
Ti50 & 5.404877e-05\\
Cr50 & 7.721316e-03\\
Cr52 & 1.548450e-01\\
Cr53 & 1.789605e-02\\
Cr54 & 4.538681e-03\\
Mn55 & 2.000005e-02\\
Fe54 & 3.813777e-02\\
Fe56 & 6.208272e-01\\
Fe57 & 1.459402e-02\\
Fe58 & 1.976233e-03\\
Co59 & 4.999996e-04\\
Ni58 & 6.047764e-02\\
Ni60 & 2.409826e-02\\
Ni61 & 1.065019e-03\\
Ni62 & 3.451319e-03\\
Ni64 & 9.073305e-04\\
Cu63 & 6.849942e-03\\
Cu65 & 3.150062e-03\\
Nb93 & 1.000001e-03\\
Mo92 & 7.108710e-04\\
Mo94 & 4.527306e-04\\
Mo95 & 7.874908e-04\\
Mo96 & 8.337687e-04\\
Mo97 & 4.823527e-04\\
Mo98 & 1.231327e-03\\
Mo100 & 5.014578e-04\\
Ta181 & 9.999964e-05\\

\caption{Table showing the isotopic description of material M105}
\label{table:material_M105}
\end{longtable}\clearpage

\begin{longtable}[ht!]
{ p{0.3\textwidth} | p{0.3\textwidth} }
\hline
Nuclide & Mass Fraction\\
\hline
H1 & 6.997078e-04\\
H2 & 1.608276e-07\\
B10 & 1.831598e-06\\
B11 & 8.106036e-06\\
C12 & 2.946747e-04\\
N14 & 6.928888e-04\\
N15 & 2.769410e-06\\
O16 & 5.539577e-03\\
Si28 & 4.564952e-03\\
Si29 & 2.400762e-04\\
Si30 & 1.637057e-04\\
P31 & 2.484376e-04\\
S & 9.937764e-05\\
Ti46 & 7.870667e-05\\
Ti47 & 7.252245e-05\\
Ti48 & 7.335356e-04\\
Ti49 & 5.497666e-05\\
Ti50 & 5.371156e-05\\
Cr50 & 7.258372e-03\\
Cr52 & 1.455608e-01\\
Cr53 & 1.682310e-02\\
Cr54 & 4.266560e-03\\
Mn55 & 1.788766e-02\\
Fe54 & 3.637981e-02\\
Fe56 & 5.922114e-01\\
Fe57 & 1.392136e-02\\
Fe58 & 1.885133e-03\\
Co59 & 4.968802e-04\\
Ni58 & 8.180318e-02\\
Ni60 & 3.259580e-02\\
Ni61 & 1.440563e-03\\
Ni62 & 4.668329e-03\\
Ni64 & 1.227273e-03\\
Cu63 & 2.042160e-03\\
Cu65 & 9.391214e-04\\
Nb93 & 9.937617e-05\\
Mo92 & 3.532189e-03\\
Mo94 & 2.249541e-03\\
Mo95 & 3.912899e-03\\
Mo96 & 4.142835e-03\\
Mo97 & 2.396706e-03\\
Mo98 & 6.118223e-03\\
Mo100 & 2.491646e-03\\
Ta181 & 9.937573e-05\\

\caption{Table showing the isotopic description of material EPPBDY}
\label{table:material_EPPBDY}
\end{longtable}\clearpage

\begin{longtable}[ht!]
{ p{0.3\textwidth} | p{0.3\textwidth} }
\hline
Nuclide & Mass Fraction\\
\hline
\\
B10 & 1.843096e-06\\
B11 & 8.156922e-06\\
C12 & 2.965246e-04\\
N14 & 1.095662e-03\\
N15 & 4.379264e-06\\
Si28 & 9.187242e-03\\
Si29 & 4.831666e-04\\
Si30 & 3.294673e-04\\
P31 & 3.999957e-04\\
S & 1.500025e-04\\
Cr50 & 7.303938e-03\\
Cr52 & 1.464745e-01\\
Cr53 & 1.692870e-02\\
Cr54 & 4.293345e-03\\
Mn55 & 2.000004e-02\\
Fe54 & 3.681150e-02\\
Fe56 & 5.992379e-01\\
Fe57 & 1.408660e-02\\
Fe58 & 1.907509e-03\\
Co59 & 2.000000e-03\\
Ni58 & 7.727700e-02\\
Ni60 & 3.079226e-02\\
Ni61 & 1.360854e-03\\
Ni62 & 4.410023e-03\\
Ni64 & 1.159367e-03\\
Nb93 & 1.000000e-03\\
Mo92 & 3.198923e-03\\
Mo94 & 2.037294e-03\\
Mo95 & 3.543714e-03\\
Mo96 & 3.751951e-03\\
Mo97 & 2.170581e-03\\
Mo98 & 5.540977e-03\\
Mo100 & 2.256555e-03\\
Ta181 & 4.999979e-04\\

\caption{Table showing the isotopic description of material M108}
\label{table:material_M108}
\end{longtable}\clearpage

\begin{longtable}[ht!]
{ p{0.3\textwidth} | p{0.3\textwidth} }
\hline
Nuclide & Mass Fraction\\
\hline
\\
B10 & 1.843112e-05\\
B11 & 8.156993e-05\\
C12 & 7.907392e-04\\
Al27 & 3.500018e-03\\
Si28 & 9.187303e-03\\
Si29 & 4.831706e-04\\
Si30 & 3.294700e-04\\
P31 & 3.999987e-04\\
S & 3.000072e-04\\
Ti46 & 1.683034e-03\\
Ti47 & 1.550790e-03\\
Ti48 & 1.568564e-02\\
Ti49 & 1.175600e-03\\
Ti50 & 1.148546e-03\\
V & 2.999879e-03\\
Cr50 & 6.156236e-03\\
Cr52 & 1.234582e-01\\
Cr53 & 1.426861e-02\\
Cr54 & 3.618704e-03\\
Mn55 & 2.000011e-02\\
Fe54 & 2.947910e-02\\
Fe56 & 4.798757e-01\\
Fe57 & 1.128065e-02\\
Fe58 & 1.527545e-03\\
Co59 & 2.000021e-03\\
Ni58 & 1.713547e-01\\
Ni60 & 6.827897e-02\\
Ni61 & 3.017577e-03\\
Ni62 & 9.778832e-03\\
Ni64 & 2.570789e-03\\
Nb93 & 1.000009e-03\\
Mo92 & 1.777191e-03\\
Mo94 & 1.131836e-03\\
Mo95 & 1.968744e-03\\
Mo96 & 2.084445e-03\\
Mo97 & 1.205889e-03\\
Mo98 & 3.078348e-03\\
Mo100 & 1.253653e-03\\
Ta181 & 5.000035e-04\\

\caption{Table showing the isotopic description of material M109}
\label{table:material_M109}
\end{longtable}\clearpage

\begin{longtable}[ht!]
{ p{0.3\textwidth} | p{0.3\textwidth} }
\hline
Nuclide & Mass Fraction\\
\hline
\\
B10 & 1.843098e-06\\
B11 & 8.156930e-06\\
C12 & 2.965249e-04\\
N14 & 6.972393e-04\\
N15 & 2.786798e-06\\
Si28 & 4.593614e-03\\
Si29 & 2.415835e-04\\
Si30 & 1.647335e-04\\
P31 & 2.499975e-04\\
S & 1.000016e-04\\
Ti46 & 7.920085e-05\\
Ti47 & 7.297781e-05\\
Ti48 & 7.381413e-04\\
Ti49 & 5.532184e-05\\
Ti50 & 5.404879e-05\\
Cr50 & 7.303946e-03\\
Cr52 & 1.464747e-01\\
Cr53 & 1.692872e-02\\
Cr54 & 4.293349e-03\\
Mn55 & 1.799997e-02\\
Fe54 & 3.660823e-02\\
Fe56 & 5.959297e-01\\
Fe57 & 1.400876e-02\\
Fe58 & 1.896969e-03\\
Co59 & 4.999999e-04\\
Ni58 & 8.231680e-02\\
Ni60 & 3.280046e-02\\
Ni61 & 1.449607e-03\\
Ni62 & 4.697640e-03\\
Ni64 & 1.234979e-03\\
Cu63 & 2.054981e-03\\
Cu65 & 9.450179e-04\\
Nb93 & 1.000001e-04\\
Mo92 & 3.554367e-03\\
Mo94 & 2.263664e-03\\
Mo95 & 3.937466e-03\\
Mo96 & 4.168845e-03\\
Mo97 & 2.411755e-03\\
Mo98 & 6.156638e-03\\
Mo100 & 2.507290e-03\\
Ta181 & 9.999969e-05\\

\caption{Table showing the isotopic description of material UppWaterPipes}
\label{table:material_UppWaterPipes}
\end{longtable}\clearpage

\begin{longtable}[ht!]
{ p{0.3\textwidth} | p{0.3\textwidth} }
\hline
Nuclide & Mass Fraction\\
\hline
\\
B10 & 1.843098e-06\\
B11 & 8.156930e-06\\
C12 & 2.965249e-04\\
N14 & 6.972393e-04\\
N15 & 2.786798e-06\\
Si28 & 4.593614e-03\\
Si29 & 2.415835e-04\\
Si30 & 1.647335e-04\\
P31 & 2.499975e-04\\
S & 1.000016e-04\\
Ti46 & 7.920085e-05\\
Ti47 & 7.297781e-05\\
Ti48 & 7.381413e-04\\
Ti49 & 5.532184e-05\\
Ti50 & 5.404879e-05\\
Cr50 & 7.303946e-03\\
Cr52 & 1.464747e-01\\
Cr53 & 1.692872e-02\\
Cr54 & 4.293349e-03\\
Mn55 & 1.799997e-02\\
Fe54 & 3.660823e-02\\
Fe56 & 5.959297e-01\\
Fe57 & 1.400876e-02\\
Fe58 & 1.896969e-03\\
Co59 & 4.999999e-04\\
Ni58 & 8.231680e-02\\
Ni60 & 3.280046e-02\\
Ni61 & 1.449607e-03\\
Ni62 & 4.697640e-03\\
Ni64 & 1.234979e-03\\
Cu63 & 2.054981e-03\\
Cu65 & 9.450179e-04\\
Nb93 & 1.000001e-04\\
Mo92 & 3.554367e-03\\
Mo94 & 2.263664e-03\\
Mo95 & 3.937466e-03\\
Mo96 & 4.168845e-03\\
Mo97 & 2.411755e-03\\
Mo98 & 6.156638e-03\\
Mo100 & 2.507290e-03\\
Ta181 & 9.999969e-05\\

\caption{Table showing the isotopic description of material EppValves}
\label{table:material_EppValves}
\end{longtable}\clearpage

\begin{longtable}[ht!]
{ p{0.3\textwidth} | p{0.3\textwidth} }
\hline
Nuclide & Mass Fraction\\
\hline
H1 & 3.312304e-03\\
H2 & 7.613329e-07\\
B10 & 1.788659e-06\\
B11 & 7.915993e-06\\
O16 & 2.653324e-02\\
Mg & 3.881843e-04\\
Al27 & 2.911380e-05\\
Si28 & 3.566347e-04\\
Si29 & 1.875587e-05\\
Si30 & 1.278942e-05\\
P31 & 1.358633e-04\\
S & 3.881918e-05\\
Cr50 & 3.037802e-04\\
Cr52 & 6.092066e-03\\
Cr53 & 7.040870e-04\\
Cr54 & 1.785663e-04\\
Mn55 & 1.940927e-05\\
Fe54 & 1.095762e-05\\
Fe56 & 1.783737e-04\\
Fe57 & 4.193123e-06\\
Fe58 & 5.678037e-07\\
Co59 & 4.852314e-04\\
Ni58 & 1.956384e-04\\
Ni60 & 7.795500e-05\\
Ni61 & 3.445201e-06\\
Ni62 & 1.116461e-05\\
Ni64 & 2.935106e-06\\
Cu63 & 6.572585e-01\\
Cu65 & 3.022507e-01\\
Zr & 1.067391e-03\\
Sn & 9.704643e-05\\
Ta181 & 9.704585e-05\\
Pb206 & 2.324848e-05\\
Pb207 & 2.142209e-05\\
Pb208 & 5.103877e-05\\
Bi209 & 2.911390e-05\\

\caption{Table showing the isotopic description of material M613}
\label{table:material_M613}
\end{longtable}\clearpage

\begin{longtable}[ht!]
{ p{0.3\textwidth} | p{0.3\textwidth} }
\hline
Nuclide & Mass Fraction\\
\hline
H1 & 7.158264e-04\\
H2 & 1.645325e-07\\
B10 & 3.662661e-06\\
B11 & 1.620970e-05\\
C12 & 2.946323e-04\\
N14 & 6.927907e-04\\
N15 & 2.769006e-06\\
O16 & 5.667191e-03\\
Si28 & 4.564291e-03\\
Si29 & 2.400416e-04\\
Si30 & 1.636824e-04\\
P31 & 2.484018e-04\\
S & 9.936296e-05\\
Ti46 & 7.869531e-05\\
Ti47 & 7.251184e-05\\
Ti48 & 7.334289e-04\\
Ti49 & 5.496867e-05\\
Ti50 & 5.370369e-05\\
Cr50 & 7.257323e-03\\
Cr52 & 1.455394e-01\\
Cr53 & 1.682069e-02\\
Cr54 & 4.265942e-03\\
Mn55 & 1.788512e-02\\
Fe54 & 3.632358e-02\\
Fe56 & 5.912951e-01\\
Fe57 & 1.389977e-02\\
Fe58 & 1.882224e-03\\
Co59 & 4.968081e-04\\
Ni58 & 8.179130e-02\\
Ni60 & 3.259106e-02\\
Ni61 & 1.440349e-03\\
Ni62 & 4.667655e-03\\
Ni64 & 1.227096e-03\\
Cu63 & 2.041861e-03\\
Cu65 & 9.389844e-04\\
Nb93 & 9.936162e-04\\
Mo92 & 3.531668e-03\\
Mo94 & 2.249205e-03\\
Mo95 & 3.912318e-03\\
Mo96 & 4.142226e-03\\
Mo97 & 2.396366e-03\\
Mo98 & 6.117348e-03\\
Mo100 & 2.491290e-03\\
Ta181 & 9.936140e-05\\

\caption{Table showing the isotopic description of material M611}
\label{table:material_M611}
\end{longtable}\clearpage

\begin{longtable}[ht!]
{ p{0.3\textwidth} | p{0.3\textwidth} }
\hline
Nuclide & Mass Fraction\\
\hline
H1 & 1.072682e-03\\
H2 & 2.465560e-07\\
B10 & 9.290169e-08\\
B11 & 4.111511e-07\\
C12 & 2.651897e-05\\
N14 & 8.908000e-06\\
N15 & 3.560435e-08\\
O16 & 8.500081e-03\\
Na23 & 8.943266e-06\\
Mg & 2.463364e-05\\
Al27 & 1.492702e-05\\
Si28 & 3.495610e-05\\
Si29 & 1.838382e-06\\
Si30 & 1.253580e-06\\
P31 & 5.191294e-05\\
S & 7.189985e-06\\
K & 8.942448e-06\\
Ca & 8.943683e-06\\
Ti46 & 7.083168e-07\\
Ti47 & 6.526604e-07\\
Ti48 & 6.601366e-06\\
Ti49 & 4.947581e-07\\
Ti50 & 4.833723e-07\\
Cr50 & 1.615143e-05\\
Cr52 & 3.239032e-04\\
Cr53 & 3.743494e-05\\
Cr54 & 9.493979e-06\\
Mn55 & 5.503136e-06\\
Fe54 & 2.110249e-06\\
Fe56 & 3.435175e-05\\
Fe57 & 8.075214e-07\\
Fe58 & 1.093490e-07\\
Co59 & 3.414588e-05\\
Ni58 & 2.249510e-05\\
Ni60 & 8.963523e-06\\
Ni61 & 3.961413e-07\\
Ni62 & 1.283744e-06\\
Ni64 & 3.374876e-07\\
Cu63 & 6.620143e-02\\
Cu65 & 3.044380e-02\\
Zr & 6.438187e-05\\
Nb93 & 8.943268e-06\\
Mo92 & 1.271505e-05\\
Mo94 & 8.097788e-06\\
Mo95 & 1.408550e-05\\
Mo96 & 1.491317e-05\\
Mo97 & 8.627601e-06\\
Mo98 & 2.202423e-05\\
Mo100 & 8.969334e-06\\
Sn & 5.134141e-06\\
Ta181 & 1.398374e-05\\
W182 & 2.344468e-01\\
W183 & 1.273014e-01\\
W184 & 2.740668e-01\\
W186 & 2.570633e-01\\
Pb206 & 3.406027e-06\\
Pb207 & 3.138446e-06\\
Pb208 & 7.477455e-06\\
Bi209 & 1.558958e-06\\

\caption{Table showing the isotopic description of material M631}
\label{table:material_M631}
\end{longtable}\clearpage

\begin{longtable}[ht!]
{ p{0.3\textwidth} | p{0.3\textwidth} }
\hline
Nuclide & Mass Fraction\\
\hline
H1 & 3.247625e-03\\
H2 & 7.464654e-07\\
B10 & 1.789722e-06\\
B11 & 7.920707e-06\\
C12 & 2.879376e-04\\
N14 & 6.770474e-04\\
N15 & 2.706093e-06\\
O16 & 2.571140e-02\\
Si28 & 4.460584e-03\\
Si29 & 2.345873e-04\\
Si30 & 1.599629e-04\\
P31 & 2.427576e-04\\
S & 9.710555e-05\\
Ti46 & 7.690721e-05\\
Ti47 & 7.086439e-05\\
Ti48 & 7.167649e-04\\
Ti49 & 5.371974e-05\\
Ti50 & 5.248355e-05\\
Cr50 & 7.092425e-03\\
Cr52 & 1.422328e-01\\
Cr53 & 1.643847e-02\\
Cr54 & 4.169015e-03\\
Mn55 & 1.747870e-02\\
Fe54 & 3.554807e-02\\
Fe56 & 5.786717e-01\\
Fe57 & 1.360307e-02\\
Fe58 & 1.842033e-03\\
Co59 & 4.855200e-04\\
Ni58 & 7.993293e-02\\
Ni60 & 3.185056e-02\\
Ni61 & 1.407627e-03\\
Ni62 & 4.561597e-03\\
Ni64 & 1.199214e-03\\
Cu63 & 1.995469e-03\\
Cu65 & 9.176504e-04\\
Nb93 & 9.710414e-05\\
Mo92 & 3.451433e-03\\
Mo94 & 2.198109e-03\\
Mo95 & 3.823438e-03\\
Mo96 & 4.048117e-03\\
Mo97 & 2.341911e-03\\
Mo98 & 5.978344e-03\\
Mo100 & 2.434680e-03\\
Ta181 & 9.710372e-05\\

\caption{Table showing the isotopic description of material EPP2L}
\label{table:material_EPP2L}
\end{longtable}\clearpage

\begin{longtable}[ht!]             
{ p{0.3\textwidth} | p{0.3\textwidth} } 
\hline
Nuclide & Mass Fraction\\
\hline
H1 & 1.087490e-02\\
H2 & 2.499587e-06\\
B10 & 3.078251e-07\\
B11 & 1.362328e-06\\
C12 & 8.848623e-04\\
N14 & 3.061224e-04\\
N15 & 1.223539e-06\\
O16 & 8.609641e-02\\
Si28 & 7.529100e-03\\
Si29 & 3.959642e-04\\
Si30 & 2.700051e-04\\
P31 & 3.393288e-04\\
S & 2.375087e-04\\
Ti46 & 1.322772e-05\\
Ti47 & 1.218838e-05\\
Ti48 & 1.232805e-04\\
Ti49 & 9.239568e-06\\
Ti50 & 9.026949e-06\\
V & 1.269429e-04\\
Cr50 & 6.434173e-03\\
Cr52 & 1.290321e-01\\
Cr53 & 1.491283e-02\\
Cr54 & 3.782086e-03\\
Mn55 & 1.290535e-02\\
Fe54 & 3.869015e-02\\
Fe56 & 6.298195e-01\\
Fe57 & 1.480546e-02\\
Fe58 & 2.004854e-03\\
Co59 & 5.467260e-04\\
Ni58 & 2.246924e-02\\
Ni60 & 8.953234e-03\\
Ni61 & 3.956860e-04\\
Ni62 & 1.282270e-03\\
Ni64 & 3.371005e-04\\
Cu63 & 3.432127e-04\\
Cu65 & 1.578320e-04\\
Nb93 & 2.743787e-04\\
Mo92 & 7.966813e-04\\
Mo94 & 5.073792e-04\\
Mo95 & 8.825497e-04\\
Mo96 & 9.344134e-04\\
Mo97 & 5.405780e-04\\
Mo98 & 1.379961e-03\\
Mo100 & 5.619871e-04\\
Ta181 & 1.670144e-05\\
\caption{Table showing the isotopic description of material M162}
\label{table:material_M162}
\end{longtable}\clearpage                          

\end{centering}


\section{Appendix - B - Groupwise Photon Sources}
\subsection{Baseline}
\subsubsection{Decay Time 1 - 1$\times$10$^{5}$ seconds}
\input{../plots/final_model_nob4c/src_dc1.tex}
\subsubsection{Decay Time 2 - 1$\times$10$^{6}$ seconds}
0.00E0-1.00E-2
\begin{figure}[ht!]
\centering
\includegraphics[trim={0cm 9cm 0cm 10cm},clip,scale=0.75]{../plots/final_model_nob4c/{Photon_Source_Density_Decay_Time_2_(Energy_Range:_0.00E0-1.00E-2_MeV)}.png}
\label{fig:source_dc$2_no4bc_0.00E0-1.00E-2}
\caption{The shutdown photon source for the case with B$_4$C for decay time 2 in the energy ranges 0.00E0-1.00E-2 MeV}
\end{figure}
1.00E-2-2.00E-2
\begin{figure}[ht!]
\centering
\includegraphics[trim={0cm 9cm 0cm 10cm},clip,scale=0.75]{../plots/final_model_nob4c/{Photon_Source_Density_Decay_Time_2_(Energy_Range:_1.00E-2-2.00E-2_MeV)}.png}
\label{fig:source_dc$2_no4bc_1.00E-2-2.00E-2}
\caption{The shutdown photon source for the case with B$_4$C for decay time 2 in the energy ranges 1.00E-2-2.00E-2 MeV}
\end{figure}
2.00E-2-5.00E-2
\begin{figure}[ht!]
\centering
\includegraphics[trim={0cm 9cm 0cm 10cm},clip,scale=0.75]{../plots/final_model_nob4c/{Photon_Source_Density_Decay_Time_2_(Energy_Range:_2.00E-2-5.00E-2_MeV)}.png}
\label{fig:source_dc$2_no4bc_2.00E-2-5.00E-2}
\caption{The shutdown photon source for the case with B$_4$C for decay time 2 in the energy ranges 2.00E-2-5.00E-2 MeV}
\end{figure}
5.00E-2-1.00E-1
\begin{figure}[ht!]
\centering
\includegraphics[trim={0cm 9cm 0cm 10cm},clip,scale=0.75]{../plots/final_model_nob4c/{Photon_Source_Density_Decay_Time_2_(Energy_Range:_5.00E-2-1.00E-1_MeV)}.png}
\label{fig:source_dc$2_no4bc_5.00E-2-1.00E-1}
\caption{The shutdown photon source for the case with B$_4$C for decay time 2 in the energy ranges 5.00E-2-1.00E-1 MeV}
\end{figure}
1.00E-1-2.00E-1
\begin{figure}[ht!]
\centering
\includegraphics[trim={0cm 9cm 0cm 10cm},clip,scale=0.75]{../plots/final_model_nob4c/{Photon_Source_Density_Decay_Time_2_(Energy_Range:_1.00E-1-2.00E-1_MeV)}.png}
\label{fig:source_dc$2_no4bc_1.00E-1-2.00E-1}
\caption{The shutdown photon source for the case with B$_4$C for decay time 2 in the energy ranges 1.00E-1-2.00E-1 MeV}
\end{figure}
2.00E-1-3.00E-1
\begin{figure}[ht!]
\centering
\includegraphics[trim={0cm 9cm 0cm 10cm},clip,scale=0.75]{../plots/final_model_nob4c/{Photon_Source_Density_Decay_Time_2_(Energy_Range:_2.00E-1-3.00E-1_MeV)}.png}
\label{fig:source_dc$2_no4bc_2.00E-1-3.00E-1}
\caption{The shutdown photon source for the case with B$_4$C for decay time 2 in the energy ranges 2.00E-1-3.00E-1 MeV}
\end{figure}
3.00E-1-4.00E-1
\begin{figure}[ht!]
\centering
\includegraphics[trim={0cm 9cm 0cm 10cm},clip,scale=0.75]{../plots/final_model_nob4c/{Photon_Source_Density_Decay_Time_2_(Energy_Range:_3.00E-1-4.00E-1_MeV)}.png}
\label{fig:source_dc$2_no4bc_3.00E-1-4.00E-1}
\caption{The shutdown photon source for the case with B$_4$C for decay time 2 in the energy ranges 3.00E-1-4.00E-1 MeV}
\end{figure}
4.00E-1-6.00E-1
\begin{figure}[ht!]
\centering
\includegraphics[trim={0cm 9cm 0cm 10cm},clip,scale=0.75]{../plots/final_model_nob4c/{Photon_Source_Density_Decay_Time_2_(Energy_Range:_4.00E-1-6.00E-1_MeV)}.png}
\label{fig:source_dc$2_no4bc_4.00E-1-6.00E-1}
\caption{The shutdown photon source for the case with B$_4$C for decay time 2 in the energy ranges 4.00E-1-6.00E-1 MeV}
\end{figure}
6.00E-1-8.00E-1
\begin{figure}[ht!]
\centering
\includegraphics[trim={0cm 9cm 0cm 10cm},clip,scale=0.75]{../plots/final_model_nob4c/{Photon_Source_Density_Decay_Time_2_(Energy_Range:_6.00E-1-8.00E-1_MeV)}.png}
\label{fig:source_dc$2_no4bc_6.00E-1-8.00E-1}
\caption{The shutdown photon source for the case with B$_4$C for decay time 2 in the energy ranges 6.00E-1-8.00E-1 MeV}
\end{figure}
8.00E-1-1.00E0
\begin{figure}[ht!]
\centering
\includegraphics[trim={0cm 9cm 0cm 10cm},clip,scale=0.75]{../plots/final_model_nob4c/{Photon_Source_Density_Decay_Time_2_(Energy_Range:_8.00E-1-1.00E0_MeV)}.png}
\label{fig:source_dc$2_no4bc_8.00E-1-1.00E0}
\caption{The shutdown photon source for the case with B$_4$C for decay time 2 in the energy ranges 8.00E-1-1.00E0 MeV}
\end{figure}
1.00E0-1.22E0
\begin{figure}[ht!]
\centering
\includegraphics[trim={0cm 9cm 0cm 10cm},clip,scale=0.75]{../plots/final_model_nob4c/{Photon_Source_Density_Decay_Time_2_(Energy_Range:_1.00E0-1.22E0_MeV)}.png}
\label{fig:source_dc$2_no4bc_1.00E0-1.22E0}
\caption{The shutdown photon source for the case with B$_4$C for decay time 2 in the energy ranges 1.00E0-1.22E0 MeV}
\end{figure}
1.22E0-1.44E0
\begin{figure}[ht!]
\centering
\includegraphics[trim={0cm 9cm 0cm 10cm},clip,scale=0.75]{../plots/final_model_nob4c/{Photon_Source_Density_Decay_Time_2_(Energy_Range:_1.22E0-1.44E0_MeV)}.png}
\label{fig:source_dc$2_no4bc_1.22E0-1.44E0}
\caption{The shutdown photon source for the case with B$_4$C for decay time 2 in the energy ranges 1.22E0-1.44E0 MeV}
\end{figure}
1.44E0-1.66E0
\begin{figure}[ht!]
\centering
\includegraphics[trim={0cm 9cm 0cm 10cm},clip,scale=0.75]{../plots/final_model_nob4c/{Photon_Source_Density_Decay_Time_2_(Energy_Range:_1.44E0-1.66E0_MeV)}.png}
\label{fig:source_dc$2_no4bc_1.44E0-1.66E0}
\caption{The shutdown photon source for the case with B$_4$C for decay time 2 in the energy ranges 1.44E0-1.66E0 MeV}
\end{figure}
1.66E0-2.00E0
\begin{figure}[ht!]
\centering
\includegraphics[trim={0cm 9cm 0cm 10cm},clip,scale=0.75]{../plots/final_model_nob4c/{Photon_Source_Density_Decay_Time_2_(Energy_Range:_1.66E0-2.00E0_MeV)}.png}
\label{fig:source_dc$2_no4bc_1.66E0-2.00E0}
\caption{The shutdown photon source for the case with B$_4$C for decay time 2 in the energy ranges 1.66E0-2.00E0 MeV}
\end{figure}
2.00E0-2.50E0
\begin{figure}[ht!]
\centering
\includegraphics[trim={0cm 9cm 0cm 10cm},clip,scale=0.75]{../plots/final_model_nob4c/{Photon_Source_Density_Decay_Time_2_(Energy_Range:_2.00E0-2.50E0_MeV)}.png}
\label{fig:source_dc$2_no4bc_2.00E0-2.50E0}
\caption{The shutdown photon source for the case with B$_4$C for decay time 2 in the energy ranges 2.00E0-2.50E0 MeV}
\end{figure}
2.50E0-3.00E0
\begin{figure}[ht!]
\centering
\includegraphics[trim={0cm 9cm 0cm 10cm},clip,scale=0.75]{../plots/final_model_nob4c/{Photon_Source_Density_Decay_Time_2_(Energy_Range:_2.50E0-3.00E0_MeV)}.png}
\label{fig:source_dc$2_no4bc_2.50E0-3.00E0}
\caption{The shutdown photon source for the case with B$_4$C for decay time 2 in the energy ranges 2.50E0-3.00E0 MeV}
\end{figure}
3.00E0-4.00E0
\begin{figure}[ht!]
\centering
\includegraphics[trim={0cm 9cm 0cm 10cm},clip,scale=0.75]{../plots/final_model_nob4c/{Photon_Source_Density_Decay_Time_2_(Energy_Range:_3.00E0-4.00E0_MeV)}.png}
\label{fig:source_dc$2_no4bc_3.00E0-4.00E0}
\caption{The shutdown photon source for the case with B$_4$C for decay time 2 in the energy ranges 3.00E0-4.00E0 MeV}
\end{figure}
4.00E0-5.00E0
\begin{figure}[ht!]
\centering
\includegraphics[trim={0cm 9cm 0cm 10cm},clip,scale=0.75]{../plots/final_model_nob4c/{Photon_Source_Density_Decay_Time_2_(Energy_Range:_4.00E0-5.00E0_MeV)}.png}
\label{fig:source_dc$2_no4bc_4.00E0-5.00E0}
\caption{The shutdown photon source for the case with B$_4$C for decay time 2 in the energy ranges 4.00E0-5.00E0 MeV}
\end{figure}
5.00E0-6.50E0
\begin{figure}[ht!]
\centering
\includegraphics[trim={0cm 9cm 0cm 10cm},clip,scale=0.75]{../plots/final_model_nob4c/{Photon_Source_Density_Decay_Time_2_(Energy_Range:_5.00E0-6.50E0_MeV)}.png}
\label{fig:source_dc$2_no4bc_5.00E0-6.50E0}
\caption{The shutdown photon source for the case with B$_4$C for decay time 2 in the energy ranges 5.00E0-6.50E0 MeV}
\end{figure}
6.50E0-8.00E0
\begin{figure}[ht!]
\centering
\includegraphics[trim={0cm 9cm 0cm 10cm},clip,scale=0.75]{../plots/final_model_nob4c/{Photon_Source_Density_Decay_Time_2_(Energy_Range:_6.50E0-8.00E0_MeV)}.png}
\label{fig:source_dc$2_no4bc_6.50E0-8.00E0}
\caption{The shutdown photon source for the case with B$_4$C for decay time 2 in the energy ranges 6.50E0-8.00E0 MeV}
\end{figure}
8.00E0-1.00E1
\begin{figure}[ht!]
\centering
\includegraphics[trim={0cm 9cm 0cm 10cm},clip,scale=0.75]{../plots/final_model_nob4c/{Photon_Source_Density_Decay_Time_2_(Energy_Range:_8.00E0-1.00E1_MeV)}.png}
\label{fig:source_dc$2_no4bc_8.00E0-1.00E1}
\caption{The shutdown photon source for the case with B$_4$C for decay time 2 in the energy ranges 8.00E0-1.00E1 MeV}
\end{figure}
1.00E1-1.20E1
\begin{figure}[ht!]
\centering
\includegraphics[trim={0cm 9cm 0cm 10cm},clip,scale=0.75]{../plots/final_model_nob4c/{Photon_Source_Density_Decay_Time_2_(Energy_Range:_1.00E1-1.20E1_MeV)}.png}
\label{fig:source_dc$2_no4bc_1.00E1-1.20E1}
\caption{The shutdown photon source for the case with B$_4$C for decay time 2 in the energy ranges 1.00E1-1.20E1 MeV}
\end{figure}
1.20E1-1.40E1
\begin{figure}[ht!]
\centering
\includegraphics[trim={0cm 9cm 0cm 10cm},clip,scale=0.75]{../plots/final_model_nob4c/{Photon_Source_Density_Decay_Time_2_(Energy_Range:_1.20E1-1.40E1_MeV)}.png}
\label{fig:source_dc$2_no4bc_1.20E1-1.40E1}
\caption{The shutdown photon source for the case with B$_4$C for decay time 2 in the energy ranges 1.20E1-1.40E1 MeV}
\end{figure}
1.40E1-2.00E1
\begin{figure}[ht!]
\centering
\includegraphics[trim={0cm 9cm 0cm 10cm},clip,scale=0.75]{../plots/final_model_nob4c/{Photon_Source_Density_Decay_Time_2_(Energy_Range:_1.40E1-2.00E1_MeV)}.png}
\label{fig:source_dc$2_no4bc_1.40E1-2.00E1}
\caption{The shutdown photon source for the case with B$_4$C for decay time 2 in the energy ranges 1.40E1-2.00E1 MeV}
\end{figure}
1.40E1-2.00E1
\begin{figure}[ht!]
\centering
\includegraphics[trim={0cm 9cm 0cm 10cm},clip,scale=0.75]{../plots/final_model_nob4c/{Photon_Source_Density_Decay_Time_2_(Energy_Range:_1.40E1-2.00E1_MeV)}.png}
\label{fig:source_dc$2_no4bc_1.40E1-2.00E1}
\caption{The shutdown photon source for the case with B$_4$C for decay time 2 in the energy ranges 1.40E1-2.00E1 MeV}
\end{figure}
\begin{figure}[ht!]
\centering
\includegraphics[trim={0cm 9cm 0cm 10cm},clip,scale=0.75]{../plots/final_model_nob4c/{Photon_Source_Density_Decay_Time_2_All_Energy_Groups}.png}
\label{fig:source_dc2_no4bc}
\caption{The total shutdown photon source for the case with B$_4$C for decay time 2}
\end{figure}

\subsubsection{Decay Time 3 - 1$\times$10$^{7}$ seconds}
\input{../plots/final_model_nob4c/src_dc3.tex}
\subsection{Including the B4C Liner}
\subsubsection{Decay Time 1 - 1$\times$10$^{5}$ seconds}
\input{../plots/final_model/src_dc1.tex}
\subsubsection{Decay Time 2 - 1$\times$10$^{6}$ seconds}
0.00E0-1.00E-2
\begin{figure}[ht!]
\centering
\includegraphics[trim={0cm 9cm 0cm 10cm},clip,scale=0.75]{../plots/final_model_nob4c/{Photon_Source_Density_Decay_Time_2_(Energy_Range:_0.00E0-1.00E-2_MeV)}.png}
\label{fig:source_dc$2_no4bc_0.00E0-1.00E-2}
\caption{The shutdown photon source for the case with B$_4$C for decay time 2 in the energy ranges 0.00E0-1.00E-2 MeV}
\end{figure}
1.00E-2-2.00E-2
\begin{figure}[ht!]
\centering
\includegraphics[trim={0cm 9cm 0cm 10cm},clip,scale=0.75]{../plots/final_model_nob4c/{Photon_Source_Density_Decay_Time_2_(Energy_Range:_1.00E-2-2.00E-2_MeV)}.png}
\label{fig:source_dc$2_no4bc_1.00E-2-2.00E-2}
\caption{The shutdown photon source for the case with B$_4$C for decay time 2 in the energy ranges 1.00E-2-2.00E-2 MeV}
\end{figure}
2.00E-2-5.00E-2
\begin{figure}[ht!]
\centering
\includegraphics[trim={0cm 9cm 0cm 10cm},clip,scale=0.75]{../plots/final_model_nob4c/{Photon_Source_Density_Decay_Time_2_(Energy_Range:_2.00E-2-5.00E-2_MeV)}.png}
\label{fig:source_dc$2_no4bc_2.00E-2-5.00E-2}
\caption{The shutdown photon source for the case with B$_4$C for decay time 2 in the energy ranges 2.00E-2-5.00E-2 MeV}
\end{figure}
5.00E-2-1.00E-1
\begin{figure}[ht!]
\centering
\includegraphics[trim={0cm 9cm 0cm 10cm},clip,scale=0.75]{../plots/final_model_nob4c/{Photon_Source_Density_Decay_Time_2_(Energy_Range:_5.00E-2-1.00E-1_MeV)}.png}
\label{fig:source_dc$2_no4bc_5.00E-2-1.00E-1}
\caption{The shutdown photon source for the case with B$_4$C for decay time 2 in the energy ranges 5.00E-2-1.00E-1 MeV}
\end{figure}
1.00E-1-2.00E-1
\begin{figure}[ht!]
\centering
\includegraphics[trim={0cm 9cm 0cm 10cm},clip,scale=0.75]{../plots/final_model_nob4c/{Photon_Source_Density_Decay_Time_2_(Energy_Range:_1.00E-1-2.00E-1_MeV)}.png}
\label{fig:source_dc$2_no4bc_1.00E-1-2.00E-1}
\caption{The shutdown photon source for the case with B$_4$C for decay time 2 in the energy ranges 1.00E-1-2.00E-1 MeV}
\end{figure}
2.00E-1-3.00E-1
\begin{figure}[ht!]
\centering
\includegraphics[trim={0cm 9cm 0cm 10cm},clip,scale=0.75]{../plots/final_model_nob4c/{Photon_Source_Density_Decay_Time_2_(Energy_Range:_2.00E-1-3.00E-1_MeV)}.png}
\label{fig:source_dc$2_no4bc_2.00E-1-3.00E-1}
\caption{The shutdown photon source for the case with B$_4$C for decay time 2 in the energy ranges 2.00E-1-3.00E-1 MeV}
\end{figure}
3.00E-1-4.00E-1
\begin{figure}[ht!]
\centering
\includegraphics[trim={0cm 9cm 0cm 10cm},clip,scale=0.75]{../plots/final_model_nob4c/{Photon_Source_Density_Decay_Time_2_(Energy_Range:_3.00E-1-4.00E-1_MeV)}.png}
\label{fig:source_dc$2_no4bc_3.00E-1-4.00E-1}
\caption{The shutdown photon source for the case with B$_4$C for decay time 2 in the energy ranges 3.00E-1-4.00E-1 MeV}
\end{figure}
4.00E-1-6.00E-1
\begin{figure}[ht!]
\centering
\includegraphics[trim={0cm 9cm 0cm 10cm},clip,scale=0.75]{../plots/final_model_nob4c/{Photon_Source_Density_Decay_Time_2_(Energy_Range:_4.00E-1-6.00E-1_MeV)}.png}
\label{fig:source_dc$2_no4bc_4.00E-1-6.00E-1}
\caption{The shutdown photon source for the case with B$_4$C for decay time 2 in the energy ranges 4.00E-1-6.00E-1 MeV}
\end{figure}
6.00E-1-8.00E-1
\begin{figure}[ht!]
\centering
\includegraphics[trim={0cm 9cm 0cm 10cm},clip,scale=0.75]{../plots/final_model_nob4c/{Photon_Source_Density_Decay_Time_2_(Energy_Range:_6.00E-1-8.00E-1_MeV)}.png}
\label{fig:source_dc$2_no4bc_6.00E-1-8.00E-1}
\caption{The shutdown photon source for the case with B$_4$C for decay time 2 in the energy ranges 6.00E-1-8.00E-1 MeV}
\end{figure}
8.00E-1-1.00E0
\begin{figure}[ht!]
\centering
\includegraphics[trim={0cm 9cm 0cm 10cm},clip,scale=0.75]{../plots/final_model_nob4c/{Photon_Source_Density_Decay_Time_2_(Energy_Range:_8.00E-1-1.00E0_MeV)}.png}
\label{fig:source_dc$2_no4bc_8.00E-1-1.00E0}
\caption{The shutdown photon source for the case with B$_4$C for decay time 2 in the energy ranges 8.00E-1-1.00E0 MeV}
\end{figure}
1.00E0-1.22E0
\begin{figure}[ht!]
\centering
\includegraphics[trim={0cm 9cm 0cm 10cm},clip,scale=0.75]{../plots/final_model_nob4c/{Photon_Source_Density_Decay_Time_2_(Energy_Range:_1.00E0-1.22E0_MeV)}.png}
\label{fig:source_dc$2_no4bc_1.00E0-1.22E0}
\caption{The shutdown photon source for the case with B$_4$C for decay time 2 in the energy ranges 1.00E0-1.22E0 MeV}
\end{figure}
1.22E0-1.44E0
\begin{figure}[ht!]
\centering
\includegraphics[trim={0cm 9cm 0cm 10cm},clip,scale=0.75]{../plots/final_model_nob4c/{Photon_Source_Density_Decay_Time_2_(Energy_Range:_1.22E0-1.44E0_MeV)}.png}
\label{fig:source_dc$2_no4bc_1.22E0-1.44E0}
\caption{The shutdown photon source for the case with B$_4$C for decay time 2 in the energy ranges 1.22E0-1.44E0 MeV}
\end{figure}
1.44E0-1.66E0
\begin{figure}[ht!]
\centering
\includegraphics[trim={0cm 9cm 0cm 10cm},clip,scale=0.75]{../plots/final_model_nob4c/{Photon_Source_Density_Decay_Time_2_(Energy_Range:_1.44E0-1.66E0_MeV)}.png}
\label{fig:source_dc$2_no4bc_1.44E0-1.66E0}
\caption{The shutdown photon source for the case with B$_4$C for decay time 2 in the energy ranges 1.44E0-1.66E0 MeV}
\end{figure}
1.66E0-2.00E0
\begin{figure}[ht!]
\centering
\includegraphics[trim={0cm 9cm 0cm 10cm},clip,scale=0.75]{../plots/final_model_nob4c/{Photon_Source_Density_Decay_Time_2_(Energy_Range:_1.66E0-2.00E0_MeV)}.png}
\label{fig:source_dc$2_no4bc_1.66E0-2.00E0}
\caption{The shutdown photon source for the case with B$_4$C for decay time 2 in the energy ranges 1.66E0-2.00E0 MeV}
\end{figure}
2.00E0-2.50E0
\begin{figure}[ht!]
\centering
\includegraphics[trim={0cm 9cm 0cm 10cm},clip,scale=0.75]{../plots/final_model_nob4c/{Photon_Source_Density_Decay_Time_2_(Energy_Range:_2.00E0-2.50E0_MeV)}.png}
\label{fig:source_dc$2_no4bc_2.00E0-2.50E0}
\caption{The shutdown photon source for the case with B$_4$C for decay time 2 in the energy ranges 2.00E0-2.50E0 MeV}
\end{figure}
2.50E0-3.00E0
\begin{figure}[ht!]
\centering
\includegraphics[trim={0cm 9cm 0cm 10cm},clip,scale=0.75]{../plots/final_model_nob4c/{Photon_Source_Density_Decay_Time_2_(Energy_Range:_2.50E0-3.00E0_MeV)}.png}
\label{fig:source_dc$2_no4bc_2.50E0-3.00E0}
\caption{The shutdown photon source for the case with B$_4$C for decay time 2 in the energy ranges 2.50E0-3.00E0 MeV}
\end{figure}
3.00E0-4.00E0
\begin{figure}[ht!]
\centering
\includegraphics[trim={0cm 9cm 0cm 10cm},clip,scale=0.75]{../plots/final_model_nob4c/{Photon_Source_Density_Decay_Time_2_(Energy_Range:_3.00E0-4.00E0_MeV)}.png}
\label{fig:source_dc$2_no4bc_3.00E0-4.00E0}
\caption{The shutdown photon source for the case with B$_4$C for decay time 2 in the energy ranges 3.00E0-4.00E0 MeV}
\end{figure}
4.00E0-5.00E0
\begin{figure}[ht!]
\centering
\includegraphics[trim={0cm 9cm 0cm 10cm},clip,scale=0.75]{../plots/final_model_nob4c/{Photon_Source_Density_Decay_Time_2_(Energy_Range:_4.00E0-5.00E0_MeV)}.png}
\label{fig:source_dc$2_no4bc_4.00E0-5.00E0}
\caption{The shutdown photon source for the case with B$_4$C for decay time 2 in the energy ranges 4.00E0-5.00E0 MeV}
\end{figure}
5.00E0-6.50E0
\begin{figure}[ht!]
\centering
\includegraphics[trim={0cm 9cm 0cm 10cm},clip,scale=0.75]{../plots/final_model_nob4c/{Photon_Source_Density_Decay_Time_2_(Energy_Range:_5.00E0-6.50E0_MeV)}.png}
\label{fig:source_dc$2_no4bc_5.00E0-6.50E0}
\caption{The shutdown photon source for the case with B$_4$C for decay time 2 in the energy ranges 5.00E0-6.50E0 MeV}
\end{figure}
6.50E0-8.00E0
\begin{figure}[ht!]
\centering
\includegraphics[trim={0cm 9cm 0cm 10cm},clip,scale=0.75]{../plots/final_model_nob4c/{Photon_Source_Density_Decay_Time_2_(Energy_Range:_6.50E0-8.00E0_MeV)}.png}
\label{fig:source_dc$2_no4bc_6.50E0-8.00E0}
\caption{The shutdown photon source for the case with B$_4$C for decay time 2 in the energy ranges 6.50E0-8.00E0 MeV}
\end{figure}
8.00E0-1.00E1
\begin{figure}[ht!]
\centering
\includegraphics[trim={0cm 9cm 0cm 10cm},clip,scale=0.75]{../plots/final_model_nob4c/{Photon_Source_Density_Decay_Time_2_(Energy_Range:_8.00E0-1.00E1_MeV)}.png}
\label{fig:source_dc$2_no4bc_8.00E0-1.00E1}
\caption{The shutdown photon source for the case with B$_4$C for decay time 2 in the energy ranges 8.00E0-1.00E1 MeV}
\end{figure}
1.00E1-1.20E1
\begin{figure}[ht!]
\centering
\includegraphics[trim={0cm 9cm 0cm 10cm},clip,scale=0.75]{../plots/final_model_nob4c/{Photon_Source_Density_Decay_Time_2_(Energy_Range:_1.00E1-1.20E1_MeV)}.png}
\label{fig:source_dc$2_no4bc_1.00E1-1.20E1}
\caption{The shutdown photon source for the case with B$_4$C for decay time 2 in the energy ranges 1.00E1-1.20E1 MeV}
\end{figure}
1.20E1-1.40E1
\begin{figure}[ht!]
\centering
\includegraphics[trim={0cm 9cm 0cm 10cm},clip,scale=0.75]{../plots/final_model_nob4c/{Photon_Source_Density_Decay_Time_2_(Energy_Range:_1.20E1-1.40E1_MeV)}.png}
\label{fig:source_dc$2_no4bc_1.20E1-1.40E1}
\caption{The shutdown photon source for the case with B$_4$C for decay time 2 in the energy ranges 1.20E1-1.40E1 MeV}
\end{figure}
1.40E1-2.00E1
\begin{figure}[ht!]
\centering
\includegraphics[trim={0cm 9cm 0cm 10cm},clip,scale=0.75]{../plots/final_model_nob4c/{Photon_Source_Density_Decay_Time_2_(Energy_Range:_1.40E1-2.00E1_MeV)}.png}
\label{fig:source_dc$2_no4bc_1.40E1-2.00E1}
\caption{The shutdown photon source for the case with B$_4$C for decay time 2 in the energy ranges 1.40E1-2.00E1 MeV}
\end{figure}
1.40E1-2.00E1
\begin{figure}[ht!]
\centering
\includegraphics[trim={0cm 9cm 0cm 10cm},clip,scale=0.75]{../plots/final_model_nob4c/{Photon_Source_Density_Decay_Time_2_(Energy_Range:_1.40E1-2.00E1_MeV)}.png}
\label{fig:source_dc$2_no4bc_1.40E1-2.00E1}
\caption{The shutdown photon source for the case with B$_4$C for decay time 2 in the energy ranges 1.40E1-2.00E1 MeV}
\end{figure}
\begin{figure}[ht!]
\centering
\includegraphics[trim={0cm 9cm 0cm 10cm},clip,scale=0.75]{../plots/final_model_nob4c/{Photon_Source_Density_Decay_Time_2_All_Energy_Groups}.png}
\label{fig:source_dc2_no4bc}
\caption{The total shutdown photon source for the case with B$_4$C for decay time 2}
\end{figure}

\subsubsection{Decay Time 3 - 1$\times$10$^{7}$ seconds}
\input{../plots/final_model/src_dc3.tex}

\section{Appendix - C - Shutdown Photon Dose Rates}
\subsection{Baseline}
\subsubsection{Decay time 1 - 1.$\times$10$^5$ s}
\input{../plots/final_model_nob4c/dc1.tex}
\clearpage
\subsubsection{Decay time 2 - 1.$\times$10$^6$ s}
\begin{figure}[ht!]
\centering
\includegraphics[trim={0cm 9cm 0cm 10cm},clip,scale=0.75]{../plots/final_model_nob4c/Photon_Dose_Rate_Decay_Time_2_Mesh_1.png}
\label{fig:photons_dc2_no4bc_m1_flux}
\caption{The total photon dose from the B$_4$C case for decay time 2 for mesh 1}
\end{figure}
\begin{figure}[ht!]
\centering
\includegraphics[trim={0cm 9cm 0cm 10cm},clip,scale=0.75]{../plots/final_model_nob4c/Photon_Dose_Relative_Error_Decay_Time_2_Mesh_1.png}
\label{fig:photons_dc2_no4bc_m1_error}
\caption{The error in photon dose from the B$_4$C case for decay time 2 for mesh 1}
\end{figure}
\begin{figure}[ht!]
\centering
\includegraphics[trim={0cm 9cm 0cm 10cm},clip,scale=0.75]{../plots/final_model_nob4c/Photon_Dose_Rate_Decay_Time_2_Mesh_2.png}
\label{fig:photons_dc2_no4bc_m2_flux}
\caption{The total photon dose from the B$_4$C case for decay time 2 for mesh 2}
\end{figure}
\begin{figure}[ht!]
\centering
\includegraphics[trim={0cm 9cm 0cm 10cm},clip,scale=0.75]{../plots/final_model_nob4c/Photon_Dose_Relative_Error_Decay_Time_2_Mesh_2.png}
\label{fig:photons_dc2_no4bc_m2_error}
\caption{The error in photon dose from the B$_4$C case for decay time 2 for mesh 2}
\end{figure}
\begin{figure}[ht!]
\centering
\includegraphics[trim={0cm 9cm 0cm 10cm},clip,scale=0.75]{../plots/final_model_nob4c/Photon_Dose_Rate_Decay_Time_2_Mesh_3.png}
\label{fig:photons_dc2_no4bc_m3_flux}
\caption{The total photon dose from the B$_4$C case for decay time 2 for mesh 3}
\end{figure}
\begin{figure}[ht!]
\centering
\includegraphics[trim={0cm 9cm 0cm 10cm},clip,scale=0.75]{../plots/final_model_nob4c/Photon_Dose_Relative_Error_Decay_Time_2_Mesh_3.png}
\label{fig:photons_dc2_no4bc_m3_error}
\caption{The error in photon dose from the B$_4$C case for decay time 2 for mesh 3}
\end{figure}
\begin{figure}[ht!]
\centering
\includegraphics[trim={0cm 9cm 0cm 10cm},clip,scale=0.75]{../plots/final_model_nob4c/Photon_Dose_Rate_Decay_Time_2_Mesh_4.png}
\label{fig:photons_dc2_no4bc_m4_flux}
\caption{The total photon dose from the B$_4$C case for decay time 2 for mesh 4}
\end{figure}
\begin{figure}[ht!]
\centering
\includegraphics[trim={0cm 9cm 0cm 10cm},clip,scale=0.75]{../plots/final_model_nob4c/Photon_Dose_Relative_Error_Decay_Time_2_Mesh_4.png}
\label{fig:photons_dc2_no4bc_m4_error}
\caption{The error in photon dose from the B$_4$C case for decay time 2 for mesh 4}
\end{figure}
\begin{figure}[ht!]
\centering
\includegraphics[trim={0cm 9cm 0cm 10cm},clip,scale=0.75]{../plots/final_model_nob4c/Photon_Dose_Rate_Decay_Time_2_Mesh_5.png}
\label{fig:photons_dc2_no4bc_m5_flux}
\caption{The total photon dose from the B$_4$C case for decay time 2 for mesh 5}
\end{figure}
\begin{figure}[ht!]
\centering
\includegraphics[trim={0cm 9cm 0cm 10cm},clip,scale=0.75]{../plots/final_model_nob4c/Photon_Dose_Relative_Error_Decay_Time_2_Mesh_5.png}
\label{fig:photons_dc2_no4bc_m5_error}
\caption{The error in photon dose from the B$_4$C case for decay time 2 for mesh 5}
\end{figure}
\begin{figure}[ht!]
\centering
\includegraphics[trim={0cm 9cm 0cm 10cm},clip,scale=0.75]{../plots/final_model_nob4c/Photon_Dose_Rate_Decay_Time_2_Mesh_6.png}
\label{fig:photons_dc2_no4bc_m6_flux}
\caption{The total photon dose from the B$_4$C case for decay time 2 for mesh 6}
\end{figure}
\begin{figure}[ht!]
\centering
\includegraphics[trim={0cm 9cm 0cm 10cm},clip,scale=0.75]{../plots/final_model_nob4c/Photon_Dose_Relative_Error_Decay_Time_2_Mesh_6.png}
\label{fig:photons_dc2_no4bc_m6_error}
\caption{The error in photon dose from the B$_4$C case for decay time 2 for mesh 6}
\end{figure}
\begin{figure}[ht!]
\centering
\includegraphics[trim={0cm 9cm 0cm 10cm},clip,scale=0.75]{../plots/final_model_nob4c/Photon_Dose_Rate_Decay_Time_2_Mesh_7.png}
\label{fig:photons_dc2_no4bc_m7_flux}
\caption{The total photon dose from the B$_4$C case for decay time 2 for mesh 7}
\end{figure}
\begin{figure}[ht!]
\centering
\includegraphics[trim={0cm 9cm 0cm 10cm},clip,scale=0.75]{../plots/final_model_nob4c/Photon_Dose_Relative_Error_Decay_Time_2_Mesh_7.png}
\label{fig:photons_dc2_no4bc_m7_error}
\caption{The error in photon dose from the B$_4$C case for decay time 2 for mesh 7}
\end{figure}

\clearpage
\subsubsection{Decay time 3 - 1.$\times$10$^7$ s}
\input{../plots/final_model_nob4c/dc3.tex}
\clearpage
\subsection{Including the B4C Liner}
\subsubsection{Decay time 1 - 1.$\times$10$^5$ s}
\input{../plots/final_model/dc1.tex}
\clearpage
\subsubsection{Decay time 2 - 1.$\times$10$^6$ s}
\begin{figure}[ht!]
\centering
\includegraphics[trim={0cm 9cm 0cm 10cm},clip,scale=0.75]{../plots/final_model_nob4c/Photon_Dose_Rate_Decay_Time_2_Mesh_1.png}
\label{fig:photons_dc2_no4bc_m1_flux}
\caption{The total photon dose from the B$_4$C case for decay time 2 for mesh 1}
\end{figure}
\begin{figure}[ht!]
\centering
\includegraphics[trim={0cm 9cm 0cm 10cm},clip,scale=0.75]{../plots/final_model_nob4c/Photon_Dose_Relative_Error_Decay_Time_2_Mesh_1.png}
\label{fig:photons_dc2_no4bc_m1_error}
\caption{The error in photon dose from the B$_4$C case for decay time 2 for mesh 1}
\end{figure}
\begin{figure}[ht!]
\centering
\includegraphics[trim={0cm 9cm 0cm 10cm},clip,scale=0.75]{../plots/final_model_nob4c/Photon_Dose_Rate_Decay_Time_2_Mesh_2.png}
\label{fig:photons_dc2_no4bc_m2_flux}
\caption{The total photon dose from the B$_4$C case for decay time 2 for mesh 2}
\end{figure}
\begin{figure}[ht!]
\centering
\includegraphics[trim={0cm 9cm 0cm 10cm},clip,scale=0.75]{../plots/final_model_nob4c/Photon_Dose_Relative_Error_Decay_Time_2_Mesh_2.png}
\label{fig:photons_dc2_no4bc_m2_error}
\caption{The error in photon dose from the B$_4$C case for decay time 2 for mesh 2}
\end{figure}
\begin{figure}[ht!]
\centering
\includegraphics[trim={0cm 9cm 0cm 10cm},clip,scale=0.75]{../plots/final_model_nob4c/Photon_Dose_Rate_Decay_Time_2_Mesh_3.png}
\label{fig:photons_dc2_no4bc_m3_flux}
\caption{The total photon dose from the B$_4$C case for decay time 2 for mesh 3}
\end{figure}
\begin{figure}[ht!]
\centering
\includegraphics[trim={0cm 9cm 0cm 10cm},clip,scale=0.75]{../plots/final_model_nob4c/Photon_Dose_Relative_Error_Decay_Time_2_Mesh_3.png}
\label{fig:photons_dc2_no4bc_m3_error}
\caption{The error in photon dose from the B$_4$C case for decay time 2 for mesh 3}
\end{figure}
\begin{figure}[ht!]
\centering
\includegraphics[trim={0cm 9cm 0cm 10cm},clip,scale=0.75]{../plots/final_model_nob4c/Photon_Dose_Rate_Decay_Time_2_Mesh_4.png}
\label{fig:photons_dc2_no4bc_m4_flux}
\caption{The total photon dose from the B$_4$C case for decay time 2 for mesh 4}
\end{figure}
\begin{figure}[ht!]
\centering
\includegraphics[trim={0cm 9cm 0cm 10cm},clip,scale=0.75]{../plots/final_model_nob4c/Photon_Dose_Relative_Error_Decay_Time_2_Mesh_4.png}
\label{fig:photons_dc2_no4bc_m4_error}
\caption{The error in photon dose from the B$_4$C case for decay time 2 for mesh 4}
\end{figure}
\begin{figure}[ht!]
\centering
\includegraphics[trim={0cm 9cm 0cm 10cm},clip,scale=0.75]{../plots/final_model_nob4c/Photon_Dose_Rate_Decay_Time_2_Mesh_5.png}
\label{fig:photons_dc2_no4bc_m5_flux}
\caption{The total photon dose from the B$_4$C case for decay time 2 for mesh 5}
\end{figure}
\begin{figure}[ht!]
\centering
\includegraphics[trim={0cm 9cm 0cm 10cm},clip,scale=0.75]{../plots/final_model_nob4c/Photon_Dose_Relative_Error_Decay_Time_2_Mesh_5.png}
\label{fig:photons_dc2_no4bc_m5_error}
\caption{The error in photon dose from the B$_4$C case for decay time 2 for mesh 5}
\end{figure}
\begin{figure}[ht!]
\centering
\includegraphics[trim={0cm 9cm 0cm 10cm},clip,scale=0.75]{../plots/final_model_nob4c/Photon_Dose_Rate_Decay_Time_2_Mesh_6.png}
\label{fig:photons_dc2_no4bc_m6_flux}
\caption{The total photon dose from the B$_4$C case for decay time 2 for mesh 6}
\end{figure}
\begin{figure}[ht!]
\centering
\includegraphics[trim={0cm 9cm 0cm 10cm},clip,scale=0.75]{../plots/final_model_nob4c/Photon_Dose_Relative_Error_Decay_Time_2_Mesh_6.png}
\label{fig:photons_dc2_no4bc_m6_error}
\caption{The error in photon dose from the B$_4$C case for decay time 2 for mesh 6}
\end{figure}
\begin{figure}[ht!]
\centering
\includegraphics[trim={0cm 9cm 0cm 10cm},clip,scale=0.75]{../plots/final_model_nob4c/Photon_Dose_Rate_Decay_Time_2_Mesh_7.png}
\label{fig:photons_dc2_no4bc_m7_flux}
\caption{The total photon dose from the B$_4$C case for decay time 2 for mesh 7}
\end{figure}
\begin{figure}[ht!]
\centering
\includegraphics[trim={0cm 9cm 0cm 10cm},clip,scale=0.75]{../plots/final_model_nob4c/Photon_Dose_Relative_Error_Decay_Time_2_Mesh_7.png}
\label{fig:photons_dc2_no4bc_m7_error}
\caption{The error in photon dose from the B$_4$C case for decay time 2 for mesh 7}
\end{figure}

\clearpage
\subsubsection{Decay time 3 - 1.$\times$10$^7$ s}
\input{../plots/final_model/dc3.tex}
\clearpage


\end{document}
This is never printed
